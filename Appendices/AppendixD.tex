% Appendix Template
\setstretch{1.3}
\chapter{NCVA Maple Worksheet for Temporal Tweezing} % Main appendix title
\label{AppendixD} % Change X to a consecutive letter; for referencing thI_{\sigma} appendix elsewhere, use \ref{AppendixX}
\lhead{Appendix A. \emph{NCVA Maple Worksheet}}
\setstretch{2}

The following NCVA Maple Worksheet was used to generate all the results in this dissertation for the Lugiato-Lefever model with phase-modulation.  The following also uses code generation for Matlab and Latex.  The resulting ODEs and integrals are explicitly written in Appendix~\ref{AppendixA} and~\ref{AppendixB}, respectively.

%\begin{landscape}
%%% Created by Maple 16.00, Mac OS X
%% Source Worksheet: 6p_GaussianAnsatzB.mw
%% Generated: Tue May 31 16:17:55 PDT 2016
%\documentclass{article}
%\usepackage{maplestd2e}
%\def\emptyline{\vspace{12pt}}
%\begin{document}
\pagestyle{empty}
\DefineParaStyle{Maple Heading 1}
\DefineParaStyle{Maple Text Output}
\DefineParaStyle{Maple Dash Item}
\DefineParaStyle{Maple Bullet Item}
\DefineParaStyle{Maple Normal}
\DefineParaStyle{Maple Heading 4}
\DefineParaStyle{Maple Heading 3}
\DefineParaStyle{Maple Heading 2}
\DefineParaStyle{Maple Warning}
\DefineParaStyle{Maple Title}
\DefineParaStyle{Maple Error}
\DefineCharStyle{Maple Hyperlink}
\DefineCharStyle{Maple 2D Math}
\DefineCharStyle{Maple Maple Input}
\DefineCharStyle{Maple 2D Output}
\DefineCharStyle{Maple 2D Input}
\begin{maplegroup}
\begin{center}
\begin{Maple Title}{
\textbf{NCVA for LLE (Temporal Tweezing) - 6-parameter Gaussian Ansatz + Phi}}\end{Maple Title}
\end{center}
\end{maplegroup}
\begin{Maple Normal}{
\begin{Maple Normal}{
Ansatz of the form u(x,t) = A(t,z) * exp(I*theta(t,z)) - z is fast time (tau is slow time) -}\end{Maple Normal}
}\end{Maple Normal}
\begin{maplegroup}
\begin{mapleinput}
\mapleinline{active}{1d}{restart;
interface(showassumed=0): assume(Ur, real): assume(Ui, real): }{}
\end{mapleinput}
\end{maplegroup}
\begin{maplegroup}
\begin{mapleinput}
\mapleinline{active}{2d}{Ustar := Ur+I*Ui; -1; UStarc := Ur-I*Ui; -1; assume(u0, complex); -1}{\[\]}
\end{mapleinput}
\end{maplegroup}
\begin{maplegroup}
\textbf{Ansatz of the form u(x,t) = A(x,t) * exp(I*theta(x,t))  z-> t, t-> x}
\textbf{A := a(z)*exp(-(t-xi(z))\symbol{94}2/(2*(s(z))\symbol{94}2)):}\end{maplegroup}
\begin{maplegroup}
\begin{mapleinput}
\mapleinline{active}{2d}{theta := b(z)+c(z)*(t-xi(z))+d(z)*(t-xi(z))^2; -1}{\[\]}
\end{mapleinput}
\end{maplegroup}
\begin{maplegroup}
\begin{mapleinput}
\mapleinline{active}{2d}{phi := alpha*exp(-(t-tau0)^2/(2*beta^2)); -1}{\[\]}
\end{mapleinput}
\end{maplegroup}
\begin{maplegroup}
\begin{mapleinput}
\mapleinline{active}{2d}{phidot1 := diff(phi, t); -1}{\[\]}
\end{mapleinput}
\end{maplegroup}
\mapleinline{inert}{2d}{}{\[\displaystyle \]}
\begin{maplegroup}
\begin{mapleinput}
\mapleinline{active}{1d}{V:= Delta + phidot1\symbol{94}2 - 2*abs(u0)\symbol{94}2:
}{}
\end{mapleinput}
\end{maplegroup}
\begin{maplegroup}
\textbf{New Lagrangian:}
\end{maplegroup}
\begin{maplegroup}
\begin{mapleinput}
\mapleinline{active}{2d}{LA := A^2*(diff(theta, z))+(diff(A, t))^2+A^2*(diff(theta, t))^2-(1/2)*A^4+Delta*A^2+phidot1^2*A^2-2*abs(u0)^2*A^2+2*(diff(phi, t))*A^2*(diff(theta, t)); -1}{\[\]}
\end{mapleinput}
\end{maplegroup}
\begin{maplegroup}
\begin{mapleinput}
\mapleinline{active}{1d}{sub1 := diff(a(z),z)=ap,diff(s(z),z)=sp,diff(c(z),z)=cp,diff(xi(z),z)=xip,diff(d(z),z)=dp,diff(b(z),z)=bp:
sub2 := xi(z)=xi,a(z)=a,c(z)=c,s(z)=s,d(z)=d, b(z)=b:
LA2 := subs(\{sub1,sub2\},LA):}{}
\end{mapleinput}
\end{maplegroup}
\begin{maplegroup}
\textbf{Leff = integral of Lag, we need to assume that a>0 and xi real to be able to evaluate integrals}
\textbf{assume(s>0): assume(xi,real):  assume(alpha, real): assume(beta>0): asume(s,real): assume(d,real): assume(c,real); assume(b, real):}
\textbf{Leff := int(LA2,t=-infinity..infinity):}\end{maplegroup}
\begin{maplegroup}
\begin{mapleinput}
\mapleinline{active}{1d}{uAs := subs(\{sub1,sub2\},A*exp(I*theta)):    
}{}
\end{mapleinput}
\end{maplegroup}
\begin{maplegroup}
\begin{mapleinput}
\mapleinline{active}{2d}{uAsC := subs({sub1, sub2}, A*exp(-I*theta)); -1}{\[\]}
\end{mapleinput}
\end{maplegroup}
\begin{maplegroup}
\begin{mapleinput}
\mapleinline{active}{2d}{AA := subs({sub1, sub2}, A); -1; theta2 := subs({sub1, sub2}, theta); -1; phi2 := subs({sub1, sub2}, phi); -1}{\[\]}
\end{mapleinput}
\end{maplegroup}
\begin{maplegroup}
\begin{mapleinput}
\mapleinline{active}{2d}{phidot := subs({sub1, sub2}, diff(phi, t)); -1; phidot2 := subs({sub1, sub2}, diff(phi, `$`(t, 2))); -1}{\[\]}
\end{mapleinput}
\end{maplegroup}
\begin{maplegroup}
\begin{mapleinput}
\mapleinline{active}{2d}{ut := subs({sub1, sub2}, diff(A*exp(I*theta), t)); -1; utC := subs({sub1, sub2}, diff(A*exp(-I*theta), t)); -1}{\[\]}
\end{mapleinput}
\end{maplegroup}
\begin{maplegroup}
\begin{mapleinput}
\mapleinline{active}{2d}{duda := diff(uAs, a); -1; dudac := diff(uAsC, a); -1}{\[\]}
\end{mapleinput}
\end{maplegroup}
\begin{maplegroup}
\begin{mapleinput}
\mapleinline{active}{1d}{dudb := diff(uAs,b): dudbc := diff(uAsC,b):
}{}
\end{mapleinput}
\end{maplegroup}
\begin{maplegroup}
\begin{mapleinput}
\mapleinline{active}{2d}{dudd := diff(uAs, d); -1; duddc := diff(uAsC, d); -1}{\[\]}
\end{mapleinput}
\end{maplegroup}
\begin{maplegroup}
\begin{mapleinput}
\mapleinline{active}{1d}{dudc := diff(uAs,c): dudcc := diff(uAsC,c):
}{}
\end{mapleinput}
\end{maplegroup}
\begin{maplegroup}
\begin{mapleinput}
\mapleinline{active}{1d}{duds := diff(uAs, s): dudsc := diff(uAsC, s):
}{}
\end{mapleinput}
\end{maplegroup}
\begin{maplegroup}
\begin{mapleinput}
\mapleinline{active}{1d}{dudxi := diff(uAs,xi): dudxic := diff(uAsC,xi):  
}{}
\end{mapleinput}
\end{maplegroup}
\begin{maplegroup}
\begin{mapleinput}
\mapleinline{active}{1d}{pertRa := evalc(int(-I*(1+phidot2)*AA\symbol{94}2/a + I*(1+phidot2)*AA\symbol{94}2/a,t=-infinity..infinity)):
}{}
\end{mapleinput}
\end{maplegroup}
\begin{maplegroup}
\begin{mapleinput}
\mapleinline{active}{2d}{FIa := (phidot^2-phidot2-3*AA^2)*AA*(2*Ur*cos(theta2)+2*Ui*sin(theta2))/a-AA^2*(2*Ur^2*cos(2*theta2)-2*Ui^2*cos(2*theta2)+4*Ui*Ur*sin(2*theta2))/a; -1}{\[\]}
\end{mapleinput}
\end{maplegroup}
\begin{maplegroup}
\begin{mapleinput}
\mapleinline{active}{2d}{pertRs := evalc(int(-I*(1+phidot2)*AA^2*(t-xi)/s^2+I*(1+phidot2)*AA^2*(t-xi)/s^2, t = -infinity .. infinity)); -1}{\[\]}
\end{mapleinput}
\end{maplegroup}
\begin{maplegroup}
\begin{mapleinput}
\mapleinline{active}{2d}{FIs := (phidot^2-phidot2-3*AA^2)*AA*(t-xi)^2*(2*Ur*cos(theta2)+2*Ui*sin(theta2))/s^3-AA^2*(t-xi)^2*(2*Ur^2*cos(2*theta2)-2*Ui^2*cos(2*theta2)+4*Ui*Ur*sin(2*theta2))/s^3; -1}{\[\]}
\end{mapleinput}
\end{maplegroup}
\begin{maplegroup}
\begin{mapleinput}
\mapleinline{active}{1d}{pertRc := int(((-I-I*phidot2)*uAs*dudcc + (I+I*phidot2)*uAsC*dudc, t=-infinity..infinity)): 
}{}
\end{mapleinput}
\end{maplegroup}
\begin{maplegroup}
\begin{mapleinput}
\mapleinline{active}{2d}{FIc := (phidot^2-phidot2-AA^2)*AA*(t-xi)*(2*Ur*cos(theta2)-2*Ui*sin(theta2))-AA^2*(t-xi)*(2*Ui^2*sin(2*theta2)-2*Ur^2*sin(2*theta2)+4*Ui*Ur*cos(2*theta2)); -1}{\[\]}
\end{mapleinput}
\end{maplegroup}
\begin{maplegroup}
\begin{mapleinput}
\mapleinline{active}{1d}{pertRd := int(simplify((-I-I*phidot2)*uAs*duddc  + (I+I*phidot2)*uAsC*dudd), t=-infinity..infinity):
}{}
\end{mapleinput}
\end{maplegroup}
\begin{maplegroup}
\begin{mapleinput}
\mapleinline{active}{2d}{FId := (phidot^2-phidot2-AA^2)*AA*(t-xi)^2*(2*Ur*cos(theta2)-2*Ui*sin(theta2))-AA^2*(t-xi)^2*(2*Ui^2*sin(2*theta2)-2*Ur^2*sin(2*theta2)+4*Ui*Ur*cos(2*theta2)); -1}{\[\]}
\end{mapleinput}
\end{maplegroup}
\begin{maplegroup}
\begin{mapleinput}
\mapleinline{active}{1d}{pertRxi := int(((-I-I*phidot2)*uAs*dudxic + (I+I*phidot2)*uAsC*dudxi), t=-infinity..infinity): 
}{}
\end{mapleinput}
\end{maplegroup}
\begin{maplegroup}
\begin{mapleinput}
\mapleinline{active}{2d}{FIxi := (phidot^2-phidot2-AA^2)*AA*(-c-2*d*(t-xi))*(2*Ur*cos(theta2)-2*Ui*sin(theta2))-AA^2*(-c-2*d*(t-xi))*(2*Ui^2*sin(2*theta2)-2*Ur^2*sin(2*theta2)+4*Ui*Ur*cos(2*theta2))+(phidot^2-phidot2-3*AA^2)*AA*(t-xi)*(2*Ur*cos(theta2)+2*Ui*sin(theta2))/s^2-AA^2*(t-xi)*(2*Ur^2*cos(2*theta2)-2*Ui^2*cos(2*theta2)+4*Ui*Ur*sin(2*theta2))/s^2; -1}{\[\]}
\end{mapleinput}
\end{maplegroup}
\begin{maplegroup}
\begin{mapleinput}
\mapleinline{active}{2d}{pertRb := int((-I-I*phidot2)*uAs*dudbc+(I+I*phidot2)*uAsC*dudb, t = -infinity .. infinity); -1}{\[\]}
\end{mapleinput}
\end{maplegroup}
\begin{maplegroup}
\begin{mapleinput}
\mapleinline{active}{1d}{FIb := (phidot\symbol{94}2-phidot2-AA\symbol{94}2)*AA*(2*Ur*cos(theta2)-2*Ui*sin(theta2))-AA\symbol{94}2*(2*Ui\symbol{94}2*sin(2*theta2)-2*Ur\symbol{94}2*sin(2*theta2)+4*Ui*Ur*cos(2*theta2)):
}{}
\end{mapleinput}
\end{maplegroup}
\begin{maplegroup}
\begin{mapleinput}
\mapleinline{active}{2d}{}{\[\]}
\end{mapleinput}
\end{maplegroup}
\begin{maplegroup}
\textbf{Put back z-dependences so that we can do Euler-Lag}
\textbf{subi1 := ap=diff(a(z),z),dp=diff(d(z),z),cp=diff(c(z),z),sp=(diff(s(z),z)),bp=(diff(b(z),z)),xip=(diff(xi(z),z)):}
\textbf{subi2 := b=b(z),s = s(z),c=c(z),xi=xi(z),a=a(z),d=d(z):}\end{maplegroup}
\begin{maplegroup}
\begin{mapleinput}
\mapleinline{active}{1d}{sub11 := ap=diff(a(z),z),sp=diff(s(z),z),cp=diff(c(z),z),xip=diff(xi(z),z),dp=diff(d(z),z),bp=diff(b(z),z):
sub22 := d=d(z), b=b(z), xi=xi(z), a=a(z), c=c(z), s=s(z): sub33 := AA = A, theta2 = theta:
}{}
\mapleinline{active}{2d}{}{\[\]}
\end{mapleinput}
\end{maplegroup}
\begin{maplegroup}
\begin{mapleinput}
\mapleinline{active}{2d}{eFIa := subs({subi1, subi2}, FIa); -1; eFIb := subs({sub11, sub22}, FIb); -1; eFIc := subs({sub11, sub22}, FIc); -1; eFId := subs({sub11, sub22}, FId); -1; eFIs := subs({sub11, sub22}, FIs); -1; eFIxi := subs({sub11, sub22}, FIxi); -1}{\[\]}
\end{mapleinput}
\end{maplegroup}
\begin{maplegroup}
\begin{mapleinput}
\mapleinline{active}{2d}{}{\[\]}
\end{mapleinput}
\end{maplegroup}
\begin{maplegroup}
\begin{Maple Normal}{
Euler-Lagrange eqs}\end{Maple Normal}

\textbf{dLsp := subs(\{subi1, subi2\},diff(Leff,sp)):}
\textbf{dLbp := subs(\{subi1, subi2\},diff(Leff,bp)):}
\textbf{dLcp := subs(\{subi1, subi2\},diff(Leff,cp)):}
\textbf{dLdp := subs(\{subi1, subi2\},diff(Leff,dp)):}
\textbf{dLap := subs(\{subi1, subi2\},diff(Leff,ap)):}
\textbf{dLxip:= subs(\{subi1, subi2\},diff(Leff,xip)):}
\textbf{eq1 := diff(dLsp,z) - subs(\{subi1,subi2\},diff(Leff,s)) = subs(\{subi1,subi2\},pertRs+Is):}
\textbf{eq2 := diff(dLbp,z) - subs(\{subi1,subi2\},diff(Leff,b)) = subs(\{subi1,subi2\},pertRb+Ib):}
\textbf{eq3 := diff(dLcp,z) - subs(\{subi1,subi2\},diff(Leff,c)) = subs(\{subi1,subi2\},pertRc+Ic):}
\textbf{eq4 := diff(dLdp,z) - subs(\{subi1,subi2\},diff(Leff,d)) = subs(\{subi1,subi2\},pertRd+Id):}
\textbf{eq5 := diff(dLap,z) - subs(\{subi1,subi2\},diff(Leff,a)) = subs(\{subi1,subi2\},pertRa+Ia):}
\textbf{eq6 := diff(dLxip,z) - subs(\{subi1,subi2\},diff(Leff,xi)) = subs(\{subi1,subi2\},pertRxi+Ixi):}\end{maplegroup}
\begin{maplegroup}
\begin{mapleinput}
\mapleinline{active}{2d}{}{\[\]}
\end{mapleinput}
\end{maplegroup}
\begin{maplegroup}
\begin{Maple Normal}{
Sove Euler-Lag ODEs simultaneously}\end{Maple Normal}

\textbf{sol:= simplify(solve(\{eq1,eq2,eq3,eq4,eq5,eq6\},\{diff(a(z),z),diff(b(z),z),diff(s(z),z),diff(d(z),z),diff(c(z),z),diff(xi(z),z)\})):}
\textbf{eqs := diff(s(z),z)=subs(sol,diff(s(z),z)):}
\textbf{eqb := diff(b(z),z)=subs(sol,diff(b(z),z)):}
\textbf{eqc := diff(c(z),z)=subs(sol,diff(c(z),z)):}
\textbf{eqd := diff(d(z),z)=subs(sol,diff(d(z),z)):}
\textbf{eqa := diff(a(z),z)=subs(sol,diff(a(z),z)):}
\textbf{eqxi := diff(xi(z),z)=subs(sol,diff(xi(z),z)):}\end{maplegroup}
\begin{maplegroup}
\begin{mapleinput}
\mapleinline{active}{1d}{subMatlab := a(z) = x(1), b(z) = x(2), c(z)=x(3), d(z) = x(4), s(z)=x(5), xi(z) = x(6), Delta = k, Ui = imag(uS), Ur=real(uS), alpha = h, beta = sigma_X: 
M1 := subs(\{subMatlab\},(eqa)):  M2 := subs(\{subMatlab\},collect(eqb,\{a(z),s(z)\},recursive)): M3 := subs(\{subMatlab\},(eqc)): M4 := subs(\{subMatlab\},collect(eqd,\{a(z),s(z)\},recursive)): M5 := subs(\{subMatlab\},(eqs)): M6 := subs(\{subMatlab\},(eqxi)): F1 := subs(\{subMatlab\},(eFIa)): F2 := subs(\{subMatlab\},(eFIb)):  F3 := subs(\{subMatlab\},(eFIc)):  F4 := subs(\{subMatlab\},(eFId)):  F5 := subs(\{subMatlab\},(eFIs)):  F6 := subs(\{subMatlab\},(eFIxi)):
with(CodeGeneration):}{}
\end{mapleinput}
\end{maplegroup}
\begin{maplegroup}
\begin{mapleinput}
\mapleinline{active}{1d}{Matlab(M1): Matlab(M2): Matlab(M3): Matlab(M4):Matlab(M5): Matlab(M6):
}{}
\end{mapleinput}
\mapleresult
\underline{}Warning, the function names \{x\} are not recognized in the target language\underline{}\mapleresult
cg = 0.0e0 == (-0.192e3 * x(1) \symbol{94} 2 * x(4) * (x(5) \symbol{94} 7) * sqrt(pi) * sqrt((x(5) \symbol{94} 2 + 2 * sigma\_X \symbol{94} 2)) * (sigma\_X \symbol{94} 4) - 0.256e3 * x(1) \symbol{94} 2 * x(4) * (x(5) \symbol{94} 5) * sqrt(pi) * sqrt((x(5) \symbol{94} 2 + 2 * sigma\_X \symbol{94} 2)) * (sigma\_X \symbol{94} 6) - 0.128e3 * x(1) \symbol{94} 2 * x(4) * (x(5) \symbol{94} 3) * sqrt(pi) * sqrt((x(5) \symbol{94} 2 + 2 * sigma\_X \symbol{94} 2)) * (sigma\_X \symbol{94} 8) - 0.64e2 * x(1) \symbol{94} 2 * x(4) * (x(5) \symbol{94} 9) * sqrt(pi) * sqrt((x(5) \symbol{94} 2 + 2 * sigma\_X \symbol{94} 2)) * (sigma\_X \symbol{94} 2) + 0.160e3 * exp(-((x(6) - tau0) \symbol{94} 2 / (x(5) \symbol{94} 2 + 2 * sigma\_X \symbol{94} 2))) * h * sigma\_X * (x(5) \symbol{94} 7) * x(1) \symbol{94} 2 * sqrt(0.2e1) * sqrt(pi) * tau0 * x(6) + 0.448e3 * exp(-((x(6) - tau0) \symbol{94} 2 / (x(5) \symbol{94} 2 + 2 * sigma\_X \symbol{94} 2))) * h * (sigma\_X \symbol{94} 3) * (x(5) \symbol{94} 5) * x(1) \symbol{94} 2 * sqrt(0.2e1) * sqrt(pi) * tau0 * x(6) + 0.256e3 * exp(-((x(6) - tau0) \symbol{94} 2 /(x(5) \symbol{94} 2 + 2 * sigma\_X \symbol{94} 2))) * h * (sigma\_X \symbol{94} 5) * (x(5) \symbol{94} 3) * x(1) \symbol{94} 2 * sqrt(0.2e1) * sqrt(pi) * tau0 * x(6) - 0.64e2 * sqrt(pi) * (x(5) \symbol{94} 5) * x(1) \symbol{94} 2 * exp(-((x(6) - tau0) \symbol{94} 2 / (x(5) \symbol{94} 2 + 2 * sigma\_X \symbol{94} 2))) * sqrt(0.2e1) * h * tau0 * (x(6) \symbol{94} 3)* sigma\_X - 0.64e2 * sqrt(pi) * (x(5) \symbol{94} 5) * x(1) \symbol{94} 2 * exp(-((x(6) - tau0) \symbol{94} 2 / (x(5) \symbol{94} 2 + 2 * sigma\_X \symbol{94} 2))) * sqrt(0.2e1) * h * (tau0 \symbol{94} 3) * x(6) * sigma\_X + 0.96e2 * sqrt(pi) * (x(5) \symbol{94} 5) * x(1) \symbol{94} 2 * exp(-((x(6) - tau0) \symbol{94} 2 / (x(5) \symbol{94} 2 + 2 * sigma\_X\symbol{94} 2))) * sqrt(0.2e1) * h * (tau0 \symbol{94} 2) * (x(6) \symbol{94} 2) * sigma\_X + 0.28e2 * sigma\_X * (x(5) \symbol{94} 9) * sqrt(pi) * exp(-((x(6) - tau0) \symbol{94} 2 / (x(5) \symbol{94} 2 + 2 * sigma\_X \symbol{94} 2))) * sqrt(0.2e1) * h * x(1) \symbol{94} 2 + 0.144e3 * (sigma\_X \symbol{94} 3) * (x(5) \symbol{94} 7) * sqrt(pi) * exp(-((x(6)- tau0) \symbol{94} 2 / (x(5) \symbol{94} 2 + 2 * sigma\_X \symbol{94} 2))) * sqrt(0.2e1) * h * x(1) \symbol{94} 2 + 0.240e3 * (sigma\_X \symbol{94} 5) * (x(5) \symbol{94} 5) * sqrt(pi) * exp(-((x(6) - tau0) \symbol{94} 2 / (x(5) \symbol{94} 2 + 2 * sigma\_X \symbol{94} 2))) * sqrt(0.2e1) * h * x(1) \symbol{94} 2 + 0.128e3 * (sigma\_X \symbol{94} 7) * (x(5) \symbol{94} 3) * sqrt(pi) * exp(-((x(6) - tau0) \symbol{94} 2 / (x(5) \symbol{94} 2 + 2 * sigma\_X \symbol{94} 2))) * sqrt(0.2e1) * h * x(1) \symbol{94} 2 - 0.2e1 * Id * sqrt((x(5) \symbol{94} 2 + 2 * sigma\_X \symbol{94} 2)) * (x(5) \symbol{94} 8) - 0.32e2 * Id * sqrt((x(5) \symbol{94} 2 + 2 * sigma\_X \symbol{94} 2)) * (sigma\_X \symbol{94} 8) + 0.3e1 * (x(5) \symbol{94} 10) * Ib * sqrt((x(5) \symbol{94} 2 + 2 * sigma\_X \symbol{94} 2)) - 0.4e1 * sqrt(pi) * (x(5) \symbol{94} 11) * x(1) \symbol{94} 2 * sqrt((x(5) \symbol{94} 2 + 2 * sigma\_X \symbol{94} 2)) - 0.64e2 * Id * sqrt((x(5) \symbol{94} 2 + 2 * sigma\_X \symbol{94} 2)) * (x(5) \symbol{94} 2) * (sigma\_X \symbol{94} 6) + 0.96e2 * (x(5) \symbol{94} 4) * Ib * sqrt((x(5) \symbol{94} 2 + 2 * sigma\_X \symbol{94} 2)) * (sigma\_X \symbol{94} 6) + 0.48e2 * (x(5) \symbol{94} 2) * Ib * sqrt((x(5) \symbol{94} 2 + 2 * sigma\_X \symbol{94} 2)) * (sigma\_X \symbol{94} 8) + 0.24e2 * (x(5) \symbol{94} 8) * Ib * sqrt((x(5) \symbol{94} 2 + 2 * sigma\_X \symbol{94} 2)) * (sigma\_X \symbol{94} 2) + 0.72e2 * (x(5) \symbol{94} 6) * Ib * sqrt((x(5) \symbol{94} 2 + 2 * sigma\_X \symbol{94} 2)) * (sigma\_X \symbol{94} 4) - 0.16e2 * Id * sqrt((x(5) \symbol{94} 2 + 2 * sigma\_X \symbol{94} 2)) * (x(5) \symbol{94} 6) * (sigma\_X \symbol{94} 2) - 0.48e2 * Id * sqrt((x(5) \symbol{94} 2 + 2 * sigma\_X \symbol{94} 2)) * (x(5) \symbol{94} 4) * (sigma\_X \symbol{94} 4) - 0.32e2 * (x(5) \symbol{94} 9) * sqrt(pi) * x(1) \symbol{94} 2 * sqrt((x(5) \symbol{94} 2 + 2 * sigma\_X \symbol{94} 2)) * (sigma\_X \symbol{94}2) - 0.96e2 * (x(5) \symbol{94} 7) * sqrt(pi) * x(1) \symbol{94} 2 * sqrt((x(5) \symbol{94} 2 + 2 * sigma\_X \symbol{94} 2)) * (sigma\_X \symbol{94} 4) - 0.128e3 * (x(5) \symbol{94} 5) * sqrt(pi) * x(1) \symbol{94} 2 * sqrt((x(5) \symbol{94} 2 + 2 * sigma\_X \symbol{94} 2)) * (sigma\_X \symbol{94} 6) - 0.64e2 * (x(5) \symbol{94} 3) * sqrt(pi) * x(1) \symbol{94} 2 * sqrt((x(5) \symbol{94}2 + 2 * sigma\_X \symbol{94} 2)) * (sigma\_X \symbol{94} 8) - 0.8e1 * x(1) \symbol{94} 2 * x(4) * (x(5) \symbol{94} 11) * sqrt(pi) * sqrt((x(5) \symbol{94} 2 + 2 * sigma\_X \symbol{94} 2)) - 0.80e2 * exp(-((x(6) - tau0) \symbol{94} 2 / (x(5) \symbol{94} 2 + 2 * sigma\_X \symbol{94} 2))) * h * sigma\_X * (x(5) \symbol{94} 7) * x(1) \symbol{94} 2 * sqrt(0.2e1) * sqrt(pi) * (tau0 \symbol{94} 2) + 0.16e2 * sqrt(pi) * (x(5) \symbol{94} 5) * x(1) \symbol{94} 2 * exp(-((x(6) - tau0) \symbol{94} 2 / (x(5) \symbol{94} 2 + 2 * sigma\_X \symbol{94} 2))) * sqrt(0.2e1) * h * (x(6) \symbol{94} 4) * sigma\_X + 0.16e2 * sqrt(pi) * (x(5) \symbol{94} 5) * x(1) \symbol{94} 2 * exp(-((x(6) - tau0) \symbol{94} 2 / (x(5) \symbol{94} 2 + 2 * sigma\_X \symbol{94}2))) * sqrt(0.2e1) * h * (tau0 \symbol{94} 4) * sigma\_X - 0.224e3 * exp(-((x(6) - tau0) \symbol{94} 2 / (x(5) \symbol{94} 2 + 2 * sigma\_X \symbol{94} 2))) * h * (sigma\_X \symbol{94} 3) * (x(5) \symbol{94} 5) * x(1) \symbol{94} 2 * sqrt(0.2e1) * sqrt(pi) * (tau0 \symbol{94} 2) - 0.224e3 * exp(-((x(6) - tau0) \symbol{94} 2 / (x(5) \symbol{94} 2 + 2 * sigma\_X \symbol{94} 2))) * h * (sigma\_X \symbol{94} 3) * (x(5) \symbol{94} 5) * x(1) \symbol{94} 2 * sqrt(0.2e1) * sqrt(pi) * (x(6) \symbol{94} 2) - 0.128e3 * exp(-((x(6) - tau0) \symbol{94} 2 / (x(5) \symbol{94} 2 + 2 * sigma\_X \symbol{94} 2))) * h * (sigma\_X \symbol{94} 5) * (x(5) \symbol{94} 3) * x(1) \symbol{94} 2 * sqrt(0.2e1) * sqrt(pi) * (tau0 \symbol{94} 2) - 0.80e2 * exp(-((x(6) - tau0) \symbol{94} 2 / (x(5) \symbol{94} 2 + 2 * sigma\_X \symbol{94} 2))) * h * sigma\_X * (x(5) \symbol{94} 7) * x(1) \symbol{94} 2 * sqrt(0.2e1) * sqrt(pi) * (x(6) \symbol{94} 2) - 0.128e3 * exp(-((x(6) - tau0) \symbol{94} 2 / (x(5) \symbol{94} 2 + 2 * sigma\_X \symbol{94} 2))) * h * (sigma\_X \symbol{94} 5) * (x(5) \symbol{94} 3) * x(1) \symbol{94} 2 * sqrt(0.2e1) * sqrt(pi) * (x(6) \symbol{94} 2)) / x(1) / (x(5) \symbol{94} 3) * pi \symbol{94} (-0.1e1 / 0.2e1) * ((x(5) \symbol{94} 2 + 2 * sigma\_X \symbol{94} 2) \symbol{94} (-0.9e1 / 0.2e1)) / 0.4e1;
\underline{}Warning, the function names \{x\} are not recognized in the target language\underline{}\mapleresult
\underline{}Warning, the following variable name replacements were made: ["cg"] = ["u0\symbol{126}"]\underline{}\mapleresult
cg0 = 0.0e0 == (0.5e1 / 0.8e1 * sqrt(0.2e1) / ((x(5) \symbol{94} 2 + sigma\_X \symbol{94} 2) \symbol{94} 4) / ((x(5) \symbol{94} 2 + 2 * sigma\_X \symbol{94} 2) \symbol{94} 4) * (x(5) \symbol{94} 16) + 0.15e2 / 0.2e1 * sqrt(0.2e1) / ((x(5) \symbol{94} 2 + sigma\_X \symbol{94} 2) \symbol{94} 4) * (sigma\_X \symbol{94} 2) / ((x(5) \symbol{94} 2 + 2 * sigma\_X \symbol{94} 2) \symbol{94} 4) * (x(5) \symbol{94} 14) + 0.155e3 / 0.4e1 * sqrt(0.2e1) / ((x(5) \symbol{94} 2 + sigma\_X \symbol{94} 2) \symbol{94} 4) * (sigma\_X \symbol{94} 4) / ((x(5) \symbol{94} 2 + 2 * sigma\_X \symbol{94} 2) \symbol{94} 4) * (x(5) \symbol{94} 12) + 0.225e3 / 0.2e1 * sqrt(0.2e1) / ((x(5) \symbol{94} 2 + sigma\_X \symbol{94} 2) \symbol{94} 4) * (sigma\_X \symbol{94} 6) / ((x(5) \symbol{94} 2 + 2 * sigma\_X \symbol{94} 2) \symbol{94} 4) * (x(5) \symbol{94} 10) + 0.1605e4 / 0.8e1 * sqrt(0.2e1) / ((x(5) \symbol{94} 2 + sigma\_X \symbol{94} 2) \symbol{94} 4) * (sigma\_X \symbol{94} 8) / ((x(5) \symbol{94} 2 + 2 * sigma\_X \symbol{94} 2) \symbol{94} 4) * (x(5) \symbol{94} 8) + 0.225e3 * sqrt(0.2e1) / ((x(5) \symbol{94} 2 + sigma\_X \symbol{94} 2) \symbol{94} 4) * (sigma\_X \symbol{94} 10) / ((x(5) \symbol{94} 2 + 2 * sigma\_X \symbol{94} 2) \symbol{94} 4) *(x(5) \symbol{94} 6) + 0.155e3 * (sigma\_X \symbol{94} 12) * sqrt(0.2e1) / ((x(5) \symbol{94} 2 + sigma\_X \symbol{94} 2) \symbol{94} 4) / ((x(5) \symbol{94} 2 + 2 * sigma\_X \symbol{94} 2) \symbol{94} 4) * (x(5) \symbol{94} 4) + 0.60e2 * (sigma\_X \symbol{94} 14) * sqrt(0.2e1) / ((x(5) \symbol{94} 2 + sigma\_X \symbol{94} 2) \symbol{94} 4) / ((x(5) \symbol{94} 2 + 2 * sigma\_X \symbol{94} 2) \symbol{94} 4) * (x(5) \symbol{94} 2) + 0.10e2 / ((x(5) \symbol{94} 2 + sigma\_X \symbol{94} 2) \symbol{94} 4) / ((x(5) \symbol{94} 2 + 2 * sigma\_X \symbol{94} 2) \symbol{94} 4) * sqrt(0.2e1) * (sigma\_X \symbol{94} 16)) * x(1) \symbol{94} 2 - (0.6e1 * exp(-((x(6) - tau0) \symbol{94} 2 / (x(5) \symbol{94} 2 + sigma\_X \symbol{94} 2))) * h \symbol{94} 2 * sqrt(pi) * sqrt((x(5) \symbol{94} 2 + 2 * sigma\_X \symbol{94} 2)) - 0.8e1 * x(3) \symbol{94} 2 * sqrt(pi) * sqrt((x(5) \symbol{94} 2 + sigma\_X \symbol{94} 2)) * sigma\_X * sqrt((x(5) \symbol{94} 2 + 2 * sigma\_X \symbol{94} 2)) + 0.8e1 * k * sqrt(pi) * sqrt((x(5) \symbol{94} 2 + sigma\_X \symbol{94} 2)) * sigma\_X * sqrt((x(5) \symbol{94} 2 + 2 * sigma\_X \symbol{94} 2)) - 0.48e2 * exp(-((x(6) - tau0) \symbol{94} 2 / (x(5) \symbol{94} 2 + 2 * sigma\_X \symbol{94} 2))) * h * (sigma\_X \symbol{94} 2) * x(4) * sqrt(0.2e1) * sqrt(pi) * sqrt((x(5) \symbol{94} 2 + sigma\_X \symbol{94} 2)) - 0.16e2 * (abs(cg) \symbol{94} 2) * sqrt(pi) * sqrt((x(5) \symbol{94} 2 + sigma\_X \symbol{94} 2)) * sigma\_X * sqrt((x(5) \symbol{94} 2 + 2 * sigma\_X \symbol{94} 2))) * pi \symbol{94} (-0.1e1 / 0.2e1) * ((x(5) \symbol{94} 2 + sigma\_X \symbol{94} 2) \symbol{94} (-0.9e1 / 0.2e1)) / sigma\_X * ((x(5) \symbol{94} 2 + 2 * sigma\_X \symbol{94} 2) \symbol{94} (-0.9e1 / 0.2e1)) * (x(5) \symbol{94} 16) / 0.8e1 - (0.96e2 * k * sqrt(pi) * sqrt((x(5) \symbol{94} 2 + sigma\_X \symbol{94} 2)) * (sigma\_X \symbol{94} 3) * sqrt((x(5) \symbol{94} 2 + 2 * sigma\_X \symbol{94} 2)) - 0.48e2 * exp(-((x(6) - tau0)\symbol{94} 2 / (x(5) \symbol{94} 2 + 2 * sigma\_X \symbol{94} 2))) * h * (sigma\_X \symbol{94} 2) * sqrt(0.2e1) * sqrt(pi) * sqrt((x(5) \symbol{94} 2 + sigma\_X \symbol{94} 2)) * tau0 * x(3) + 0.60e2 * exp(-((x(6) - tau0) \symbol{94} 2 / (x(5) \symbol{94} 2 + sigma\_X \symbol{94} 2))) * h \symbol{94} 2 * sqrt(pi) * sqrt((x(5) \symbol{94} 2 + 2 * sigma\_X \symbol{94} 2)) * (sigma\_X \symbol{94} 2) + 0.8e1 * exp(-((x(6) - tau0) \symbol{94} 2 / (x(5) \symbol{94} 2 + sigma\_X \symbol{94} 2))) * h \symbol{94} 2 * sqrt(pi) * sqrt((x(5) \symbol{94} 2 + 2 * sigma\_X \symbol{94} 2)) * tau0 * x(6) + 0.192e3 * exp(-((x(6) - tau0) \symbol{94} 2 / (x(5) \symbol{94} 2 + 2 * sigma\_X \symbol{94} 2))) * h * (sigma\_X \symbol{94} 2) * sqrt(0.2e1) * sqrt(pi)* sqrt((x(5) \symbol{94} 2 + sigma\_X \symbol{94} 2)) * x(4) * (x(6) \symbol{94} 2) + 0.192e3 * exp(-((x(6) - tau0) \symbol{94} 2 / (x(5) \symbol{94} 2 + 2 * sigma\_X \symbol{94} 2))) * h * (sigma\_X \symbol{94} 2) * sqrt(0.2e1) * sqrt(pi) * sqrt((x(5) \symbol{94} 2 + sigma\_X \symbol{94} 2)) * x(4) * (tau0 \symbol{94} 2) + 0.48e2 * exp(-((x(6) - tau0) \symbol{94} 2/ (x(5) \symbol{94} 2 + 2 * sigma\_X \symbol{94} 2))) * h * (sigma\_X \symbol{94} 2) * sqrt(0.2e1) * sqrt(pi) * sqrt((x(5) \symbol{94} 2 + sigma\_X \symbol{94} 2)) * x(6) * x(3) - 0.96e2 * (sigma\_X \symbol{94} 3) * sqrt((x(5) \symbol{94} 2 + sigma\_X \symbol{94} 2)) * x(3) \symbol{94} 2 * sqrt(pi) * sqrt((x(5) \symbol{94} 2 + 2 * sigma\_X \symbol{94} 2)) - 0.384e3 * exp(-((x(6) - tau0) \symbol{94} 2 / (x(5) \symbol{94} 2 + 2 * sigma\_X \symbol{94} 2))) * h * (sigma\_X \symbol{94} 2) * sqrt(0.2e1) * sqrt(pi) * sqrt((x(5) \symbol{94} 2 + sigma\_X \symbol{94} 2)) * x(6) * x(4) * tau0 - 0.4e1 * exp(-((x(6) - tau0) \symbol{94} 2 / (x(5) \symbol{94} 2 + sigma\_X \symbol{94} 2))) * h \symbol{94} 2 * sqrt(pi) * sqrt((x(5) \symbol{94} 2 + 2* sigma\_X \symbol{94} 2)) * (tau0 \symbol{94} 2) - 0.4e1 * exp(-((x(6) - tau0) \symbol{94} 2 / (x(5) \symbol{94} 2 + sigma\_X \symbol{94} 2))) * h \symbol{94} 2 * sqrt(pi) * sqrt((x(5) \symbol{94} 2 + 2 * sigma\_X \symbol{94} 2)) * (x(6) \symbol{94} 2) - 0.192e3 * (abs(cg) \symbol{94} 2) * sqrt(pi) * sqrt((x(5) \symbol{94} 2 + sigma\_X \symbol{94} 2)) * (sigma\_X \symbol{94} 3) * sqrt((x(5) \symbol{94} 2 + 2 * sigma\_X \symbol{94} 2)) + 0.8e1 * sqrt(pi) * sqrt((x(5) \symbol{94} 2 + sigma\_X \symbol{94} 2)) * sigma\_X * sqrt((x(5) \symbol{94} 2 + 2 * sigma\_X \symbol{94} 2)) - 0.384e3 * exp(-((x(6) - tau0) \symbol{94} 2 / (x(5) \symbol{94} 2 + 2 * sigma\_X \symbol{94} 2))) * h * (sigma\_X \symbol{94} 4) * x(4) * sqrt(0.2e1) * sqrt(pi) * sqrt((x(5) \symbol{94} 2 + sigma\_X \symbol{94} 2))) * pi \symbol{94} (-0.1e1 / 0.2e1) * ((x(5) \symbol{94} 2 + sigma\_X \symbol{94} 2) \symbol{94} (-0.9e1 / 0.2e1)) / sigma\_X * ((x(5) \symbol{94} 2 + 2 * sigma\_X \symbol{94} 2) \symbol{94} (-0.9e1 / 0.2e1)) * (x(5) \symbol{94} 14) / 0.8e1 - (-0.384e3 * exp(-((x(6) - tau0) \symbol{94} 2 / (x(5) \symbol{94} 2 + 2 * sigma\_X \symbol{94} 2))) *h * (sigma\_X \symbol{94} 4) * sqrt(0.2e1) * sqrt(pi) * sqrt((x(5) \symbol{94} 2 + sigma\_X \symbol{94} 2)) * tau0 * x(3) + 0.96e2 * sqrt((x(5) \symbol{94} 2 + sigma\_X \symbol{94} 2)) * (sigma\_X \symbol{94} 2) * x(3) * sqrt(pi) * exp(-((x(6) - tau0) \symbol{94} 2 / (x(5) \symbol{94} 2 + 2 * sigma\_X \symbol{94} 2))) * sqrt(0.2e1) * h * tau0 * (x(6) \symbol{94} 2) + 0.96e2 * sqrt(pi) * sqrt((x(5) \symbol{94} 2 + sigma\_X \symbol{94} 2)) * (sigma\_X \symbol{94} 3) * sqrt((x(5) \symbol{94} 2 + 2 * sigma\_X \symbol{94} 2)) - 0.8e1 * exp(-((x(6) - tau0) \symbol{94} 2 / (x(5) \symbol{94} 2 + sigma\_X \symbol{94} 2))) * h \symbol{94} 2 * sqrt(pi) * sqrt((x(5) \symbol{94} 2 + 2 * sigma\_X \symbol{94} 2)) * (tau0 \symbol{94} 2) * (sigma\_X\symbol{94} 2) + 0.384e3 * exp(-((x(6) - tau0) \symbol{94} 2 / (x(5) \symbol{94} 2 + 2 * sigma\_X \symbol{94} 2))) * h * (sigma\_X \symbol{94} 4) * sqrt(0.2e1) * sqrt(pi) * sqrt((x(5) \symbol{94} 2 + sigma\_X \symbol{94} 2)) * x(6) * x(3) + 0.1152e4 * exp(-((x(6) - tau0) \symbol{94} 2 / (x(5) \symbol{94} 2 + 2 * sigma\_X \symbol{94} 2))) * h * (sigma\_X \symbol{94} 4)* sqrt(0.2e1) * sqrt(pi) * sqrt((x(5) \symbol{94} 2 + sigma\_X \symbol{94} 2)) * x(4) * (tau0 \symbol{94} 2) - 0.384e3 * exp(-((x(6) - tau0) \symbol{94} 2 / (x(5) \symbol{94} 2 + 2 * sigma\_X \symbol{94} 2))) * h * (sigma\_X \symbol{94} 2) * sqrt(0.2e1) * sqrt(pi) * sqrt((x(5) \symbol{94} 2 + sigma\_X \symbol{94} 2)) * (tau0 \symbol{94} 2) * (x(6) \symbol{94} 2) * x(4) + 0.256e3 * exp(-((x(6) - tau0) \symbol{94} 2 / (x(5) \symbol{94} 2 + 2 * sigma\_X \symbol{94} 2))) * h * (sigma\_X \symbol{94} 2) * sqrt(0.2e1) * sqrt(pi) * sqrt((x(5) \symbol{94} 2 + sigma\_X \symbol{94} 2)) * (x(6) \symbol{94} 3) * x(4) * tau0 + 0.32e2 * sqrt((x(5) \symbol{94} 2 + sigma\_X \symbol{94} 2)) * (sigma\_X \symbol{94} 2) * x(3) * sqrt(pi) * exp(-((x(6) - tau0) \symbol{94} 2 / (x(5) \symbol{94} 2 + 2 * sigma\_X \symbol{94} 2))) * sqrt(0.2e1) * h * (tau0 \symbol{94} 3) - 0.1248e4 * exp(-((x(6) - tau0) \symbol{94} 2 / (x(5) \symbol{94} 2 + 2 * sigma\_X \symbol{94} 2))) * h * (sigma\_X \symbol{94} 6) * x(4) * sqrt(0.2e1) * sqrt(pi) * sqrt((x(5) \symbol{94} 2 + sigma\_X \symbol{94} 2)) + 0.256e3 * exp(-((x(6) - tau0) \symbol{94} 2 / (x(5) \symbol{94} 2 + 2 * sigma\_X \symbol{94} 2))) * h * (sigma\_X \symbol{94} 2) * sqrt(0.2e1) * sqrt(pi) * sqrt((x(5) \symbol{94} 2 + sigma\_X \symbol{94} 2)) * (tau0 \symbol{94} 3) * x(6) * x(4) + 0.16e2 * sqrt((x(5) \symbol{94} 2 + 2 * sigma\_X \symbol{94} 2)) * exp(-((x(6) - tau0) \symbol{94} 2 / (x(5) \symbol{94} 2 + sigma\_X \symbol{94} 2))) * h \symbol{94} 2 * (sigma\_X \symbol{94} 2) * sqrt(pi) * tau0 * x(6) - 0.496e3 * (sigma\_X \symbol{94} 5) * x(3) \symbol{94} 2 * sqrt(pi) * sqrt((x(5) \symbol{94} 2 + sigma\_X \symbol{94} 2)) * sqrt((x(5) \symbol{94} 2 + 2 * sigma\_X \symbol{94} 2)) + 0.1152e4 * exp(-((x(6) - tau0) \symbol{94} 2 / (x(5) \symbol{94} 2 + 2 * sigma\_X \symbol{94} 2))) * h * (sigma\_X\symbol{94} 4) * sqrt(0.2e1) * sqrt(pi) * sqrt((x(5) \symbol{94} 2 + sigma\_X \symbol{94} 2)) * x(4) * (x(6) \symbol{94} 2) - 0.64e2 * exp(-((x(6) - tau0) \symbol{94} 2 / (x(5) \symbol{94} 2 + 2 * sigma\_X \symbol{94} 2))) * h * (sigma\_X \symbol{94} 2) * sqrt(0.2e1) * sqrt(pi) * sqrt((x(5) \symbol{94} 2 + sigma\_X \symbol{94} 2)) * (x(6) \symbol{94} 4) * x(4) - 0.96e2 * sqrt((x(5) \symbol{94} 2 + sigma\_X \symbol{94} 2)) * (sigma\_X \symbol{94} 2) * x(3) * sqrt(pi) * exp(-((x(6) - tau0) \symbol{94} 2 / (x(5) \symbol{94} 2 + 2 * sigma\_X \symbol{94} 2))) * sqrt(0.2e1) * h * (tau0 \symbol{94} 2) * x(6) - 0.64e2 * exp(-((x(6) - tau0) \symbol{94} 2 / (x(5) \symbol{94} 2 + 2 * sigma\_X \symbol{94} 2))) * h * (sigma\_X \symbol{94} 2) *sqrt(0.2e1) * sqrt(pi) * sqrt((x(5) \symbol{94} 2 + sigma\_X \symbol{94} 2)) * (tau0 \symbol{94} 4) * x(4) - 0.992e3 * (sigma\_X \symbol{94} 5) * (abs(cg) \symbol{94} 2) * sqrt(pi) * sqrt((x(5) \symbol{94} 2 + sigma\_X \symbol{94} 2)) * sqrt((x(5) \symbol{94} 2 + 2 * sigma\_X \symbol{94} 2)) + 0.246e3 * exp(-((x(6) - tau0) \symbol{94} 2 / (x(5) \symbol{94} 2 + sigma\_X\symbol{94} 2))) * h \symbol{94} 2 * sqrt(pi) * sqrt((x(5) \symbol{94} 2 + 2 * sigma\_X \symbol{94} 2)) * (sigma\_X \symbol{94} 4) - 0.8e1 * exp(-((x(6) - tau0) \symbol{94} 2 / (x(5) \symbol{94} 2 + sigma\_X \symbol{94} 2))) * h \symbol{94} 2 * sqrt(pi) * sqrt((x(5) \symbol{94} 2 + 2 * sigma\_X \symbol{94} 2)) * (x(6) \symbol{94} 2) * (sigma\_X \symbol{94} 2) - 0.2304e4 * exp(-((x(6) - tau0) \symbol{94} 2 / (x(5) \symbol{94} 2 + 2 * sigma\_X \symbol{94} 2))) * h * (sigma\_X \symbol{94} 4) * sqrt(0.2e1) * sqrt(pi) * sqrt((x(5) \symbol{94} 2 + sigma\_X \symbol{94} 2)) * x(6) * x(4) * tau0 - 0.32e2 * sqrt((x(5) \symbol{94} 2 + sigma\_X \symbol{94} 2)) * (sigma\_X \symbol{94} 2) * x(3) * sqrt(pi) * exp(-((x(6) - tau0) \symbol{94} 2 / (x(5) \symbol{94} 2 +2 * sigma\_X \symbol{94} 2))) * sqrt(0.2e1) * h * (x(6) \symbol{94} 3) + 0.496e3 * (sigma\_X \symbol{94} 5) * k * sqrt(pi) * sqrt((x(5) \symbol{94} 2 + sigma\_X \symbol{94} 2)) * sqrt((x(5) \symbol{94} 2 + 2 * sigma\_X \symbol{94} 2))) * pi \symbol{94} (-0.1e1 / 0.2e1) * ((x(5) \symbol{94} 2 + sigma\_X \symbol{94} 2) \symbol{94} (-0.9e1 / 0.2e1)) / sigma\_X * ((x(5) \symbol{94}2 + 2 * sigma\_X \symbol{94} 2) \symbol{94} (-0.9e1 / 0.2e1)) * (x(5) \symbol{94} 12) / 0.8e1 - (-0.256e3 * exp(-((x(6) - tau0) \symbol{94} 2 / (x(5) \symbol{94} 2 + 2 * sigma\_X \symbol{94} 2))) * h * (sigma\_X \symbol{94} 4) * sqrt(0.2e1) * sqrt(pi) * sqrt((x(5) \symbol{94} 2 + sigma\_X \symbol{94} 2)) * (tau0 \symbol{94} 4) * x(4) - 0.192e3 * (sigma\_X \symbol{94} 4) * sqrt((x(5) \symbol{94} 2 + sigma\_X \symbol{94} 2)) * x(3) * sqrt(pi) * exp(-((x(6) - tau0) \symbol{94} 2 / (x(5) \symbol{94} 2 + 2 * sigma\_X \symbol{94} 2))) * sqrt(0.2e1) * h * (x(6) \symbol{94} 3) + 0.2688e4 * exp(-((x(6) - tau0) \symbol{94} 2 / (x(5) \symbol{94} 2 + 2 * sigma\_X \symbol{94} 2))) * h * (sigma\_X \symbol{94} 6) * sqrt(0.2e1) * sqrt(pi) * sqrt((x(5) \symbol{94} 2 + sigma\_X \symbol{94} 2)) * x(4) * (tau0 \symbol{94} 2) + 0.192e3 * sqrt((x(5) \symbol{94} 2 + sigma\_X \symbol{94} 2)) * (sigma\_X \symbol{94} 4) * x(3) * sqrt(pi) * exp(-((x(6) - tau0) \symbol{94} 2 / (x(5) \symbol{94} 2 + 2 * sigma\_X \symbol{94} 2))) * sqrt(0.2e1) * h * (tau0 \symbol{94} 3) - 0.1248e4 * (sigma\_X \symbol{94} 6) * exp(-((x(6) - tau0) \symbol{94} 2 / (x(5) \symbol{94} 2 + 2 * sigma\_X \symbol{94} 2))) * h * sqrt(0.2e1) * sqrt(pi) * sqrt((x(5) \symbol{94} 2 + sigma\_X \symbol{94} 2)) * tau0 * x(3) + 0.1248e4 * (sigma\_X \symbol{94} 6) * exp(-((x(6) - tau0) \symbol{94} 2 / (x(5) \symbol{94} 2 + 2 * sigma\_X \symbol{94} 2))) * h * sqrt(0.2e1) * sqrt(pi) * sqrt((x(5)\symbol{94} 2 + sigma\_X \symbol{94} 2)) * x(6) * x(3) + 0.2688e4 * exp(-((x(6) - tau0) \symbol{94} 2 / (x(5) \symbol{94} 2 + 2 * sigma\_X \symbol{94} 2))) * h * (sigma\_X \symbol{94} 6) * sqrt(0.2e1) * sqrt(pi) * sqrt((x(5) \symbol{94} 2 + sigma\_X \symbol{94} 2)) * x(4) * (x(6) \symbol{94} 2) - 0.256e3 * exp(-((x(6) - tau0) \symbol{94} 2 / (x(5) \symbol{94} 2 + 2 *sigma\_X \symbol{94} 2))) * h * (sigma\_X \symbol{94} 4) * sqrt(0.2e1) * sqrt(pi) * sqrt((x(5) \symbol{94} 2 + sigma\_X \symbol{94} 2)) * (x(6) \symbol{94} 4) * x(4) + 0.132e3 * exp(-((x(6) - tau0) \symbol{94} 2 / (x(5) \symbol{94} 2 + sigma\_X \symbol{94} 2))) * h \symbol{94} 2 * sqrt(pi) * sqrt((x(5) \symbol{94} 2 + 2 * sigma\_X \symbol{94} 2)) * (x(6) \symbol{94} 2) * (sigma\_X \symbol{94} 4) + 0.132e3 * exp(-((x(6) - tau0) \symbol{94} 2 / (x(5) \symbol{94} 2 + sigma\_X \symbol{94} 2))) * h \symbol{94} 2 * sqrt(pi) * sqrt((x(5) \symbol{94} 2 + 2 * sigma\_X \symbol{94} 2)) * (tau0 \symbol{94} 2) * (sigma\_X \symbol{94} 4) - 0.8e1 * exp(-((x(6) - tau0) \symbol{94} 2 / (x(5) \symbol{94} 2 + sigma\_X \symbol{94} 2))) * h \symbol{94} 2 * (sigma\_X \symbol{94} 2) * sqrt(pi) *sqrt((x(5) \symbol{94} 2 + 2 * sigma\_X \symbol{94} 2)) * (tau0 \symbol{94} 4) - 0.8e1 * exp(-((x(6) - tau0) \symbol{94} 2 / (x(5) \symbol{94} 2 + sigma\_X \symbol{94} 2))) * h \symbol{94} 2 * (sigma\_X \symbol{94} 2) * sqrt(pi) * sqrt((x(5) \symbol{94} 2 + 2 * sigma\_X \symbol{94} 2)) * (x(6) \symbol{94} 4) + 0.496e3 * sqrt(pi) * sqrt((x(5) \symbol{94} 2 + sigma\_X \symbol{94} 2)) * (sigma\_X \symbol{94} 5) * sqrt((x(5) \symbol{94} 2 + 2 * sigma\_X \symbol{94} 2)) + 0.528e3 * exp(-((x(6) - tau0) \symbol{94} 2 / (x(5) \symbol{94} 2 + sigma\_X \symbol{94} 2))) * h \symbol{94} 2 * sqrt(pi) * sqrt((x(5) \symbol{94} 2 + 2 * sigma\_X \symbol{94} 2)) * (sigma\_X \symbol{94} 6) + 0.1024e4 * exp(-((x(6) - tau0) \symbol{94} 2 / (x(5) \symbol{94} 2 + 2 * sigma\_X \symbol{94} 2))) *h * (sigma\_X \symbol{94} 4) * sqrt(0.2e1) * sqrt(pi) * sqrt((x(5) \symbol{94} 2 + sigma\_X \symbol{94} 2)) * (x(6) \symbol{94} 3) * x(4) * tau0 + 0.576e3 * sqrt((x(5) \symbol{94} 2 + sigma\_X \symbol{94} 2)) * (sigma\_X \symbol{94} 4) * x(3) * sqrt(pi) * exp(-((x(6) - tau0) \symbol{94} 2 / (x(5) \symbol{94} 2 + 2 * sigma\_X \symbol{94} 2))) * sqrt(0.2e1) *h * tau0 * (x(6) \symbol{94} 2) - 0.576e3 * (sigma\_X \symbol{94} 4) * sqrt((x(5) \symbol{94} 2 + sigma\_X \symbol{94} 2)) * x(3) * sqrt(pi) * exp(-((x(6) - tau0) \symbol{94} 2 / (x(5) \symbol{94} 2 + 2 * sigma\_X \symbol{94} 2))) * sqrt(0.2e1) * h * (tau0 \symbol{94} 2) * x(6) + 0.1024e4 * exp(-((x(6) - tau0) \symbol{94} 2 / (x(5) \symbol{94} 2 + 2 * sigma\_X \symbol{94} 2))) * h * (sigma\_X \symbol{94} 4) * sqrt(0.2e1) * sqrt(pi) * sqrt((x(5) \symbol{94} 2 + sigma\_X \symbol{94} 2)) * (tau0 \symbol{94} 3) * x(6) * x(4) - 0.1536e4 * exp(-((x(6) - tau0) \symbol{94} 2 / (x(5) \symbol{94} 2 + 2 * sigma\_X \symbol{94} 2))) * h * (sigma\_X \symbol{94} 4) * sqrt(0.2e1) * sqrt(pi) * sqrt((x(5) \symbol{94} 2 + sigma\_X\symbol{94} 2)) * (tau0 \symbol{94} 2) * (x(6) \symbol{94} 2) * x(4) - 0.5376e4 * exp(-((x(6) - tau0) \symbol{94} 2 / (x(5) \symbol{94} 2 + 2 * sigma\_X \symbol{94} 2))) * h * (sigma\_X \symbol{94} 6) * sqrt(0.2e1) * sqrt(pi) * sqrt((x(5) \symbol{94} 2 + sigma\_X \symbol{94} 2)) * x(6) * x(4) * tau0 + 0.1440e4 * (sigma\_X \symbol{94} 7) * k * sqrt(pi) * sqrt((x(5) \symbol{94} 2 + sigma\_X \symbol{94} 2)) * sqrt((x(5) \symbol{94} 2 + 2 * sigma\_X \symbol{94} 2)) - 0.2880e4 * (sigma\_X \symbol{94} 7) * (abs(cg) \symbol{94} 2) * sqrt(pi) * sqrt((x(5) \symbol{94} 2 + sigma\_X \symbol{94} 2)) * sqrt((x(5) \symbol{94} 2 + 2 * sigma\_X \symbol{94} 2)) - 0.1440e4 * (sigma\_X \symbol{94} 7) * x(3) \symbol{94} 2 * sqrt(pi) * sqrt((x(5) \symbol{94} 2 +sigma\_X \symbol{94} 2)) * sqrt((x(5) \symbol{94} 2 + 2 * sigma\_X \symbol{94} 2)) + 0.32e2 * exp(-((x(6) - tau0) \symbol{94} 2 / (x(5) \symbol{94} 2 + sigma\_X \symbol{94} 2))) * h \symbol{94} 2 * (sigma\_X \symbol{94} 2) * sqrt(pi) * sqrt((x(5) \symbol{94} 2 + 2 * sigma\_X \symbol{94} 2)) * (x(6) \symbol{94} 3) * tau0 - 0.48e2 * exp(-((x(6) - tau0) \symbol{94} 2 / (x(5) \symbol{94} 2 +sigma\_X \symbol{94} 2))) * h \symbol{94} 2 * (sigma\_X \symbol{94} 2) * sqrt(pi) * sqrt((x(5) \symbol{94} 2 + 2 * sigma\_X \symbol{94} 2)) * (tau0 \symbol{94} 2) * (x(6) \symbol{94} 2) + 0.32e2 * exp(-((x(6) - tau0) \symbol{94} 2 / (x(5) \symbol{94} 2 + sigma\_X \symbol{94} 2))) * h \symbol{94} 2 * (sigma\_X \symbol{94} 2) * sqrt(pi) * sqrt((x(5) \symbol{94} 2 + 2 * sigma\_X \symbol{94} 2)) * (tau0 \symbol{94} 3) * x(6) - 0.264e3 * sqrt((x(5) \symbol{94} 2 + 2 * sigma\_X \symbol{94} 2)) * exp(-((x(6) - tau0) \symbol{94} 2 / (x(5) \symbol{94} 2 + sigma\_X \symbol{94} 2))) * h \symbol{94} 2 * (sigma\_X \symbol{94} 4) * sqrt(pi) * tau0 * x(6) - 0.2112e4 * exp(-((x(6) - tau0) \symbol{94} 2 / (x(5) \symbol{94} 2 + 2 * sigma\_X \symbol{94} 2))) * h * (sigma\_X \symbol{94} 8) *x(4) * sqrt(0.2e1) * sqrt(pi) * sqrt((x(5) \symbol{94} 2 + sigma\_X \symbol{94} 2))) * pi \symbol{94} (-0.1e1 / 0.2e1) * ((x(5) \symbol{94} 2 + sigma\_X \symbol{94} 2) \symbol{94} (-0.9e1 / 0.2e1)) / sigma\_X * ((x(5) \symbol{94} 2 + 2 * sigma\_X \symbol{94} 2) \symbol{94} (-0.9e1 / 0.2e1)) * (x(5) \symbol{94} 10) / 0.8e1 - (0.448e3 * sqrt((x(5) \symbol{94} 2 + sigma\_X \symbol{94} 2)) * (sigma\_X \symbol{94} 6) * x(3) * sqrt(pi) * exp(-((x(6) - tau0) \symbol{94} 2 / (x(5) \symbol{94} 2 + 2 * sigma\_X \symbol{94} 2))) * sqrt(0.2e1) * h * (tau0 \symbol{94} 3) + 0.3072e4 * exp(-((x(6) - tau0) \symbol{94} 2 / (x(5) \symbol{94} 2 + 2 * sigma\_X \symbol{94} 2))) * h * (sigma\_X \symbol{94} 8) * sqrt(0.2e1) * sqrt(pi) * sqrt((x(5) \symbol{94} 2 + sigma\_X \symbol{94} 2)) * x(4) * (tau0 \symbol{94} 2) + 0.3072e4 * exp(-((x(6) - tau0) \symbol{94} 2 / (x(5) \symbol{94} 2 + 2 * sigma\_X \symbol{94} 2))) * h * (sigma\_X \symbol{94} 8) * sqrt(0.2e1) * sqrt(pi) * sqrt((x(5) \symbol{94} 2 + sigma\_X \symbol{94} 2)) * x(4) * (x(6) \symbol{94} 2) - 0.384e3 * exp(-((x(6) - tau0) \symbol{94} 2 / (x(5)\symbol{94} 2 + 2 * sigma\_X \symbol{94} 2))) * h * (sigma\_X \symbol{94} 6) * sqrt(0.2e1) * sqrt(pi) * sqrt((x(5) \symbol{94} 2 + sigma\_X \symbol{94} 2)) * (x(6) \symbol{94} 4) * x(4) - 0.384e3 * exp(-((x(6) - tau0) \symbol{94} 2 / (x(5) \symbol{94} 2 + 2 * sigma\_X \symbol{94} 2))) * h * (sigma\_X \symbol{94} 6) * sqrt(0.2e1) * sqrt(pi) * sqrt((x(5) \symbol{94} 2 +sigma\_X \symbol{94} 2)) * (tau0 \symbol{94} 4) * x(4) - 0.448e3 * (sigma\_X \symbol{94} 6) * sqrt((x(5) \symbol{94} 2 + sigma\_X \symbol{94} 2)) * x(3) * sqrt(pi) * exp(-((x(6) - tau0) \symbol{94} 2 / (x(5) \symbol{94} 2 + 2 * sigma\_X \symbol{94} 2))) * sqrt(0.2e1) * h * (x(6) \symbol{94} 3) + 0.2112e4 * (sigma\_X \symbol{94} 8) * exp(-((x(6) - tau0) \symbol{94} 2 /(x(5) \symbol{94} 2 + 2 * sigma\_X \symbol{94} 2))) * h * sqrt(0.2e1) * sqrt(pi) * sqrt((x(5) \symbol{94} 2 + sigma\_X \symbol{94} 2)) * x(6) * x(3) - 0.2112e4 * (sigma\_X \symbol{94} 8) * exp(-((x(6) - tau0) \symbol{94} 2 / (x(5) \symbol{94} 2 + 2 * sigma\_X \symbol{94} 2))) * h * sqrt(0.2e1) * sqrt(pi) * sqrt((x(5) \symbol{94} 2 + sigma\_X \symbol{94} 2))* tau0 * x(3) + 0.744e3 * exp(-((x(6) - tau0) \symbol{94} 2 / (x(5) \symbol{94} 2 + sigma\_X \symbol{94} 2))) * h \symbol{94} 2 * sqrt(pi) * sqrt((x(5) \symbol{94} 2 + 2 * sigma\_X \symbol{94} 2)) * (tau0 \symbol{94} 2) * (sigma\_X \symbol{94} 6) - 0.64e2 * exp(-((x(6) - tau0) \symbol{94} 2 / (x(5) \symbol{94} 2 + sigma\_X \symbol{94} 2))) * h \symbol{94} 2 * (sigma\_X \symbol{94} 4) * sqrt(pi) * sqrt((x(5) \symbol{94} 2 + 2 * sigma\_X \symbol{94} 2)) * (tau0 \symbol{94} 4) - 0.64e2 * exp(-((x(6) - tau0) \symbol{94} 2 / (x(5) \symbol{94} 2 + sigma\_X \symbol{94} 2))) * h \symbol{94} 2 * (sigma\_X \symbol{94} 4) * sqrt(pi) * sqrt((x(5) \symbol{94} 2 + 2 * sigma\_X \symbol{94} 2)) * (x(6) \symbol{94} 4) + 0.744e3 * exp(-((x(6) - tau0) \symbol{94} 2 / (x(5) \symbol{94} 2 +sigma\_X \symbol{94} 2))) * h \symbol{94} 2 * sqrt(pi) * sqrt((x(5) \symbol{94} 2 + 2 * sigma\_X \symbol{94} 2)) * (x(6) \symbol{94} 2) * (sigma\_X \symbol{94} 6) + 0.1440e4 * (sigma\_X \symbol{94} 7) * sqrt(pi) * sqrt((x(5) \symbol{94} 2 + sigma\_X \symbol{94} 2)) * sqrt((x(5) \symbol{94} 2 + 2 * sigma\_X \symbol{94} 2)) - 0.2304e4 * exp(-((x(6) - tau0) \symbol{94} 2 / (x(5) \symbol{94} 2+ 2 * sigma\_X \symbol{94} 2))) * h * (sigma\_X \symbol{94} 6) * sqrt(0.2e1) * sqrt(pi) * sqrt((x(5) \symbol{94} 2 + sigma\_X \symbol{94} 2)) * (tau0 \symbol{94} 2) * (x(6) \symbol{94} 2) * x(4) - 0.6144e4 * exp(-((x(6) - tau0) \symbol{94} 2 / (x(5) \symbol{94} 2 + 2 * sigma\_X \symbol{94} 2))) * h * (sigma\_X \symbol{94} 8) * sqrt(0.2e1) * sqrt(pi) * sqrt((x(5) \symbol{94} 2 + sigma\_X \symbol{94} 2)) * x(6) * x(4) * tau0 + 0.1536e4 * exp(-((x(6) - tau0) \symbol{94} 2 / (x(5) \symbol{94} 2 + 2 * sigma\_X \symbol{94} 2))) * h * (sigma\_X \symbol{94} 6) * sqrt(0.2e1) * sqrt(pi) * sqrt((x(5) \symbol{94} 2 + sigma\_X \symbol{94} 2)) * (x(6) \symbol{94} 3) * x(4) * tau0 - 0.1344e4 * (sigma\_X \symbol{94} 6) * sqrt((x(5) \symbol{94} 2 + sigma\_X \symbol{94} 2)) * x(3) * sqrt(pi) * exp(-((x(6) - tau0) \symbol{94} 2 / (x(5) \symbol{94} 2 + 2 * sigma\_X \symbol{94} 2))) * sqrt(0.2e1) * h * (tau0 \symbol{94} 2) * x(6) + 0.1536e4 * exp(-((x(6) - tau0) \symbol{94} 2 / (x(5) \symbol{94} 2 + 2 * sigma\_X \symbol{94} 2))) * h * (sigma\_X \symbol{94} 6) * sqrt(0.2e1) * sqrt(pi) *sqrt((x(5) \symbol{94} 2 + sigma\_X \symbol{94} 2)) * (tau0 \symbol{94} 3) * x(6) * x(4) + 0.1344e4 * (sigma\_X \symbol{94} 6) * sqrt((x(5) \symbol{94} 2 + sigma\_X \symbol{94} 2)) * x(3) * sqrt(pi) * exp(-((x(6) - tau0) \symbol{94} 2 / (x(5) \symbol{94} 2 + 2 * sigma\_X \symbol{94} 2))) * sqrt(0.2e1) * h * tau0 * (x(6) \symbol{94} 2) + 0.624e3 * exp(-((x(6) - tau0) \symbol{94} 2 / (x(5) \symbol{94} 2 + sigma\_X \symbol{94} 2))) * h \symbol{94} 2 * sqrt(pi) * sqrt((x(5) \symbol{94} 2 + 2 * sigma\_X \symbol{94} 2)) * (sigma\_X \symbol{94} 8) + 0.2568e4 * (sigma\_X \symbol{94} 9) * k * sqrt(pi) * sqrt((x(5) \symbol{94} 2 + sigma\_X \symbol{94} 2)) * sqrt((x(5) \symbol{94} 2 + 2 * sigma\_X \symbol{94} 2)) - 0.5136e4 * (sigma\_X \symbol{94} 9) *(abs(cg) \symbol{94} 2) * sqrt(pi) * sqrt((x(5) \symbol{94} 2 + sigma\_X \symbol{94} 2)) * sqrt((x(5) \symbol{94} 2 + 2 * sigma\_X \symbol{94} 2)) - 0.2568e4 * (sigma\_X \symbol{94} 9) * x(3) \symbol{94} 2 * sqrt(pi) * sqrt((x(5) \symbol{94} 2 + sigma\_X \symbol{94} 2)) * sqrt((x(5) \symbol{94} 2 + 2 * sigma\_X \symbol{94} 2)) - 0.1968e4 * exp(-((x(6) - tau0) \symbol{94} 2 / (x(5) \symbol{94} 2 + 2 * sigma\_X \symbol{94} 2))) * h * (sigma\_X \symbol{94} 10) * x(4) * sqrt(0.2e1) * sqrt(pi) * sqrt((x(5) \symbol{94} 2 + sigma\_X \symbol{94} 2)) - 0.1488e4 * sqrt((x(5) \symbol{94} 2 + 2 * sigma\_X \symbol{94} 2)) * exp(-((x(6) - tau0) \symbol{94} 2 / (x(5) \symbol{94} 2 + sigma\_X \symbol{94} 2))) * h \symbol{94} 2 * (sigma\_X \symbol{94} 6) * sqrt(pi) * tau0 * x(6) - 0.384e3 * exp(-((x(6) - tau0) \symbol{94} 2 / (x(5) \symbol{94} 2 + sigma\_X \symbol{94} 2))) * h \symbol{94} 2 * (sigma\_X \symbol{94} 4) * sqrt(pi) * sqrt((x(5) \symbol{94} 2 + 2 * sigma\_X \symbol{94} 2)) * (tau0 \symbol{94} 2) * (x(6) \symbol{94} 2) + 0.256e3 * exp(-((x(6) - tau0) \symbol{94} 2 / (x(5) \symbol{94} 2 + sigma\_X \symbol{94} 2))) * h \symbol{94} 2 * (sigma\_X\symbol{94} 4) * sqrt(pi) * sqrt((x(5) \symbol{94} 2 + 2 * sigma\_X \symbol{94} 2)) * (tau0 \symbol{94} 3) * x(6) + 0.256e3 * exp(-((x(6) - tau0) \symbol{94} 2 / (x(5) \symbol{94} 2 + sigma\_X \symbol{94} 2))) * h \symbol{94} 2 * (sigma\_X \symbol{94} 4) * sqrt(pi) * sqrt((x(5) \symbol{94} 2 + 2 * sigma\_X \symbol{94} 2)) * (x(6) \symbol{94} 3) * tau0) * pi \symbol{94} (-0.1e1 / 0.2e1)* ((x(5) \symbol{94} 2 + sigma\_X \symbol{94} 2) \symbol{94} (-0.9e1 / 0.2e1)) / sigma\_X * ((x(5) \symbol{94} 2 + 2 * sigma\_X \symbol{94} 2) \symbol{94} (-0.9e1 / 0.2e1)) * (x(5) \symbol{94} 8) / 0.8e1 - (0.512e3 * (sigma\_X \symbol{94} 8) * sqrt((x(5) \symbol{94} 2 + sigma\_X \symbol{94} 2)) * x(3) * sqrt(pi) * exp(-((x(6) - tau0) \symbol{94} 2 / (x(5) \symbol{94} 2 + 2 * sigma\_X \symbol{94} 2))) * sqrt(0.2e1) * h * (tau0 \symbol{94} 3) - 0.256e3 * exp(-((x(6) - tau0) \symbol{94} 2 / (x(5) \symbol{94} 2 + 2 * sigma\_X \symbol{94} 2))) * h * (sigma\_X \symbol{94} 8) * sqrt(0.2e1) * sqrt(pi) * sqrt((x(5) \symbol{94} 2 + sigma\_X \symbol{94} 2)) * (x(6) \symbol{94} 4) * x(4) - 0.256e3 * exp(-((x(6) - tau0) \symbol{94} 2 / (x(5) \symbol{94}2 + 2 * sigma\_X \symbol{94} 2))) * h * (sigma\_X \symbol{94} 8) * sqrt(0.2e1) * sqrt(pi) * sqrt((x(5) \symbol{94} 2 + sigma\_X \symbol{94} 2)) * (tau0 \symbol{94} 4) * x(4) - 0.1968e4 * (sigma\_X \symbol{94} 10) * exp(-((x(6) - tau0) \symbol{94} 2 / (x(5) \symbol{94} 2 + 2 * sigma\_X \symbol{94} 2))) * h * sqrt(0.2e1) * sqrt(pi) * sqrt((x(5) \symbol{94} 2 +sigma\_X \symbol{94} 2)) * tau0 * x(3) - 0.512e3 * (sigma\_X \symbol{94} 8) * sqrt((x(5) \symbol{94} 2 + sigma\_X \symbol{94} 2)) * x(3) * sqrt(pi) * exp(-((x(6) - tau0) \symbol{94} 2 / (x(5) \symbol{94} 2 + 2 * sigma\_X \symbol{94} 2))) * sqrt(0.2e1) * h * (x(6) \symbol{94} 3) + 0.1968e4 * (sigma\_X \symbol{94} 10) * exp(-((x(6) - tau0) \symbol{94} 2 / (x(5)\symbol{94} 2 + 2 * sigma\_X \symbol{94} 2))) * h * sqrt(0.2e1) * sqrt(pi) * sqrt((x(5) \symbol{94} 2 + sigma\_X \symbol{94} 2)) * x(6) * x(3) + 0.1728e4 * exp(-((x(6) - tau0) \symbol{94} 2 / (x(5) \symbol{94} 2 + 2 * sigma\_X \symbol{94} 2))) * h * (sigma\_X \symbol{94} 10) * sqrt(0.2e1) * sqrt(pi) * sqrt((x(5) \symbol{94} 2 + sigma\_X \symbol{94} 2)) * x(4) * (x(6) \symbol{94} 2) + 0.1728e4 * exp(-((x(6) - tau0) \symbol{94} 2 / (x(5) \symbol{94} 2 + 2 * sigma\_X \symbol{94} 2))) * h * (sigma\_X \symbol{94} 10) * sqrt(0.2e1) * sqrt(pi) * sqrt((x(5) \symbol{94} 2 + sigma\_X \symbol{94} 2)) * x(4) * (tau0 \symbol{94} 2) - 0.192e3 * exp(-((x(6) - tau0) \symbol{94} 2 / (x(5) \symbol{94} 2 + sigma\_X \symbol{94} 2))) * h \symbol{94} 2* (sigma\_X \symbol{94} 6) * sqrt(pi) * sqrt((x(5) \symbol{94} 2 + 2 * sigma\_X \symbol{94} 2)) * (tau0 \symbol{94} 4) - 0.192e3 * exp(-((x(6) - tau0) \symbol{94} 2 / (x(5) \symbol{94} 2 + sigma\_X \symbol{94} 2))) * h \symbol{94} 2 * (sigma\_X \symbol{94} 6) * sqrt(pi) * sqrt((x(5) \symbol{94} 2 + 2 * sigma\_X \symbol{94} 2)) * (x(6) \symbol{94} 4) + 0.1632e4 * exp(-((x(6) - tau0) \symbol{94} 2 / (x(5) \symbol{94} 2 + sigma\_X \symbol{94} 2))) * h \symbol{94} 2 * sqrt(pi) * sqrt((x(5) \symbol{94} 2 + 2 * sigma\_X \symbol{94} 2)) * (tau0 \symbol{94} 2) * (sigma\_X \symbol{94} 8) + 0.1632e4 * exp(-((x(6) - tau0) \symbol{94} 2 / (x(5) \symbol{94} 2 + sigma\_X \symbol{94} 2))) * h \symbol{94} 2 * sqrt(pi) * sqrt((x(5) \symbol{94} 2 + 2 * sigma\_X \symbol{94} 2)) * (x(6) \symbol{94} 2) * (sigma\_X \symbol{94} 8) + 0.2568e4 * (sigma\_X \symbol{94} 9) * sqrt(pi) * sqrt((x(5) \symbol{94} 2 + sigma\_X \symbol{94} 2)) * sqrt((x(5) \symbol{94} 2 + 2 * sigma\_X \symbol{94} 2)) - 0.1536e4 * exp(-((x(6) - tau0) \symbol{94} 2 / (x(5) \symbol{94} 2 + 2 * sigma\_X \symbol{94} 2))) * h * (sigma\_X \symbol{94} 8) * sqrt(0.2e1) * sqrt(pi) * sqrt((x(5) \symbol{94}2 + sigma\_X \symbol{94} 2)) * (tau0 \symbol{94} 2) * (x(6) \symbol{94} 2) * x(4) - 0.3456e4 * (sigma\_X \symbol{94} 10) * exp(-((x(6) - tau0) \symbol{94} 2 / (x(5) \symbol{94} 2 + 2 * sigma\_X \symbol{94} 2))) * h * sqrt(0.2e1) * sqrt(pi) * sqrt((x(5) \symbol{94} 2 + sigma\_X \symbol{94} 2)) * x(6) * x(4) * tau0 + 0.1024e4 * exp(-((x(6) - tau0) \symbol{94}2 / (x(5) \symbol{94} 2 + 2 * sigma\_X \symbol{94} 2))) * h * (sigma\_X \symbol{94} 8) * sqrt(0.2e1) * sqrt(pi) * sqrt((x(5) \symbol{94} 2 + sigma\_X \symbol{94} 2)) * (tau0 \symbol{94} 3) * x(6) * x(4) + 0.1024e4 * exp(-((x(6) - tau0) \symbol{94} 2 / (x(5) \symbol{94} 2 + 2 * sigma\_X \symbol{94} 2))) * h * (sigma\_X \symbol{94} 8) * sqrt(0.2e1) * sqrt(pi) *sqrt((x(5) \symbol{94} 2 + sigma\_X \symbol{94} 2)) * (x(6) \symbol{94} 3) * x(4) * tau0 - 0.1536e4 * (sigma\_X \symbol{94} 8) * sqrt((x(5) \symbol{94} 2 + sigma\_X \symbol{94} 2)) * x(3) * sqrt(pi) * exp(-((x(6) - tau0) \symbol{94} 2 / (x(5) \symbol{94} 2 + 2 * sigma\_X \symbol{94} 2))) * sqrt(0.2e1) * h * (tau0 \symbol{94} 2) * x(6) + 0.1536e4 * (sigma\_X\symbol{94} 8) * sqrt((x(5) \symbol{94} 2 + sigma\_X \symbol{94} 2)) * x(3) * sqrt(pi) * exp(-((x(6) - tau0) \symbol{94} 2 / (x(5) \symbol{94} 2 + 2 * sigma\_X \symbol{94} 2))) * sqrt(0.2e1) * h * tau0 * (x(6) \symbol{94} 2) + 0.2880e4 * (sigma\_X \symbol{94} 11) * k * sqrt(pi) * sqrt((x(5) \symbol{94} 2 + sigma\_X \symbol{94} 2)) * sqrt((x(5) \symbol{94} 2 + 2 * sigma\_X \symbol{94} 2)) - 0.5760e4 * (sigma\_X \symbol{94} 11) * (abs(cg) \symbol{94} 2) * sqrt(pi) * sqrt((x(5) \symbol{94} 2 + sigma\_X \symbol{94} 2)) * sqrt((x(5) \symbol{94} 2 + 2 * sigma\_X \symbol{94} 2)) - 0.2880e4 * (sigma\_X \symbol{94} 11) * x(3) \symbol{94} 2 * sqrt(pi) * sqrt((x(5) \symbol{94} 2 + sigma\_X \symbol{94} 2)) * sqrt((x(5) \symbol{94} 2 + 2 * sigma\_X \symbol{94} 2)) +0.384e3 * (sigma\_X \symbol{94} 10) * exp(-((x(6) - tau0) \symbol{94} 2 / (x(5) \symbol{94} 2 + sigma\_X \symbol{94} 2))) * h \symbol{94} 2 * sqrt(pi) * sqrt((x(5) \symbol{94} 2 + 2 * sigma\_X \symbol{94} 2)) + 0.768e3 * exp(-((x(6) - tau0) \symbol{94} 2 / (x(5) \symbol{94} 2 + sigma\_X \symbol{94} 2))) * h \symbol{94} 2 * (sigma\_X \symbol{94} 6) * sqrt(pi) * sqrt((x(5) \symbol{94} 2 +2 * sigma\_X \symbol{94} 2)) * (tau0 \symbol{94} 3) * x(6) + 0.768e3 * exp(-((x(6) - tau0) \symbol{94} 2 / (x(5) \symbol{94} 2 + sigma\_X \symbol{94} 2))) * h \symbol{94} 2 * (sigma\_X \symbol{94} 6) * sqrt(pi) * sqrt((x(5) \symbol{94} 2 + 2 * sigma\_X \symbol{94} 2)) * (x(6) \symbol{94} 3) * tau0 - 0.3264e4 * sqrt((x(5) \symbol{94} 2 + 2 * sigma\_X \symbol{94} 2)) * exp(-((x(6)- tau0) \symbol{94} 2 / (x(5) \symbol{94} 2 + sigma\_X \symbol{94} 2))) * h \symbol{94} 2 * (sigma\_X \symbol{94} 8) * sqrt(pi) * tau0 * x(6) - 0.960e3 * (sigma\_X \symbol{94} 12) * exp(-((x(6) - tau0) \symbol{94} 2 / (x(5) \symbol{94} 2 + 2 * sigma\_X \symbol{94} 2))) * h * x(4) * sqrt(0.2e1) * sqrt(pi) * sqrt((x(5) \symbol{94} 2 + sigma\_X \symbol{94} 2)) - 0.1152e4* exp(-((x(6) - tau0) \symbol{94} 2 / (x(5) \symbol{94} 2 + sigma\_X \symbol{94} 2))) * h \symbol{94} 2 * (sigma\_X \symbol{94} 6) * sqrt(pi) * sqrt((x(5) \symbol{94} 2 + 2 * sigma\_X \symbol{94} 2)) * (tau0 \symbol{94} 2) * (x(6) \symbol{94} 2)) * pi \symbol{94} (-0.1e1 / 0.2e1) * ((x(5) \symbol{94} 2 + sigma\_X \symbol{94} 2) \symbol{94} (-0.9e1 / 0.2e1)) / sigma\_X * ((x(5) \symbol{94} 2 + 2 *sigma\_X \symbol{94} 2) \symbol{94} (-0.9e1 / 0.2e1)) * (x(5) \symbol{94} 6) / 0.8e1 - (0.384e3 * exp(-((x(6) - tau0) \symbol{94} 2 / (x(5) \symbol{94} 2 + 2 * sigma\_X \symbol{94} 2))) * h * (sigma\_X \symbol{94} 12) * sqrt(0.2e1) * sqrt(pi) * sqrt((x(5) \symbol{94} 2 + sigma\_X \symbol{94} 2)) * x(4) * (x(6) \symbol{94} 2) + 0.384e3 * exp(-((x(6) - tau0) \symbol{94}2 / (x(5) \symbol{94} 2 + 2 * sigma\_X \symbol{94} 2))) * h * (sigma\_X \symbol{94} 12) * sqrt(0.2e1) * sqrt(pi) * sqrt((x(5) \symbol{94} 2 + sigma\_X \symbol{94} 2)) * x(4) * (tau0 \symbol{94} 2) - 0.64e2 * exp(-((x(6) - tau0) \symbol{94} 2 / (x(5) \symbol{94} 2 + 2 * sigma\_X \symbol{94} 2))) * h * (sigma\_X \symbol{94} 10) * sqrt(0.2e1) * sqrt(pi) * sqrt((x(5) \symbol{94} 2 + sigma\_X \symbol{94} 2)) * (x(6) \symbol{94} 4) * x(4) - 0.64e2 * exp(-((x(6) - tau0) \symbol{94} 2 / (x(5) \symbol{94} 2 + 2 * sigma\_X \symbol{94} 2))) * h * (sigma\_X \symbol{94} 10) * sqrt(0.2e1) * sqrt(pi) * sqrt((x(5) \symbol{94} 2 + sigma\_X \symbol{94} 2)) * (tau0 \symbol{94} 4) * x(4) + 0.960e3 * (sigma\_X \symbol{94} 12) * exp(-((x(6) -tau0) \symbol{94} 2 / (x(5) \symbol{94} 2 + 2 * sigma\_X \symbol{94} 2))) * h * sqrt(0.2e1) * sqrt(pi) * sqrt((x(5) \symbol{94} 2 + sigma\_X \symbol{94} 2)) * x(6) * x(3) - 0.960e3 * (sigma\_X \symbol{94} 12) * exp(-((x(6) - tau0) \symbol{94} 2 / (x(5) \symbol{94} 2 + 2 * sigma\_X \symbol{94} 2))) * h * sqrt(0.2e1) * sqrt(pi) * sqrt((x(5) \symbol{94} 2 + sigma\_X \symbol{94} 2)) * tau0 * x(3) + 0.288e3 * (sigma\_X \symbol{94} 10) * sqrt((x(5) \symbol{94} 2 + sigma\_X \symbol{94} 2)) * x(3) * sqrt(pi) * exp(-((x(6) - tau0) \symbol{94} 2 / (x(5) \symbol{94} 2 + 2 * sigma\_X \symbol{94} 2))) * sqrt(0.2e1) * h * (tau0 \symbol{94} 3) - 0.288e3 * (sigma\_X \symbol{94} 10) * sqrt((x(5) \symbol{94} 2 + sigma\_X \symbol{94} 2)) * x(3) * sqrt(pi) * exp(-((x(6) - tau0) \symbol{94} 2 / (x(5) \symbol{94} 2 + 2 * sigma\_X \symbol{94} 2))) * sqrt(0.2e1) * h * (x(6) \symbol{94} 3) - 0.256e3 * exp(-((x(6) - tau0) \symbol{94} 2 / (x(5) \symbol{94} 2 + sigma\_X \symbol{94} 2))) * h \symbol{94} 2 * (sigma\_X \symbol{94} 8) * sqrt(pi) * sqrt((x(5) \symbol{94} 2 + 2 * sigma\_X \symbol{94} 2)) * (tau0 \symbol{94} 4) -0.256e3 * exp(-((x(6) - tau0) \symbol{94} 2 / (x(5) \symbol{94} 2 + sigma\_X \symbol{94} 2))) * h \symbol{94} 2 * (sigma\_X \symbol{94} 8) * sqrt(pi) * sqrt((x(5) \symbol{94} 2 + 2 * sigma\_X \symbol{94} 2)) * (x(6) \symbol{94} 4) + 0.1728e4 * sqrt((x(5) \symbol{94} 2 + 2 * sigma\_X \symbol{94} 2)) * exp(-((x(6) - tau0) \symbol{94} 2 / (x(5) \symbol{94} 2 + sigma\_X \symbol{94} 2))) * h\symbol{94} 2 * (sigma\_X \symbol{94} 10) * sqrt(pi) * (tau0 \symbol{94} 2) + 0.1728e4 * sqrt((x(5) \symbol{94} 2 + 2 * sigma\_X \symbol{94} 2)) * exp(-((x(6) - tau0) \symbol{94} 2 / (x(5) \symbol{94} 2 + sigma\_X \symbol{94} 2))) * h \symbol{94} 2 * (sigma\_X \symbol{94} 10) * sqrt(pi) * (x(6) \symbol{94} 2) + 0.2880e4 * (sigma\_X \symbol{94} 11) * sqrt(pi) * sqrt((x(5) \symbol{94} 2 + sigma\_X \symbol{94} 2)) * sqrt((x(5) \symbol{94} 2 + 2 * sigma\_X \symbol{94} 2)) - 0.384e3 * exp(-((x(6) - tau0) \symbol{94} 2 / (x(5) \symbol{94} 2 + 2 * sigma\_X \symbol{94} 2))) * h * (sigma\_X \symbol{94} 10) * sqrt(0.2e1) * sqrt(pi) * sqrt((x(5) \symbol{94} 2 + sigma\_X \symbol{94} 2)) * (tau0 \symbol{94} 2) * (x(6) \symbol{94} 2) * x(4) - 0.864e3 * (sigma\_X \symbol{94} 10) * sqrt((x(5) \symbol{94} 2 + sigma\_X \symbol{94} 2)) * x(3) * sqrt(pi) * exp(-((x(6) - tau0) \symbol{94} 2 / (x(5) \symbol{94} 2 + 2 * sigma\_X \symbol{94} 2))) * sqrt(0.2e1) * h * (tau0 \symbol{94} 2) * x(6) + 0.864e3 * (sigma\_X \symbol{94} 10) * sqrt((x(5) \symbol{94} 2 + sigma\_X \symbol{94} 2)) * x(3) * sqrt(pi) * exp(-((x(6) - tau0) \symbol{94} 2 /(x(5) \symbol{94} 2 + 2 * sigma\_X \symbol{94} 2))) * sqrt(0.2e1) * h * tau0 * (x(6) \symbol{94} 2) - 0.768e3 * (sigma\_X \symbol{94} 12) * exp(-((x(6) - tau0) \symbol{94} 2 / (x(5) \symbol{94} 2 + 2 * sigma\_X \symbol{94} 2))) * h * sqrt(0.2e1) * sqrt(pi) * sqrt((x(5) \symbol{94} 2 + sigma\_X \symbol{94} 2)) * x(6) * x(4) * tau0 + 0.256e3 * exp(-((x(6) - tau0) \symbol{94} 2 / (x(5) \symbol{94} 2 + 2 * sigma\_X \symbol{94} 2))) * h * (sigma\_X \symbol{94} 10) * sqrt(0.2e1) * sqrt(pi) * sqrt((x(5) \symbol{94} 2 + sigma\_X \symbol{94} 2)) * (tau0 \symbol{94} 3) * x(6) * x(4) + 0.256e3 * exp(-((x(6) - tau0) \symbol{94} 2 / (x(5) \symbol{94} 2 + 2 * sigma\_X \symbol{94} 2))) * h * (sigma\_X \symbol{94} 10) * sqrt(0.2e1) * sqrt(pi) * sqrt((x(5) \symbol{94} 2 + sigma\_X \symbol{94} 2)) * (x(6) \symbol{94} 3) * x(4) * tau0 + 0.96e2 * (sigma\_X \symbol{94} 12) * exp(-((x(6) - tau0) \symbol{94} 2 / (x(5) \symbol{94} 2 + sigma\_X \symbol{94} 2))) * h \symbol{94} 2 * sqrt(pi) * sqrt((x(5) \symbol{94} 2 + 2 * sigma\_X \symbol{94} 2)) + 0.1984e4 * (sigma\_X \symbol{94} 13) * k * sqrt(pi)* sqrt((x(5) \symbol{94} 2 + sigma\_X \symbol{94} 2)) * sqrt((x(5) \symbol{94} 2 + 2 * sigma\_X \symbol{94} 2)) - 0.3968e4 * (sigma\_X \symbol{94} 13) * (abs(cg) \symbol{94} 2) * sqrt(pi) * sqrt((x(5) \symbol{94} 2 + sigma\_X \symbol{94} 2)) * sqrt((x(5) \symbol{94} 2 + 2 * sigma\_X \symbol{94} 2)) - 0.1984e4 * (sigma\_X \symbol{94} 13) * x(3) \symbol{94} 2 * sqrt(pi) * sqrt((x(5) \symbol{94} 2 + sigma\_X \symbol{94} 2)) * sqrt((x(5) \symbol{94} 2 + 2 * sigma\_X \symbol{94} 2)) + 0.1024e4 * exp(-((x(6) - tau0) \symbol{94} 2 / (x(5) \symbol{94} 2 + sigma\_X \symbol{94} 2))) * h \symbol{94} 2 * (sigma\_X \symbol{94} 8) * sqrt(pi) * sqrt((x(5) \symbol{94} 2 + 2 * sigma\_X \symbol{94} 2)) * (x(6) \symbol{94} 3) * tau0 + 0.1024e4 * exp(-((x(6) - tau0) \symbol{94} 2 /(x(5) \symbol{94} 2 + sigma\_X \symbol{94} 2))) * h \symbol{94} 2 * (sigma\_X \symbol{94} 8) * sqrt(pi) * sqrt((x(5) \symbol{94} 2 + 2 * sigma\_X \symbol{94} 2)) * (tau0 \symbol{94} 3) * x(6) - 0.1536e4 * exp(-((x(6) - tau0) \symbol{94} 2 / (x(5) \symbol{94} 2 + sigma\_X \symbol{94} 2))) * h \symbol{94} 2 * (sigma\_X \symbol{94} 8) * sqrt(pi) * sqrt((x(5) \symbol{94} 2 + 2 * sigma\_X \symbol{94} 2))* (tau0 \symbol{94} 2) * (x(6) \symbol{94} 2) - 0.192e3 * (sigma\_X \symbol{94} 14) * exp(-((x(6) - tau0) \symbol{94} 2 / (x(5) \symbol{94} 2 + 2 * sigma\_X \symbol{94} 2))) * h * x(4) * sqrt(0.2e1) * sqrt(pi) * sqrt((x(5) \symbol{94} 2 + sigma\_X \symbol{94} 2)) - 0.3456e4 * sqrt((x(5) \symbol{94} 2 + 2 * sigma\_X \symbol{94} 2)) * exp(-((x(6) - tau0) \symbol{94} 2/ (x(5) \symbol{94} 2 + sigma\_X \symbol{94} 2))) * h \symbol{94} 2 * (sigma\_X \symbol{94} 10) * sqrt(pi) * tau0 * x(6)) * pi \symbol{94} (-0.1e1 / 0.2e1) * ((x(5) \symbol{94} 2 + sigma\_X \symbol{94} 2) \symbol{94} (-0.9e1 / 0.2e1)) / sigma\_X * ((x(5) \symbol{94} 2 + 2 * sigma\_X \symbol{94} 2) \symbol{94} (-0.9e1 / 0.2e1)) * (x(5) \symbol{94} 4) / 0.8e1 - (0.512e3 * exp(-((x(6) - tau0) \symbol{94} 2 / (x(5) \symbol{94} 2 + sigma\_X \symbol{94} 2))) * h \symbol{94} 2 * (sigma\_X \symbol{94} 10) * sqrt(pi) * sqrt((x(5) \symbol{94} 2 + 2 * sigma\_X \symbol{94} 2)) * (tau0 \symbol{94} 3) * x(6) - 0.1664e4 * sqrt((x(5) \symbol{94} 2 + 2 * sigma\_X \symbol{94} 2)) * exp(-((x(6) - tau0) \symbol{94} 2 / (x(5) \symbol{94} 2 + sigma\_X \symbol{94} 2))) * h \symbol{94} 2 * (sigma\_X \symbol{94} 12) * sqrt(pi) * tau0 * x(6) + 0.192e3 * (sigma\_X \symbol{94} 12) * sqrt((x(5) \symbol{94} 2 + sigma\_X \symbol{94} 2)) * x(3) * sqrt(pi) * exp(-((x(6) - tau0) \symbol{94} 2 / (x(5) \symbol{94} 2 + 2 * sigma\_X \symbol{94} 2))) * sqrt(0.2e1) * h * tau0 * (x(6) \symbol{94} 2) + 0.1984e4 * (sigma\_X \symbol{94} 13) * sqrt(pi) * sqrt((x(5) \symbol{94} 2 + sigma\_X \symbol{94} 2)) * sqrt((x(5) \symbol{94} 2 + 2 * sigma\_X \symbol{94} 2)) - 0.128e3 * exp(-((x(6) - tau0) \symbol{94} 2 / (x(5) \symbol{94} 2 + sigma\_X \symbol{94} 2))) * h \symbol{94} 2 * (sigma\_X \symbol{94} 10) * sqrt(pi) * sqrt((x(5) \symbol{94} 2 + 2 * sigma\_X \symbol{94} 2)) * (x(6) \symbol{94} 4) - 0.192e3 * (sigma\_X \symbol{94} 12) * sqrt((x(5) \symbol{94}2 + sigma\_X \symbol{94} 2)) * x(3) * sqrt(pi) * exp(-((x(6) - tau0) \symbol{94} 2 / (x(5) \symbol{94} 2 + 2 * sigma\_X \symbol{94} 2))) * sqrt(0.2e1) * h * (tau0 \symbol{94} 2) * x(6) - 0.768e3 * (sigma\_X \symbol{94} 15) * x(3) \symbol{94} 2 * sqrt(pi) * sqrt((x(5) \symbol{94} 2 + sigma\_X \symbol{94} 2)) * sqrt((x(5) \symbol{94} 2 + 2 * sigma\_X \symbol{94} 2)) - 0.64e2 * (sigma\_X \symbol{94} 12) * sqrt((x(5) \symbol{94} 2 + sigma\_X \symbol{94} 2)) * x(3) * sqrt(pi) * exp(-((x(6) - tau0) \symbol{94} 2 / (x(5) \symbol{94} 2 + 2 * sigma\_X \symbol{94} 2))) * sqrt(0.2e1) * h * (x(6) \symbol{94} 3) - 0.1536e4 * (sigma\_X \symbol{94} 15) * (abs(cg) \symbol{94} 2) * sqrt(pi) * sqrt((x(5) \symbol{94} 2 + sigma\_X \symbol{94} 2)) * sqrt((x(5) \symbol{94} 2 + 2 * sigma\_X \symbol{94} 2)) - 0.768e3 * exp(-((x(6) - tau0) \symbol{94} 2 / (x(5) \symbol{94} 2 + sigma\_X \symbol{94} 2))) * h \symbol{94} 2 * (sigma\_X \symbol{94} 10) * sqrt(pi) * sqrt((x(5) \symbol{94} 2 + 2 * sigma\_X \symbol{94} 2)) * (tau0 \symbol{94} 2) * (x(6) \symbol{94} 2) - 0.192e3 * (sigma\_X \symbol{94} 14) * exp(-((x(6) - tau0) \symbol{94} 2 / (x(5)\symbol{94} 2 + 2 * sigma\_X \symbol{94} 2))) * h * sqrt(0.2e1) * sqrt(pi) * sqrt((x(5) \symbol{94} 2 + sigma\_X \symbol{94} 2)) * tau0 * x(3) + 0.768e3 * (sigma\_X \symbol{94} 15) * k * sqrt(pi) * sqrt((x(5) \symbol{94} 2 + sigma\_X \symbol{94} 2)) * sqrt((x(5) \symbol{94} 2 + 2 * sigma\_X \symbol{94} 2)) + 0.832e3 * sqrt((x(5) \symbol{94} 2 + 2 * sigma\_X \symbol{94}2)) * exp(-((x(6) - tau0) \symbol{94} 2 / (x(5) \symbol{94} 2 + sigma\_X \symbol{94} 2))) * h \symbol{94} 2 * (sigma\_X \symbol{94} 12) * sqrt(pi) * (tau0 \symbol{94} 2) + 0.64e2 * (sigma\_X \symbol{94} 12) * sqrt((x(5) \symbol{94} 2 + sigma\_X \symbol{94} 2)) * x(3) * sqrt(pi) * exp(-((x(6) - tau0) \symbol{94} 2 / (x(5) \symbol{94} 2 + 2 * sigma\_X \symbol{94} 2))) * sqrt(0.2e1) * h * (tau0 \symbol{94} 3) + 0.192e3 * (sigma\_X \symbol{94} 14) * exp(-((x(6) - tau0) \symbol{94} 2 / (x(5) \symbol{94} 2 + 2 * sigma\_X \symbol{94} 2))) * h * sqrt(0.2e1) * sqrt(pi) * sqrt((x(5) \symbol{94} 2 + sigma\_X \symbol{94} 2)) * x(6) * x(3) - 0.128e3 * exp(-((x(6) - tau0) \symbol{94} 2 / (x(5) \symbol{94} 2 + sigma\_X \symbol{94} 2))) * h \symbol{94} 2 *(sigma\_X \symbol{94} 10) * sqrt(pi) * sqrt((x(5) \symbol{94} 2 + 2 * sigma\_X \symbol{94} 2)) * (tau0 \symbol{94} 4) + 0.832e3 * (sigma\_X \symbol{94} 12) * exp(-((x(6) - tau0) \symbol{94} 2 / (x(5) \symbol{94} 2 + sigma\_X \symbol{94} 2))) * h \symbol{94} 2 * sqrt(pi) * sqrt((x(5) \symbol{94} 2 + 2 * sigma\_X \symbol{94} 2)) * (x(6) \symbol{94} 2) + 0.512e3 * exp(-((x(6) - tau0) \symbol{94} 2 / (x(5) \symbol{94} 2 + sigma\_X \symbol{94} 2))) * h \symbol{94} 2 * (sigma\_X \symbol{94} 10) * sqrt(pi) * sqrt((x(5) \symbol{94} 2 + 2 * sigma\_X \symbol{94} 2)) * (x(6) \symbol{94} 3) * tau0) * pi \symbol{94} (-0.1e1 / 0.2e1) * ((x(5) \symbol{94} 2 + sigma\_X \symbol{94} 2) \symbol{94} (-0.9e1 / 0.2e1)) / sigma\_X * ((x(5) \symbol{94} 2 + 2 * sigma\_X \symbol{94} 2) \symbol{94} (-0.9e1 /0.2e1)) * (x(5) \symbol{94} 2) / 0.8e1 - (-0.256e3 * (sigma\_X \symbol{94} 17) * (abs(cg) \symbol{94} 2) * sqrt(pi) * sqrt((x(5) \symbol{94} 2 + sigma\_X \symbol{94} 2)) * sqrt((x(5) \symbol{94} 2 + 2 * sigma\_X \symbol{94} 2)) - 0.128e3 * (sigma\_X \symbol{94} 17) * sqrt((x(5) \symbol{94} 2 + sigma\_X \symbol{94} 2)) * x(3) \symbol{94} 2 * sqrt(pi) * sqrt((x(5) \symbol{94} 2 +2 * sigma\_X \symbol{94} 2)) + 0.128e3 * sqrt((x(5) \symbol{94} 2 + 2 * sigma\_X \symbol{94} 2)) * exp(-((x(6) - tau0) \symbol{94} 2 / (x(5) \symbol{94} 2 + sigma\_X \symbol{94} 2))) * h \symbol{94} 2 * (sigma\_X \symbol{94} 14) * sqrt(pi) * (tau0 \symbol{94} 2) + 0.768e3 * (sigma\_X \symbol{94} 15) * sqrt(pi) * sqrt((x(5) \symbol{94} 2 + sigma\_X \symbol{94} 2)) * sqrt((x(5) \symbol{94}2 + 2 * sigma\_X \symbol{94} 2)) + 0.128e3 * (sigma\_X \symbol{94} 14) * exp(-((x(6) - tau0) \symbol{94} 2 / (x(5) \symbol{94} 2 + sigma\_X \symbol{94} 2))) * h \symbol{94} 2 * sqrt(pi) * sqrt((x(5) \symbol{94} 2 + 2 * sigma\_X \symbol{94} 2)) * (x(6) \symbol{94} 2) - 0.256e3 * sqrt((x(5) \symbol{94} 2 + 2 * sigma\_X \symbol{94} 2)) * exp(-((x(6) - tau0) \symbol{94} 2 / (x(5) \symbol{94}2 + sigma\_X \symbol{94} 2))) * h \symbol{94} 2 * (sigma\_X \symbol{94} 14) * sqrt(pi) * tau0 * x(6) + 0.128e3 * (sigma\_X \symbol{94} 17) * k * sqrt(pi) * sqrt((x(5) \symbol{94} 2 + sigma\_X \symbol{94} 2)) * sqrt((x(5) \symbol{94} 2 + 2 * sigma\_X \symbol{94} 2))) * pi \symbol{94} (-0.1e1 / 0.2e1) * ((x(5) \symbol{94} 2 + sigma\_X \symbol{94} 2) \symbol{94} (-0.9e1 / 0.2e1)) /sigma\_X * ((x(5) \symbol{94} 2 + 2 * sigma\_X \symbol{94} 2) \symbol{94} (-0.9e1 / 0.2e1)) / 0.8e1 - (16 / (x(5) \symbol{94} 2 + sigma\_X \symbol{94} 2) \symbol{94} 4 / (x(5) \symbol{94} 2 + 2 * sigma\_X \symbol{94} 2) \symbol{94} 4 * sigma\_X \symbol{94} 16 / x(5) \symbol{94} 2) + (-0.3e1 / 0.4e1 * Ia / ((x(5) \symbol{94} 2 + sigma\_X \symbol{94} 2) \symbol{94} 4) / ((x(5) \symbol{94} 2 + 2 * sigma\_X \symbol{94} 2) \symbol{94}4) * pi \symbol{94} (-0.1e1 / 0.2e1) * (x(5) \symbol{94} 15) - 0.9e1 * Ia / ((x(5) \symbol{94} 2 + sigma\_X \symbol{94} 2) \symbol{94} 4) * (sigma\_X \symbol{94} 2) / ((x(5) \symbol{94} 2 + 2 * sigma\_X \symbol{94} 2) \symbol{94} 4) * pi \symbol{94} (-0.1e1 / 0.2e1) * (x(5) \symbol{94} 13) - 0.93e2 / 0.2e1 * Ia / ((x(5) \symbol{94} 2 + sigma\_X \symbol{94} 2) \symbol{94} 4) * (sigma\_X \symbol{94} 4) / ((x(5) \symbol{94} 2 + 2 * sigma\_X \symbol{94} 2) \symbol{94} 4) * pi \symbol{94} (-0.1e1 / 0.2e1) * (x(5) \symbol{94} 11) - 0.135e3 * (sigma\_X \symbol{94} 6) * Ia / ((x(5) \symbol{94} 2 + sigma\_X \symbol{94} 2) \symbol{94} 4) / ((x(5) \symbol{94} 2 + 2 * sigma\_X \symbol{94} 2) \symbol{94} 4) * pi \symbol{94} (-0.1e1 / 0.2e1) * (x(5) \symbol{94} 9) - 0.963e3 / 0.4e1 * (sigma\_X \symbol{94} 8) * Ia / ((x(5) \symbol{94}2 + sigma\_X \symbol{94} 2) \symbol{94} 4) / ((x(5) \symbol{94} 2 + 2 * sigma\_X \symbol{94} 2) \symbol{94} 4) * pi \symbol{94} (-0.1e1 / 0.2e1) * (x(5) \symbol{94} 7) - 0.270e3 * (sigma\_X \symbol{94} 10) * Ia / ((x(5) \symbol{94} 2 + sigma\_X \symbol{94} 2) \symbol{94} 4) / ((x(5) \symbol{94} 2 + 2 * sigma\_X \symbol{94} 2) \symbol{94} 4) * pi \symbol{94} (-0.1e1 / 0.2e1) * (x(5) \symbol{94} 5) - 0.186e3 * (sigma\_X\symbol{94} 12) * Ia / ((x(5) \symbol{94} 2 + sigma\_X \symbol{94} 2) \symbol{94} 4) / ((x(5) \symbol{94} 2 + 2 * sigma\_X \symbol{94} 2) \symbol{94} 4) * pi \symbol{94} (-0.1e1 / 0.2e1) * (x(5) \symbol{94} 3) - 0.72e2 * (sigma\_X \symbol{94} 14) * Ia / ((x(5) \symbol{94} 2 + sigma\_X \symbol{94} 2) \symbol{94} 4) / ((x(5) \symbol{94} 2 + 2 * sigma\_X \symbol{94} 2) \symbol{94} 4) * pi \symbol{94} (-0.1e1 / 0.2e1) * x(5) - 0.12e2 * (sigma\_X \symbol{94} 16) * Ia / ((x(5) \symbol{94} 2 + sigma\_X \symbol{94} 2) \symbol{94} 4) / ((x(5) \symbol{94} 2 + 2 * sigma\_X \symbol{94} 2) \symbol{94} 4) * pi \symbol{94} (-0.1e1 / 0.2e1) / x(5)) / x(1) + (Is / ((x(5) \symbol{94} 2 + sigma\_X \symbol{94} 2) \symbol{94} 4) / ((x(5) \symbol{94} 2 + 2 * sigma\_X \symbol{94} 2) \symbol{94} 4) * pi \symbol{94} (-0.1e1 / 0.2e1) * (x(5) \symbol{94} 16) / 0.2e1+ 0.1e1 / ((x(5) \symbol{94} 2 + sigma\_X \symbol{94} 2) \symbol{94} 4) * x(3) * Ic / ((x(5) \symbol{94} 2 + 2 * sigma\_X \symbol{94} 2) \symbol{94} 4) * pi \symbol{94} (-0.1e1 / 0.2e1) * (x(5) \symbol{94} 15) + 0.6e1 * Is / ((x(5) \symbol{94} 2 + sigma\_X \symbol{94} 2) \symbol{94} 4) * (sigma\_X \symbol{94} 2) / ((x(5) \symbol{94} 2 + 2 * sigma\_X \symbol{94} 2) \symbol{94} 4) * pi \symbol{94} (-0.1e1 / 0.2e1) * (x(5) \symbol{94} 14) + 0.12e2 * (sigma\_X \symbol{94} 2) / ((x(5) \symbol{94} 2 + sigma\_X \symbol{94} 2) \symbol{94} 4) * x(3) * Ic / ((x(5) \symbol{94} 2 + 2 * sigma\_X \symbol{94} 2) \symbol{94} 4) * pi \symbol{94} (-0.1e1 / 0.2e1) * (x(5) \symbol{94} 13) + 0.31e2 * Is / ((x(5) \symbol{94} 2 + sigma\_X \symbol{94} 2) \symbol{94} 4) * (sigma\_X \symbol{94} 4) / ((x(5) \symbol{94} 2 + 2 * sigma\_X \symbol{94} 2) \symbol{94} 4) *pi \symbol{94} (-0.1e1 / 0.2e1) * (x(5) \symbol{94} 12) + 0.62e2 * (sigma\_X \symbol{94} 4) / ((x(5) \symbol{94} 2 + sigma\_X \symbol{94} 2) \symbol{94} 4) * x(3) * Ic / ((x(5) \symbol{94} 2 + 2 * sigma\_X \symbol{94} 2) \symbol{94} 4) * pi \symbol{94} (-0.1e1 / 0.2e1) * (x(5) \symbol{94} 11) + 0.90e2 * Is / ((x(5) \symbol{94} 2 + sigma\_X \symbol{94} 2) \symbol{94} 4) * (sigma\_X \symbol{94} 6) / ((x(5) \symbol{94}2 + 2 * sigma\_X \symbol{94} 2) \symbol{94} 4) * pi \symbol{94} (-0.1e1 / 0.2e1) * (x(5) \symbol{94} 10) + 0.180e3 * (sigma\_X \symbol{94} 6) / ((x(5) \symbol{94} 2 + sigma\_X \symbol{94} 2) \symbol{94} 4) * x(3) * Ic / ((x(5) \symbol{94} 2 + 2 * sigma\_X \symbol{94} 2) \symbol{94} 4) * pi \symbol{94} (-0.1e1 / 0.2e1) * (x(5) \symbol{94} 9) + 0.321e3 / 0.2e1 * Is / ((x(5) \symbol{94} 2 + sigma\_X \symbol{94}2) \symbol{94} 4) * (sigma\_X \symbol{94} 8) / ((x(5) \symbol{94} 2 + 2 * sigma\_X \symbol{94} 2) \symbol{94} 4) * pi \symbol{94} (-0.1e1 / 0.2e1) * (x(5) \symbol{94} 8) + 0.321e3 * (sigma\_X \symbol{94} 8) / ((x(5) \symbol{94} 2 + sigma\_X \symbol{94} 2) \symbol{94} 4) * x(3) * Ic / ((x(5) \symbol{94} 2 + 2 * sigma\_X \symbol{94} 2) \symbol{94} 4) * pi \symbol{94} (-0.1e1 / 0.2e1) * (x(5) \symbol{94} 7) + 0.180e3 *Is / ((x(5) \symbol{94} 2 + sigma\_X \symbol{94} 2) \symbol{94} 4) * (sigma\_X \symbol{94} 10) / ((x(5) \symbol{94} 2 + 2 * sigma\_X \symbol{94} 2) \symbol{94} 4) * pi \symbol{94} (-0.1e1 / 0.2e1) * (x(5) \symbol{94} 6) + 0.360e3 * (sigma\_X \symbol{94} 10) / ((x(5) \symbol{94} 2 + sigma\_X \symbol{94} 2) \symbol{94} 4) * x(3) * Ic / ((x(5) \symbol{94} 2 + 2 * sigma\_X \symbol{94} 2) \symbol{94} 4) * pi \symbol{94} (-0.1e1 / 0.2e1) * (x(5) \symbol{94} 5) + 0.124e3 * Is / ((x(5) \symbol{94} 2 + sigma\_X \symbol{94} 2) \symbol{94} 4) * (sigma\_X \symbol{94} 12) / ((x(5) \symbol{94} 2 + 2 * sigma\_X \symbol{94} 2) \symbol{94} 4) * pi \symbol{94} (-0.1e1 / 0.2e1) * (x(5) \symbol{94} 4) + 0.248e3 * (sigma\_X \symbol{94} 12) / ((x(5) \symbol{94} 2 + sigma\_X \symbol{94} 2) \symbol{94} 4) * x(3) * Ic / ((x(5) \symbol{94} 2 + 2 * sigma\_X \symbol{94}2) \symbol{94} 4) * pi \symbol{94} (-0.1e1 / 0.2e1) * (x(5) \symbol{94} 3) + 0.48e2 * Is / ((x(5) \symbol{94} 2 + sigma\_X \symbol{94} 2) \symbol{94} 4) * (sigma\_X \symbol{94} 14) / ((x(5) \symbol{94} 2 + 2 * sigma\_X \symbol{94} 2) \symbol{94} 4) * pi \symbol{94} (-0.1e1 / 0.2e1) * (x(5) \symbol{94} 2) + 0.96e2 * (sigma\_X \symbol{94} 14) / ((x(5) \symbol{94} 2 + sigma\_X \symbol{94} 2) \symbol{94} 4) * x(3) * Ic /((x(5) \symbol{94} 2 + 2 * sigma\_X \symbol{94} 2) \symbol{94} 4) * pi \symbol{94} (-0.1e1 / 0.2e1) * x(5) + 0.8e1 * Is / ((x(5) \symbol{94} 2 + sigma\_X \symbol{94} 2) \symbol{94} 4) * (sigma\_X \symbol{94} 16) / ((x(5) \symbol{94} 2 + 2 * sigma\_X \symbol{94} 2) \symbol{94} 4) * pi \symbol{94} (-0.1e1 / 0.2e1) + 0.16e2 * (sigma\_X \symbol{94} 16) / ((x(5) \symbol{94} 2 + sigma\_X \symbol{94} 2) \symbol{94} 4) * x(3)* Ic / ((x(5) \symbol{94} 2 + 2 * sigma\_X \symbol{94} 2) \symbol{94} 4) * pi \symbol{94} (-0.1e1 / 0.2e1) / x(5)) / x(1) \symbol{94} 2;
\underline{}Warning, the function names \{x\} are not recognized in the target language\underline{}\mapleresult
cg1 = 0.0e0 == -(-0.144e3 * exp(-((x(6) - tau0) \symbol{94} 2 / (x(5) \symbol{94} 2 + 2 * sigma\_X \symbol{94} 2))) * h * (sigma\_X \symbol{94} 6) * (x(5) \symbol{94} 5) * x(1) \symbol{94} 2 * sqrt(0.2e1) * sqrt(pi) * sqrt((x(5) \symbol{94} 2 + sigma\_X \symbol{94} 2)) * tau0 * x(6) * x(3) - 0.112e3 * exp(-((x(6) - tau0) \symbol{94} 2 / (x(5) \symbol{94} 2+ 2 * sigma\_X \symbol{94} 2))) * h * (sigma\_X \symbol{94} 8) * (x(5) \symbol{94} 3) * x(1) \symbol{94} 2 * sqrt(0.2e1) * sqrt(pi) * sqrt((x(5) \symbol{94} 2 + sigma\_X \symbol{94} 2)) * tau0 * x(6) * x(3) - 0.32e2 * exp(-((x(6) - tau0) \symbol{94} 2 / (x(5) \symbol{94} 2 + 2 * sigma\_X \symbol{94} 2))) * h * (sigma\_X \symbol{94} 10) * x(5) * x(1) \symbol{94} 2 * sqrt(0.2e1) * sqrt(pi) * sqrt((x(5) \symbol{94} 2 + sigma\_X \symbol{94} 2)) * tau0 * x(6) * x(3) + 0.96e2 * exp(-((x(6) - tau0) \symbol{94} 2 / (x(5) \symbol{94} 2 + 2 * sigma\_X \symbol{94} 2))) * h * (sigma\_X \symbol{94} 2) * (x(5) \symbol{94} 9) * x(1) \symbol{94} 2 * sqrt(0.2e1) * sqrt(pi) * sqrt((x(5) \symbol{94} 2 + sigma\_X \symbol{94} 2)) * (x(6) \symbol{94} 2)* x(4) * tau0 + 0.288e3 * exp(-((x(6) - tau0) \symbol{94} 2 / (x(5) \symbol{94} 2 + 2 * sigma\_X \symbol{94} 2))) * h * (sigma\_X \symbol{94} 4) * (x(5) \symbol{94} 7) * x(1) \symbol{94} 2 * sqrt(0.2e1) * sqrt(pi) * sqrt((x(5) \symbol{94} 2 + sigma\_X \symbol{94} 2)) * (x(6) \symbol{94} 2) * x(4) * tau0 + 0.288e3 * exp(-((x(6) - tau0) \symbol{94} 2 / (x(5)\symbol{94} 2 + 2 * sigma\_X \symbol{94} 2))) * h * (sigma\_X \symbol{94} 6) * (x(5) \symbol{94} 5) * x(1) \symbol{94} 2 * sqrt(0.2e1) * sqrt(pi) * sqrt((x(5) \symbol{94} 2 + sigma\_X \symbol{94} 2)) * (x(6) \symbol{94} 2) * x(4) * tau0 + 0.96e2 * exp(-((x(6) - tau0) \symbol{94} 2 / (x(5) \symbol{94} 2 + 2 * sigma\_X \symbol{94} 2))) * h * (sigma\_X \symbol{94} 8) * (x(5) \symbol{94} 3)* x(1) \symbol{94} 2 * sqrt(0.2e1) * sqrt(pi) * sqrt((x(5) \symbol{94} 2 + sigma\_X \symbol{94} 2)) * (x(6) \symbol{94} 2) * x(4) * tau0 - 0.16e2 * exp(-((x(6) - tau0) \symbol{94} 2 / (x(5) \symbol{94} 2 + 2 * sigma\_X \symbol{94} 2))) * h * (sigma\_X \symbol{94} 2) * (x(5) \symbol{94} 9) * x(1) \symbol{94} 2 * sqrt(0.2e1) * sqrt(pi) * sqrt((x(5) \symbol{94} 2 + sigma\_X \symbol{94} 2)) * tau0 * x(6) * x(3) - 0.80e2 * exp(-((x(6) - tau0) \symbol{94} 2 / (x(5) \symbol{94} 2 + 2 * sigma\_X \symbol{94} 2))) * h * (sigma\_X \symbol{94} 4) * (x(5) \symbol{94} 7) * x(1) \symbol{94} 2 * sqrt(0.2e1) * sqrt(pi) * sqrt((x(5) \symbol{94} 2 + sigma\_X \symbol{94} 2)) * tau0 * x(6) * x(3) - 0.96e2 * exp(-((x(6) - tau0) \symbol{94} 2/ (x(5) \symbol{94} 2 + 2 * sigma\_X \symbol{94} 2))) * h * (sigma\_X \symbol{94} 2) * (x(5) \symbol{94} 9) * x(1) \symbol{94} 2 * sqrt(0.2e1) * sqrt(pi) * sqrt((x(5) \symbol{94} 2 + sigma\_X \symbol{94} 2)) * (tau0 \symbol{94} 2) * x(4) * x(6) - 0.288e3 * exp(-((x(6) - tau0) \symbol{94} 2 / (x(5) \symbol{94} 2 + 2 * sigma\_X \symbol{94} 2))) * h * (sigma\_X \symbol{94} 4) * (x(5) \symbol{94} 7) * x(1) \symbol{94} 2 * sqrt(0.2e1) * sqrt(pi) * sqrt((x(5) \symbol{94} 2 + sigma\_X \symbol{94} 2)) * (tau0 \symbol{94} 2) * x(4) * x(6) - 0.288e3 * exp(-((x(6) - tau0) \symbol{94} 2 / (x(5) \symbol{94} 2 + 2 * sigma\_X \symbol{94} 2))) * h * (sigma\_X \symbol{94} 6) * (x(5) \symbol{94} 5) * x(1) \symbol{94} 2 * sqrt(0.2e1) * sqrt(pi) * sqrt((x(5)\symbol{94} 2 + sigma\_X \symbol{94} 2)) * (tau0 \symbol{94} 2) * x(4) * x(6) - 0.96e2 * exp(-((x(6) - tau0) \symbol{94} 2 / (x(5) \symbol{94} 2 + 2 * sigma\_X \symbol{94} 2))) * h * (sigma\_X \symbol{94} 8) * (x(5) \symbol{94} 3) * x(1) \symbol{94} 2 * sqrt(0.2e1) * sqrt(pi) * sqrt((x(5) \symbol{94} 2 + sigma\_X \symbol{94} 2)) * (tau0 \symbol{94} 2) * x(4) * x(6) - 0.16e2 * exp(-((x(6) - tau0) \symbol{94} 2 / (x(5) \symbol{94} 2 + sigma\_X \symbol{94} 2))) * h \symbol{94} 2 * x(1) \symbol{94} 2 * (x(5) \symbol{94} 5) * sqrt(pi) * sqrt((x(5) \symbol{94} 2 + 2 * sigma\_X \symbol{94} 2)) * tau0 * (sigma\_X \symbol{94} 6) - 0.5e1 * exp(-((x(6) - tau0) \symbol{94} 2 / (x(5) \symbol{94} 2 + sigma\_X \symbol{94} 2))) * h \symbol{94} 2 * x(1) \symbol{94} 2 * (x(5) \symbol{94} 9) * sqrt(pi) * sqrt((x(5) \symbol{94} 2 + 2 * sigma\_X \symbol{94} 2)) * x(6) * (sigma\_X \symbol{94} 2) - 0.4e1 * exp(-((x(6) - tau0) \symbol{94} 2 / (x(5) \symbol{94} 2 + sigma\_X \symbol{94} 2))) * h \symbol{94} 2 * x(1) \symbol{94} 2 * (x(5) \symbol{94} 7) * sqrt(pi) * sqrt((x(5) \symbol{94} 2 + 2 * sigma\_X \symbol{94} 2)) * x(6) * (sigma\_X \symbol{94} 4) + 0.16e2 * exp(-((x(6) -tau0) \symbol{94} 2 / (x(5) \symbol{94} 2 + sigma\_X \symbol{94} 2))) * h \symbol{94} 2 * x(1) \symbol{94} 2 * (x(5) \symbol{94} 5) * sqrt(pi) * sqrt((x(5) \symbol{94} 2 + 2 * sigma\_X \symbol{94} 2)) * x(6) * (sigma\_X \symbol{94} 6) + 0.32e2 * exp(-((x(6) - tau0) \symbol{94} 2 / (x(5) \symbol{94} 2 + sigma\_X \symbol{94} 2))) * h \symbol{94} 2 * x(1) \symbol{94} 2 * (x(5) \symbol{94} 3) * sqrt(pi) * sqrt((x(5) \symbol{94} 2 + 2 * sigma\_X \symbol{94} 2)) * x(6) * (sigma\_X \symbol{94} 8) - 0.2e1 * exp(-((x(6) - tau0) \symbol{94} 2 / (x(5) \symbol{94} 2 + sigma\_X \symbol{94} 2))) * h \symbol{94} 2 * (sigma\_X \symbol{94} 2) * (x(5) \symbol{94} 7) * x(1) \symbol{94} 2 * sqrt(pi) * sqrt((x(5) \symbol{94} 2 + 2 * sigma\_X \symbol{94} 2)) * (x(6) \symbol{94} 3) - 0.12e2 * exp(-((x(6) - tau0)\symbol{94} 2 / (x(5) \symbol{94} 2 + sigma\_X \symbol{94} 2))) * h \symbol{94} 2 * (sigma\_X \symbol{94} 4) * (x(5) \symbol{94} 5) * x(1) \symbol{94} 2 * sqrt(pi) * sqrt((x(5) \symbol{94} 2 + 2 * sigma\_X \symbol{94} 2)) * (x(6) \symbol{94} 3) - 0.32e2 * exp(-((x(6) - tau0) \symbol{94} 2 / (x(5) \symbol{94} 2 + sigma\_X \symbol{94} 2))) * h \symbol{94} 2 * x(1) \symbol{94} 2 * (x(5) \symbol{94} 3) * sqrt(pi) * sqrt((x(5) \symbol{94} 2 + 2 * sigma\_X \symbol{94} 2)) * tau0 * (sigma\_X \symbol{94} 8) + 0.16e2 * exp(-((x(6) - tau0) \symbol{94} 2 / (x(5) \symbol{94} 2 + sigma\_X \symbol{94} 2))) * h \symbol{94} 2 * (sigma\_X \symbol{94} 10) * x(5) * x(1) \symbol{94} 2 * sqrt(pi) * sqrt((x(5) \symbol{94} 2 + 2 * sigma\_X \symbol{94} 2)) * x(6) - 0.16e2 * exp(-((x(6) - tau0) \symbol{94} 2 / (x(5) \symbol{94} 2 + sigma\_X \symbol{94} 2))) * h \symbol{94} 2 * (sigma\_X \symbol{94} 10) * x(5) * x(1) \symbol{94} 2 * sqrt(pi) * sqrt((x(5) \symbol{94} 2 + 2 * sigma\_X \symbol{94} 2)) * tau0 + 0.24e2 * exp(-((x(6) - tau0) \symbol{94} 2 / (x(5) \symbol{94} 2 + sigma\_X \symbol{94} 2))) * h \symbol{94} 2 * (sigma\_X \symbol{94} 6) * (x(5) \symbol{94} 3) * x(1) \symbol{94} 2 * sqrt(pi) * sqrt((x(5)\symbol{94} 2 + 2 * sigma\_X \symbol{94} 2)) * (tau0 \symbol{94} 3) + 0.16e2 * exp(-((x(6) - tau0) \symbol{94} 2 / (x(5) \symbol{94} 2 + sigma\_X \symbol{94} 2))) * h \symbol{94} 2 * (sigma\_X \symbol{94} 8) * x(5) * x(1) \symbol{94} 2 * sqrt(pi) * sqrt((x(5) \symbol{94} 2 + 2 * sigma\_X \symbol{94} 2)) * (tau0 \symbol{94} 3) - 0.24e2 * exp(-((x(6) - tau0) \symbol{94} 2 / (x(5) \symbol{94} 2 + sigma\_X \symbol{94} 2))) * h \symbol{94} 2 * (sigma\_X \symbol{94} 6) * (x(5) \symbol{94} 3) * x(1) \symbol{94} 2 * sqrt(pi) * sqrt((x(5) \symbol{94} 2 + 2 * sigma\_X \symbol{94} 2)) * (x(6) \symbol{94} 3) - 0.16e2 * exp(-((x(6) - tau0) \symbol{94} 2 / (x(5) \symbol{94} 2 + sigma\_X \symbol{94} 2))) * h \symbol{94} 2 * (sigma\_X \symbol{94} 8) * x(5) * x(1) \symbol{94} 2 * sqrt(pi) * sqrt((x(5) \symbol{94} 2+ 2 * sigma\_X \symbol{94} 2)) * (x(6) \symbol{94} 3) + 0.2e1 * exp(-((x(6) - tau0) \symbol{94} 2 / (x(5) \symbol{94} 2 + sigma\_X \symbol{94} 2))) * h \symbol{94} 2 * (sigma\_X \symbol{94} 2) * (x(5) \symbol{94} 7) * x(1) \symbol{94} 2 * sqrt(pi) * sqrt((x(5) \symbol{94} 2 + 2 * sigma\_X \symbol{94} 2)) * (tau0 \symbol{94} 3) + 0.12e2 * exp(-((x(6) - tau0) \symbol{94} 2 / (x(5) \symbol{94} 2 + sigma\_X \symbol{94} 2))) * h \symbol{94} 2 * (sigma\_X \symbol{94} 4) * (x(5) \symbol{94} 5) * x(1) \symbol{94} 2 * sqrt(pi) * sqrt((x(5) \symbol{94} 2 + 2 * sigma\_X \symbol{94} 2)) * (tau0 \symbol{94} 3) + 0.5e1 * exp(-((x(6) - tau0) \symbol{94} 2 / (x(5) \symbol{94} 2 + sigma\_X \symbol{94} 2))) * h \symbol{94} 2 * x(1) \symbol{94} 2 * (x(5) \symbol{94} 9) * sqrt(pi) * sqrt((x(5) \symbol{94} 2 + 2 * sigma\_X \symbol{94} 2)) * tau0 * (sigma\_X \symbol{94} 2) + 0.4e1 * exp(-((x(6) - tau0) \symbol{94} 2 / (x(5) \symbol{94} 2 + sigma\_X \symbol{94} 2))) * h \symbol{94} 2 * x(1) \symbol{94} 2 * (x(5) \symbol{94} 7) * sqrt(pi) * sqrt((x(5) \symbol{94} 2 + 2 * sigma\_X \symbol{94} 2)) * tau0 * (sigma\_X \symbol{94} 4) + 0.33e2 * x(3) * sqrt((x(5) \symbol{94} 2 + sigma\_X \symbol{94} 2)) * (sigma\_X \symbol{94} 5) * Ib * sqrt((x(5) \symbol{94} 2 + 2 * sigma\_X \symbol{94} 2)) * (x(5) \symbol{94} 8) + 0.63e2 * x(3) * sqrt((x(5) \symbol{94} 2 + sigma\_X \symbol{94} 2)) * (sigma\_X \symbol{94} 7) * Ib * sqrt((x(5) \symbol{94} 2 + 2 * sigma\_X \symbol{94} 2)) * (x(5) \symbol{94} 6) + 0.66e2 * x(3) * sqrt((x(5) \symbol{94} 2 + sigma\_X \symbol{94} 2)) * (sigma\_X \symbol{94} 9) * Ib * sqrt((x(5) \symbol{94} 2 + 2 * sigma\_X \symbol{94} 2)) * (x(5) \symbol{94} 4) + 0.36e2 * x(3) * sqrt((x(5) \symbol{94} 2 + sigma\_X \symbol{94} 2)) * (sigma\_X \symbol{94} 11) * Ib * sqrt((x(5) \symbol{94} 2 + 2 * sigma\_X \symbol{94} 2)) * (x(5) \symbol{94} 2) + x(3) * sqrt((x(5) \symbol{94} 2 + sigma\_X \symbol{94} 2)) * sigma\_X * Ib * sqrt((x(5) \symbol{94} 2 + 2 * sigma\_X \symbol{94} 2)) * (x(5) \symbol{94} 12) + 0.9e1 * x(3) * sqrt((x(5) \symbol{94} 2 + sigma\_X \symbol{94} 2)) * (sigma\_X \symbol{94} 3) * Ib * sqrt((x(5) \symbol{94} 2 + 2 * sigma\_X \symbol{94} 2)) * (x(5) \symbol{94} 10) + 0.8e1 * Ixi * sqrt((x(5) \symbol{94} 2 + sigma\_X \symbol{94} 2)) * (sigma\_X \symbol{94} 13) * sqrt((x(5) \symbol{94} 2 + 2 * sigma\_X \symbol{94} 2)) - 0.32e2 * exp(-((x(6) - tau0) \symbol{94} 2 / (x(5) \symbol{94} 2 + 2 * sigma\_X \symbol{94} 2))) * h * (sigma\_X \symbol{94} 8) * (x(5) \symbol{94} 3) * x(1) \symbol{94} 2 * sqrt(0.2e1) * sqrt(pi) * sqrt((x(5) \symbol{94} 2 + sigma\_X \symbol{94} 2)) * (x(6) \symbol{94} 3) * x(4) + 0.96e2 * exp(-((x(6) - tau0) \symbol{94} 2 / (x(5) \symbol{94} 2 + 2 * sigma\_X \symbol{94} 2))) * h * (sigma\_X \symbol{94}6) * (x(5) \symbol{94} 5) * x(1) \symbol{94} 2 * sqrt(0.2e1) * sqrt(pi) * sqrt((x(5) \symbol{94} 2 + sigma\_X \symbol{94} 2)) * (tau0 \symbol{94} 3) * x(4) + 0.32e2 * exp(-((x(6) - tau0) \symbol{94} 2 / (x(5) \symbol{94} 2 + 2 * sigma\_X \symbol{94} 2))) * h * (sigma\_X \symbol{94} 8) * (x(5) \symbol{94} 3) * x(1) \symbol{94} 2 * sqrt(0.2e1) * sqrt(pi) * sqrt((x(5)\symbol{94} 2 + sigma\_X \symbol{94} 2)) * (tau0 \symbol{94} 3) * x(4) + 0.8e1 * exp(-((x(6) - tau0) \symbol{94} 2 / (x(5) \symbol{94} 2 + 2 * sigma\_X \symbol{94} 2))) * h * (sigma\_X \symbol{94} 2) * (x(5) \symbol{94} 9) * x(1) \symbol{94} 2 * sqrt(0.2e1) * sqrt(pi) * sqrt((x(5) \symbol{94} 2 + sigma\_X \symbol{94} 2)) * (x(6) \symbol{94} 2) * x(3) + 0.16e2 * exp(-((x(6) - tau0) \symbol{94} 2 / (x(5) \symbol{94} 2 + 2 * sigma\_X \symbol{94} 2))) * h * (sigma\_X \symbol{94} 10) * x(5) * x(1) \symbol{94} 2 * sqrt(0.2e1) * sqrt(pi) * sqrt((x(5) \symbol{94} 2 + sigma\_X \symbol{94} 2)) * (tau0 \symbol{94} 2) * x(3) + 0.56e2 * exp(-((x(6) - tau0) \symbol{94} 2 / (x(5) \symbol{94} 2 + 2 * sigma\_X \symbol{94} 2))) * h * (sigma\_X \symbol{94} 8) * (x(5) \symbol{94}3) * x(1) \symbol{94} 2 * sqrt(0.2e1) * sqrt(pi) * sqrt((x(5) \symbol{94} 2 + sigma\_X \symbol{94} 2)) * (tau0 \symbol{94} 2) * x(3) + 0.96e2 * exp(-((x(6) - tau0) \symbol{94} 2 / (x(5) \symbol{94} 2 + 2 * sigma\_X \symbol{94} 2))) * h * (sigma\_X \symbol{94} 4) * (x(5) \symbol{94} 7) * x(1) \symbol{94} 2 * sqrt(0.2e1) * sqrt(pi) * sqrt((x(5) \symbol{94} 2 + sigma\_X\symbol{94} 2)) * (tau0 \symbol{94} 3) * x(4) + 0.48e2 * exp(-((x(6) - tau0) \symbol{94} 2 / (x(5) \symbol{94} 2 + 2 * sigma\_X \symbol{94} 2))) * h * (sigma\_X \symbol{94} 2) * (x(5) \symbol{94} 11) * x(1) \symbol{94} 2 * sqrt(0.2e1) * sqrt(pi) * sqrt((x(5) \symbol{94} 2 + sigma\_X \symbol{94} 2)) * x(4) * x(6) + 0.240e3 * exp(-((x(6) - tau0) \symbol{94} 2 / (x(5) \symbol{94}2 + 2 * sigma\_X \symbol{94} 2))) * h * (sigma\_X \symbol{94} 4) * (x(5) \symbol{94} 9) * x(1) \symbol{94} 2 * sqrt(0.2e1) * sqrt(pi) * sqrt((x(5) \symbol{94} 2 + sigma\_X \symbol{94} 2)) * x(4) * x(6) + 0.432e3 * exp(-((x(6) - tau0) \symbol{94} 2 / (x(5) \symbol{94} 2 + 2 * sigma\_X \symbol{94} 2))) * h * (sigma\_X \symbol{94} 6) * (x(5) \symbol{94} 7) * x(1) \symbol{94} 2 * sqrt(0.2e1) * sqrt(pi) * sqrt((x(5) \symbol{94} 2 + sigma\_X \symbol{94} 2)) * x(4) * x(6) + 0.336e3 * exp(-((x(6) - tau0) \symbol{94} 2 / (x(5) \symbol{94} 2 + 2 * sigma\_X \symbol{94} 2))) * h * (sigma\_X \symbol{94} 8) * (x(5) \symbol{94} 5) * x(1) \symbol{94} 2 * sqrt(0.2e1) * sqrt(pi) * sqrt((x(5) \symbol{94} 2 + sigma\_X \symbol{94} 2)) * x(4) * x(6) +0.96e2 * exp(-((x(6) - tau0) \symbol{94} 2 / (x(5) \symbol{94} 2 + 2 * sigma\_X \symbol{94} 2))) * h * (sigma\_X \symbol{94} 10) * (x(5) \symbol{94} 3) * x(1) \symbol{94} 2 * sqrt(0.2e1) * sqrt(pi) * sqrt((x(5) \symbol{94} 2 + sigma\_X \symbol{94} 2)) * x(4) * x(6) - 0.48e2 * exp(-((x(6) - tau0) \symbol{94} 2 / (x(5) \symbol{94} 2 + 2 * sigma\_X \symbol{94} 2))) * h *(sigma\_X \symbol{94} 2) * (x(5) \symbol{94} 11) * x(1) \symbol{94} 2 * sqrt(0.2e1) * sqrt(pi) * sqrt((x(5) \symbol{94} 2 + sigma\_X \symbol{94} 2)) * x(4) * tau0 - 0.240e3 * exp(-((x(6) - tau0) \symbol{94} 2 / (x(5) \symbol{94} 2 + 2 * sigma\_X \symbol{94} 2))) * h * (sigma\_X \symbol{94} 4) * (x(5) \symbol{94} 9) * x(1) \symbol{94} 2 * sqrt(0.2e1) * sqrt(pi) * sqrt((x(5) \symbol{94} 2 + sigma\_X \symbol{94} 2)) * x(4) * tau0 - 0.432e3 * exp(-((x(6) - tau0) \symbol{94} 2 / (x(5) \symbol{94} 2 + 2 * sigma\_X \symbol{94} 2))) * h * (sigma\_X \symbol{94} 6) * (x(5) \symbol{94} 7) * x(1) \symbol{94} 2 * sqrt(0.2e1) * sqrt(pi) * sqrt((x(5) \symbol{94} 2 + sigma\_X \symbol{94} 2)) * x(4) * tau0 - 0.336e3 * exp(-((x(6) - tau0) \symbol{94} 2 / (x(5) \symbol{94} 2 + 2 * sigma\_X \symbol{94} 2))) * h * (sigma\_X \symbol{94} 8) * (x(5) \symbol{94} 5) * x(1) \symbol{94} 2 * sqrt(0.2e1) * sqrt(pi) * sqrt((x(5) \symbol{94} 2 + sigma\_X \symbol{94} 2)) * x(4) * tau0 - 0.96e2 * exp(-((x(6) - tau0) \symbol{94} 2 / (x(5) \symbol{94} 2 + 2 * sigma\_X \symbol{94} 2))) * h * (sigma\_X \symbol{94} 10) * (x(5) \symbol{94} 3)* x(1) \symbol{94} 2 * sqrt(0.2e1) * sqrt(pi) * sqrt((x(5) \symbol{94} 2 + sigma\_X \symbol{94} 2)) * x(4) * tau0 + 0.8e1 * exp(-((x(6) - tau0) \symbol{94} 2 / (x(5) \symbol{94} 2 + 2 * sigma\_X \symbol{94} 2))) * h * (sigma\_X \symbol{94} 2) * (x(5) \symbol{94} 9) * x(1) \symbol{94} 2 * sqrt(0.2e1) * sqrt(pi) * sqrt((x(5) \symbol{94} 2 + sigma\_X \symbol{94} 2)) * (tau0 \symbol{94} 2) * x(3) + 0.40e2 * exp(-((x(6) - tau0) \symbol{94} 2 / (x(5) \symbol{94} 2 + 2 * sigma\_X \symbol{94} 2))) * h * (sigma\_X \symbol{94} 4) * (x(5) \symbol{94} 7) * x(1) \symbol{94} 2 * sqrt(0.2e1) * sqrt(pi) * sqrt((x(5) \symbol{94} 2 + sigma\_X \symbol{94} 2)) * (tau0 \symbol{94} 2) * x(3) + 0.72e2 * exp(-((x(6) - tau0) \symbol{94} 2 / (x(5) \symbol{94} 2 +2 * sigma\_X \symbol{94} 2))) * h * (sigma\_X \symbol{94} 6) * (x(5) \symbol{94} 5) * x(1) \symbol{94} 2 * sqrt(0.2e1) * sqrt(pi) * sqrt((x(5) \symbol{94} 2 + sigma\_X \symbol{94} 2)) * (tau0 \symbol{94} 2) * x(3) + 0.32e2 * exp(-((x(6) - tau0) \symbol{94} 2 / (x(5) \symbol{94} 2 + 2 * sigma\_X \symbol{94} 2))) * h * (sigma\_X \symbol{94} 2) * (x(5) \symbol{94} 9) * x(1) \symbol{94} 2 * sqrt(0.2e1) * sqrt(pi) * sqrt((x(5) \symbol{94} 2 + sigma\_X \symbol{94} 2)) * (tau0 \symbol{94} 3) * x(4) + 0.40e2 * exp(-((x(6) - tau0) \symbol{94} 2 / (x(5) \symbol{94} 2 + 2 * sigma\_X \symbol{94} 2))) * h * (sigma\_X \symbol{94} 4) * (x(5) \symbol{94} 7) * x(1) \symbol{94} 2 * sqrt(0.2e1) * sqrt(pi) * sqrt((x(5) \symbol{94} 2 + sigma\_X \symbol{94} 2)) * (x(6) \symbol{94} 2) * x(3) + 0.72e2 * exp(-((x(6) - tau0) \symbol{94} 2 / (x(5) \symbol{94} 2 + 2 * sigma\_X \symbol{94} 2))) * h * (sigma\_X \symbol{94} 6) * (x(5) \symbol{94} 5) * x(1) \symbol{94} 2 * sqrt(0.2e1) * sqrt(pi) * sqrt((x(5) \symbol{94} 2 + sigma\_X \symbol{94} 2)) * (x(6) \symbol{94} 2) * x(3) + 0.56e2 * exp(-((x(6) - tau0) \symbol{94} 2 / (x(5) \symbol{94} 2 + 2 * sigma\_X \symbol{94} 2))) * h * (sigma\_X \symbol{94} 8) * (x(5) \symbol{94} 3) * x(1) \symbol{94} 2 * sqrt(0.2e1) * sqrt(pi) * sqrt((x(5) \symbol{94} 2 + sigma\_X \symbol{94} 2)) * (x(6) \symbol{94} 2) * x(3) + 0.16e2 * exp(-((x(6) - tau0) \symbol{94} 2 / (x(5) \symbol{94} 2 + 2 * sigma\_X \symbol{94} 2))) * h * (sigma\_X \symbol{94} 10) * x(5) * x(1) \symbol{94} 2 * sqrt(0.2e1) *sqrt(pi) * sqrt((x(5) \symbol{94} 2 + sigma\_X \symbol{94} 2)) * (x(6) \symbol{94} 2) * x(3) - 0.32e2 * exp(-((x(6) - tau0) \symbol{94} 2 / (x(5) \symbol{94} 2 + 2 * sigma\_X \symbol{94} 2))) * h * (sigma\_X \symbol{94} 2) * (x(5) \symbol{94} 9) * x(1) \symbol{94} 2 * sqrt(0.2e1) * sqrt(pi) * sqrt((x(5) \symbol{94} 2 + sigma\_X \symbol{94} 2)) * (x(6) \symbol{94} 3) * x(4) - 0.96e2 * exp(-((x(6) - tau0) \symbol{94} 2 / (x(5) \symbol{94} 2 + 2 * sigma\_X \symbol{94} 2))) * h * (sigma\_X \symbol{94} 4) * (x(5) \symbol{94} 7) * x(1) \symbol{94} 2 * sqrt(0.2e1) * sqrt(pi) * sqrt((x(5) \symbol{94} 2 + sigma\_X \symbol{94} 2)) * (x(6) \symbol{94} 3) * x(4) - 0.96e2 * exp(-((x(6) - tau0) \symbol{94} 2 / (x(5) \symbol{94} 2 + 2 * sigma\_X \symbol{94} 2))) *h * (sigma\_X \symbol{94} 6) * (x(5) \symbol{94} 5) * x(1) \symbol{94} 2 * sqrt(0.2e1) * sqrt(pi) * sqrt((x(5) \symbol{94} 2 + sigma\_X \symbol{94} 2)) * (x(6) \symbol{94} 3) * x(4) - 0.4e1 * exp(-((x(6) - tau0) \symbol{94} 2 / (x(5) \symbol{94} 2 + 2 * sigma\_X \symbol{94} 2))) * h * (sigma\_X \symbol{94} 2) * (x(5) \symbol{94} 11) * x(1) \symbol{94} 2 * sqrt(0.2e1) * sqrt(pi)* sqrt((x(5) \symbol{94} 2 + sigma\_X \symbol{94} 2)) * x(3) - 0.6e1 * exp(-((x(6) - tau0) \symbol{94} 2 / (x(5) \symbol{94} 2 + sigma\_X \symbol{94} 2))) * h \symbol{94} 2 * (sigma\_X \symbol{94} 2) * (x(5) \symbol{94} 7) * x(1) \symbol{94} 2 * sqrt(pi) * sqrt((x(5) \symbol{94} 2 + 2 * sigma\_X \symbol{94} 2)) * (tau0 \symbol{94} 2) * x(6) - 0.36e2 * exp(-((x(6) - tau0) \symbol{94} 2 /(x(5) \symbol{94} 2 + sigma\_X \symbol{94} 2))) * h \symbol{94} 2 * (sigma\_X \symbol{94} 4) * (x(5) \symbol{94} 5) * x(1) \symbol{94} 2 * sqrt(pi) * sqrt((x(5) \symbol{94} 2 + 2 * sigma\_X \symbol{94} 2)) * (tau0 \symbol{94} 2) * x(6) - 0.72e2 * exp(-((x(6) - tau0) \symbol{94} 2 / (x(5) \symbol{94} 2 + sigma\_X \symbol{94} 2))) * h \symbol{94} 2 * (sigma\_X \symbol{94} 6) * (x(5) \symbol{94} 3) * x(1) \symbol{94} 2* sqrt(pi) * sqrt((x(5) \symbol{94} 2 + 2 * sigma\_X \symbol{94} 2)) * (tau0 \symbol{94} 2) * x(6) - 0.48e2 * exp(-((x(6) - tau0) \symbol{94} 2 / (x(5) \symbol{94} 2 + sigma\_X \symbol{94} 2))) * h \symbol{94} 2 * (sigma\_X \symbol{94} 8) * x(5) * x(1) \symbol{94} 2 * sqrt(pi) * sqrt((x(5) \symbol{94} 2 + 2 * sigma\_X \symbol{94} 2)) * (tau0 \symbol{94} 2) * x(6) - 0.28e2 * exp(-((x(6) - tau0) \symbol{94} 2 / (x(5) \symbol{94} 2 + 2 * sigma\_X \symbol{94} 2))) * h * (sigma\_X \symbol{94} 4) * (x(5) \symbol{94} 9) * x(1) \symbol{94} 2 * sqrt(0.2e1) * sqrt(pi) * sqrt((x(5) \symbol{94} 2 + sigma\_X \symbol{94} 2)) * x(3) - 0.76e2 * exp(-((x(6) - tau0) \symbol{94} 2 / (x(5) \symbol{94} 2 + 2 * sigma\_X \symbol{94} 2))) * h * (sigma\_X \symbol{94} 6) * (x(5) \symbol{94} 7) * x(1) \symbol{94} 2 * sqrt(0.2e1) * sqrt(pi) * sqrt((x(5) \symbol{94} 2 + sigma\_X \symbol{94} 2)) * x(3) - 0.100e3 * exp(-((x(6) - tau0) \symbol{94} 2 / (x(5) \symbol{94} 2 + 2 * sigma\_X \symbol{94} 2))) * h * (sigma\_X \symbol{94} 8) * (x(5) \symbol{94} 5) * x(1) \symbol{94} 2 * sqrt(0.2e1) * sqrt(pi) * sqrt((x(5) \symbol{94} 2 + sigma\_X \symbol{94} 2)) *x(3) - 0.64e2 * exp(-((x(6) - tau0) \symbol{94} 2 / (x(5) \symbol{94} 2 + 2 * sigma\_X \symbol{94} 2))) * h * (sigma\_X \symbol{94} 10) * (x(5) \symbol{94} 3) * x(1) \symbol{94} 2 * sqrt(0.2e1) * sqrt(pi) * sqrt((x(5) \symbol{94} 2 + sigma\_X \symbol{94} 2)) * x(3) + 0.6e1 * exp(-((x(6) - tau0) \symbol{94} 2 / (x(5) \symbol{94} 2 + sigma\_X \symbol{94} 2))) * h \symbol{94} 2 *(sigma\_X \symbol{94} 2) * (x(5) \symbol{94} 7) * x(1) \symbol{94} 2 * sqrt(pi) * sqrt((x(5) \symbol{94} 2 + 2 * sigma\_X \symbol{94} 2)) * tau0 * (x(6) \symbol{94} 2) + 0.36e2 * exp(-((x(6) - tau0) \symbol{94} 2 / (x(5) \symbol{94} 2 + sigma\_X \symbol{94} 2))) * h \symbol{94} 2 * (sigma\_X \symbol{94} 4) * (x(5) \symbol{94} 5) * x(1) \symbol{94} 2 * sqrt(pi) * sqrt((x(5) \symbol{94} 2 + 2 * sigma\_X \symbol{94} 2)) * tau0 * (x(6) \symbol{94} 2) + 0.72e2 * exp(-((x(6) - tau0) \symbol{94} 2 / (x(5) \symbol{94} 2 + sigma\_X \symbol{94} 2))) * h \symbol{94} 2 * (sigma\_X \symbol{94} 6) * (x(5) \symbol{94} 3) * x(1) \symbol{94} 2 * sqrt(pi) * sqrt((x(5) \symbol{94} 2 + 2 * sigma\_X \symbol{94} 2)) * tau0 * (x(6) \symbol{94} 2) + 0.48e2 * exp(-((x(6) - tau0) \symbol{94} 2 / (x(5) \symbol{94}2 + sigma\_X \symbol{94} 2))) * h \symbol{94} 2 * (sigma\_X \symbol{94} 8) * x(5) * x(1) \symbol{94} 2 * sqrt(pi) * sqrt((x(5) \symbol{94} 2 + 2 * sigma\_X \symbol{94} 2)) * tau0 * (x(6) \symbol{94} 2) - 0.16e2 * exp(-((x(6) - tau0) \symbol{94} 2 / (x(5) \symbol{94} 2 + 2 * sigma\_X \symbol{94} 2))) * h * (sigma\_X \symbol{94} 12) * x(5) * x(1) \symbol{94} 2 * sqrt(0.2e1) * sqrt(pi) * sqrt((x(5) \symbol{94} 2 + sigma\_X \symbol{94} 2)) * x(3) + Ixi * sqrt((x(5) \symbol{94} 2 + sigma\_X \symbol{94} 2)) * sigma\_X * sqrt((x(5) \symbol{94} 2 + 2 * sigma\_X \symbol{94} 2)) * (x(5) \symbol{94} 12) + 0.9e1 * Ixi * sqrt((x(5) \symbol{94} 2 + sigma\_X \symbol{94} 2)) * (sigma\_X \symbol{94} 3) * sqrt((x(5) \symbol{94} 2 + 2 * sigma\_X \symbol{94} 2)) * (x(5) \symbol{94} 10) + 0.33e2 * Ixi * sqrt((x(5) \symbol{94} 2 + sigma\_X \symbol{94} 2)) * (sigma\_X \symbol{94} 5) * sqrt((x(5) \symbol{94} 2 + 2 * sigma\_X \symbol{94} 2)) * (x(5) \symbol{94} 8) + 0.63e2 * Ixi * sqrt((x(5) \symbol{94} 2 + sigma\_X \symbol{94} 2)) * (sigma\_X \symbol{94} 7) * sqrt((x(5) \symbol{94} 2 + 2 * sigma\_X \symbol{94} 2)) * (x(5) \symbol{94} 6) + 0.66e2 * Ixi * sqrt((x(5) \symbol{94} 2 + sigma\_X \symbol{94} 2)) * (sigma\_X \symbol{94} 9) * sqrt((x(5) \symbol{94} 2 + 2 * sigma\_X \symbol{94} 2)) * (x(5) \symbol{94} 4) + 0.36e2 * Ixi * sqrt((x(5) \symbol{94} 2 + sigma\_X \symbol{94} 2)) * (sigma\_X \symbol{94} 11) * sqrt((x(5) \symbol{94} 2 + 2 * sigma\_X \symbol{94} 2)) * (x(5) \symbol{94} 2) + 0.8e1 * x(3) * sqrt((x(5) \symbol{94} 2 + sigma\_X \symbol{94} 2)) * (sigma\_X \symbol{94} 13) * Ib * sqrt((x(5) \symbol{94} 2 + 2 * sigma\_X \symbol{94} 2)) - exp(-((x(6) - tau0) \symbol{94} 2 / (x(5) \symbol{94} 2 + sigma\_X \symbol{94} 2))) * h \symbol{94} 2 * x(1) \symbol{94} 2 * (x(5) \symbol{94} 11) * sqrt(pi) * sqrt((x(5) \symbol{94} 2 + 2 * sigma\_X \symbol{94} 2)) * x(6) + exp(-((x(6) - tau0) \symbol{94} 2 / (x(5) \symbol{94} 2 + sigma\_X \symbol{94} 2))) * h\symbol{94} 2 * x(1) \symbol{94} 2 * (x(5) \symbol{94} 11) * sqrt(pi) * sqrt((x(5) \symbol{94} 2 + 2 * sigma\_X \symbol{94} 2)) * tau0) / x(1) \symbol{94} 2 / x(5) * pi \symbol{94} (-0.1e1 / 0.2e1) * ((x(5) \symbol{94} 2 + sigma\_X \symbol{94} 2) \symbol{94} (-0.7e1 / 0.2e1)) / sigma\_X * ((x(5) \symbol{94} 2 + 2 * sigma\_X \symbol{94} 2) \symbol{94} (-0.7e1 / 0.2e1));
\underline{}Warning, the function names \{x\} are not recognized in the target language\underline{}\mapleresult
cg2 = 0.0e0 == (-sqrt(0.2e1) / ((x(5) \symbol{94} 2 + sigma\_X \symbol{94} 2) \symbol{94} 4) / ((x(5) \symbol{94} 2 + 2 * sigma\_X \symbol{94} 2) \symbol{94} 4) * (x(5) \symbol{94} 14) / 0.4e1 - 0.3e1 * sqrt(0.2e1) / ((x(5) \symbol{94} 2 + sigma\_X \symbol{94} 2) \symbol{94} 4) * (sigma\_X \symbol{94} 2) / ((x(5) \symbol{94} 2 + 2 * sigma\_X \symbol{94} 2) \symbol{94} 4) * (x(5) \symbol{94} 12) - 0.31e2 / 0.2e1 * sqrt(0.2e1) / ((x(5) \symbol{94} 2 + sigma\_X \symbol{94} 2) \symbol{94} 4) * (sigma\_X \symbol{94} 4) / ((x(5) \symbol{94} 2 + 2 * sigma\_X \symbol{94} 2) \symbol{94} 4) * (x(5) \symbol{94} 10) - 0.45e2 * sqrt(0.2e1) / ((x(5) \symbol{94} 2 + sigma\_X \symbol{94} 2) \symbol{94} 4) * (sigma\_X \symbol{94} 6) / ((x(5) \symbol{94} 2 + 2 * sigma\_X \symbol{94} 2) \symbol{94} 4) * (x(5) \symbol{94} 8) - 0.321e3 / 0.4e1 * sqrt(0.2e1) / ((x(5) \symbol{94} 2 + sigma\_X \symbol{94} 2) \symbol{94} 4) * (sigma\_X \symbol{94} 8) / ((x(5) \symbol{94} 2 + 2 * sigma\_X \symbol{94} 2) \symbol{94} 4) * (x(5) \symbol{94} 6) - 0.90e2 * sqrt(0.2e1) / ((x(5) \symbol{94} 2 + sigma\_X \symbol{94} 2) \symbol{94} 4) * (sigma\_X \symbol{94} 10) / ((x(5) \symbol{94} 2 + 2 * sigma\_X \symbol{94} 2) \symbol{94} 4) * (x(5) \symbol{94} 4) - 0.62e2 * (sigma\_X \symbol{94} 12) * sqrt(0.2e1) / ((x(5) \symbol{94} 2 + sigma\_X \symbol{94} 2) \symbol{94} 4) / ((x(5) \symbol{94} 2 + 2 * sigma\_X \symbol{94} 2) \symbol{94} 4) * (x(5) \symbol{94} 2) - 0.24e2 * (sigma\_X \symbol{94} 14) * sqrt(0.2e1) / ((x(5) \symbol{94} 2 + sigma\_X \symbol{94} 2) \symbol{94} 4) / ((x(5) \symbol{94} 2 + 2 * sigma\_X \symbol{94} 2) \symbol{94} 4) - 0.4e1 / ((x(5) \symbol{94} 2 + sigma\_X \symbol{94} 2) \symbol{94} 4)/ ((x(5) \symbol{94} 2 + 2 * sigma\_X \symbol{94} 2) \symbol{94} 4) * sqrt(0.2e1) * (sigma\_X \symbol{94} 16) / (x(5) \symbol{94} 2)) * x(1) \symbol{94} 2 - (4 * x(4) \symbol{94} 2 / (x(5) \symbol{94} 2 + sigma\_X \symbol{94} 2) \symbol{94} 4 / (x(5) \symbol{94} 2 + 2 * sigma\_X \symbol{94} 2) \symbol{94} 4 * x(5) \symbol{94} 16) + (0.2e1 * exp(-((x(6) - tau0) \symbol{94} 2 / (x(5) \symbol{94} 2 + sigma\_X \symbol{94} 2))) * h\symbol{94} 2 * sqrt(pi) * sqrt((x(5) \symbol{94} 2 + 2 * sigma\_X \symbol{94} 2)) - 0.192e3 * (x(4) \symbol{94} 2) * sqrt(pi) * sqrt((x(5) \symbol{94} 2 + sigma\_X \symbol{94} 2)) * (sigma\_X \symbol{94} 3) * sqrt((x(5) \symbol{94} 2 + 2 * sigma\_X \symbol{94} 2)) - 0.16e2 * exp(-((x(6) - tau0) \symbol{94} 2 / (x(5) \symbol{94} 2 + 2 * sigma\_X \symbol{94} 2))) * h * (sigma\_X \symbol{94}2) * x(4) * sqrt(0.2e1) * sqrt(pi) * sqrt((x(5) \symbol{94} 2 + sigma\_X \symbol{94} 2))) * pi \symbol{94} (-0.1e1 / 0.2e1) * ((x(5) \symbol{94} 2 + sigma\_X \symbol{94} 2) \symbol{94} (-0.9e1 / 0.2e1)) / sigma\_X * ((x(5) \symbol{94} 2 + 2 * sigma\_X \symbol{94} 2) \symbol{94} (-0.9e1 / 0.2e1)) * (x(5) \symbol{94} 14) / 0.4e1 + (0.8e1 * exp(-((x(6) - tau0)\symbol{94} 2 / (x(5) \symbol{94} 2 + sigma\_X \symbol{94} 2))) * h \symbol{94} 2 * sqrt(pi) * sqrt((x(5) \symbol{94} 2 + 2 * sigma\_X \symbol{94} 2)) * tau0 * x(6) + 0.4e1 * sqrt(pi) * sqrt((x(5) \symbol{94} 2 + sigma\_X \symbol{94} 2)) * sigma\_X * sqrt((x(5) \symbol{94} 2 + 2 * sigma\_X \symbol{94} 2)) + 0.16e2 * exp(-((x(6) - tau0) \symbol{94} 2 / (x(5) \symbol{94} 2 + sigma\_X \symbol{94} 2))) * h \symbol{94} 2 * sqrt(pi) * sqrt((x(5) \symbol{94} 2 + 2 * sigma\_X \symbol{94} 2)) * (sigma\_X \symbol{94} 2) - 0.64e2 * exp(-((x(6) - tau0) \symbol{94} 2 / (x(5) \symbol{94} 2 + 2 * sigma\_X \symbol{94} 2))) * h * (sigma\_X \symbol{94} 4) * x(4) * sqrt(0.2e1) * sqrt(pi) * sqrt((x(5) \symbol{94} 2 + sigma\_X \symbol{94} 2)) + 0.128e3 * exp(-((x(6) - tau0) \symbol{94} 2 / (x(5) \symbol{94} 2 + 2 * sigma\_X \symbol{94} 2))) * h * (sigma\_X \symbol{94} 2) * sqrt(0.2e1) * sqrt(pi) * sqrt((x(5) \symbol{94} 2 + sigma\_X \symbol{94} 2)) * x(4) * (tau0 \symbol{94} 2) + 0.128e3 * exp(-((x(6) - tau0) \symbol{94} 2 / (x(5) \symbol{94} 2 + 2 * sigma\_X \symbol{94} 2))) * h * (sigma\_X \symbol{94} 2) * sqrt(0.2e1) * sqrt(pi) * sqrt((x(5) \symbol{94} 2 + sigma\_X \symbol{94} 2)) * x(4) * (x(6) \symbol{94} 2) - 0.48e2 * exp(-((x(6) - tau0) \symbol{94} 2 / (x(5) \symbol{94} 2 + 2 * sigma\_X \symbol{94} 2))) * h * (sigma\_X \symbol{94} 2) * sqrt(0.2e1) * sqrt(pi) * sqrt((x(5) \symbol{94} 2 + sigma\_X \symbol{94} 2)) * x(6) * x(3) - 0.992e3 * (x(4) \symbol{94} 2) * sqrt(pi) * sqrt((x(5) \symbol{94} 2 + sigma\_X \symbol{94} 2)) * (sigma\_X \symbol{94} 5) * sqrt((x(5) \symbol{94} 2 + 2 * sigma\_X \symbol{94} 2)) - 0.256e3 * exp(-((x(6) - tau0) \symbol{94} 2 / (x(5) \symbol{94} 2 + 2 * sigma\_X \symbol{94} 2))) * h * (sigma\_X \symbol{94} 2) * sqrt(0.2e1) * sqrt(pi) * sqrt((x(5) \symbol{94} 2 + sigma\_X \symbol{94} 2)) * x(6) * x(4) * tau0 - 0.4e1 * exp(-((x(6) - tau0) \symbol{94} 2 / (x(5) \symbol{94} 2 + sigma\_X \symbol{94} 2))) * h \symbol{94} 2 * sqrt(pi) * sqrt((x(5) \symbol{94} 2 + 2 * sigma\_X \symbol{94} 2)) * (tau0 \symbol{94} 2) - 0.4e1 * exp(-((x(6) - tau0) \symbol{94} 2 / (x(5) \symbol{94} 2 + sigma\_X \symbol{94} 2))) * h \symbol{94} 2 * sqrt(pi) * sqrt((x(5) \symbol{94} 2 + 2 * sigma\_X \symbol{94} 2)) * (x(6) \symbol{94} 2) + 0.48e2 * exp(-((x(6) - tau0) \symbol{94} 2 / (x(5) \symbol{94} 2 + 2 * sigma\_X \symbol{94} 2))) * h * (sigma\_X \symbol{94} 2) * sqrt(0.2e1) * sqrt(pi) * sqrt((x(5) \symbol{94} 2 + sigma\_X \symbol{94} 2)) * tau0 * x(3)) * pi \symbol{94} (-0.1e1 / 0.2e1) * ((x(5) \symbol{94} 2 + sigma\_X \symbol{94} 2) \symbol{94} (-0.9e1 / 0.2e1)) / sigma\_X * ((x(5) \symbol{94}2 + 2 * sigma\_X \symbol{94} 2) \symbol{94} (-0.9e1 / 0.2e1)) * (x(5) \symbol{94} 12) / 0.4e1 + (-0.384e3 * exp(-((x(6) - tau0) \symbol{94} 2 / (x(5) \symbol{94} 2 + 2 * sigma\_X \symbol{94} 2))) * h * (sigma\_X \symbol{94} 2) * sqrt(0.2e1) * sqrt(pi) * sqrt((x(5) \symbol{94} 2 + sigma\_X \symbol{94} 2)) * (tau0 \symbol{94} 2) * (x(6) \symbol{94} 2) * x(4) - 0.64e2 *exp(-((x(6) - tau0) \symbol{94} 2 / (x(5) \symbol{94} 2 + 2 * sigma\_X \symbol{94} 2))) * h * (sigma\_X \symbol{94} 2) * sqrt(0.2e1) * sqrt(pi) * sqrt((x(5) \symbol{94} 2 + sigma\_X \symbol{94} 2)) * (x(6) \symbol{94} 4) * x(4) - 0.96e2 * sqrt((x(5) \symbol{94} 2 + sigma\_X \symbol{94} 2)) * (sigma\_X \symbol{94} 2) * x(3) * sqrt(pi) * exp(-((x(6) - tau0) \symbol{94} 2/ (x(5) \symbol{94} 2 + 2 * sigma\_X \symbol{94} 2))) * sqrt(0.2e1) * h * tau0 * (x(6) \symbol{94} 2) + 0.640e3 * exp(-((x(6) - tau0) \symbol{94} 2 / (x(5) \symbol{94} 2 + 2 * sigma\_X \symbol{94} 2))) * h * (sigma\_X \symbol{94} 4) * sqrt(0.2e1) * sqrt(pi) * sqrt((x(5) \symbol{94} 2 + sigma\_X \symbol{94} 2)) * x(4) * (tau0 \symbol{94} 2) + 0.96e2 * sqrt((x(5) \symbol{94} 2 + sigma\_X \symbol{94} 2)) * (sigma\_X \symbol{94} 2) * x(3) * sqrt(pi) * exp(-((x(6) - tau0) \symbol{94} 2 / (x(5) \symbol{94} 2 + 2 * sigma\_X \symbol{94} 2))) * sqrt(0.2e1) * h * (tau0 \symbol{94} 2) * x(6) + 0.32e2 * sqrt((x(5) \symbol{94} 2 + sigma\_X \symbol{94} 2)) * (sigma\_X \symbol{94} 2) * x(3) * sqrt(pi) * exp(-((x(6) - tau0) \symbol{94}2 / (x(5) \symbol{94} 2 + 2 * sigma\_X \symbol{94} 2))) * sqrt(0.2e1) * h * (x(6) \symbol{94} 3) - 0.2880e4 * (x(4) \symbol{94} 2) * sqrt(pi) * sqrt((x(5) \symbol{94} 2 + sigma\_X \symbol{94} 2)) * (sigma\_X \symbol{94} 7) * sqrt((x(5) \symbol{94} 2 + 2 * sigma\_X \symbol{94} 2)) - 0.16e2 * exp(-((x(6) - tau0) \symbol{94} 2 / (x(5) \symbol{94} 2 + sigma\_X \symbol{94} 2))) * h \symbol{94}2 * sqrt(pi) * sqrt((x(5) \symbol{94} 2 + 2 * sigma\_X \symbol{94} 2)) * (x(6) \symbol{94} 2) * (sigma\_X \symbol{94} 2) + 0.384e3 * exp(-((x(6) - tau0) \symbol{94} 2 / (x(5) \symbol{94} 2 + 2 * sigma\_X \symbol{94} 2))) * h * (sigma\_X \symbol{94} 4) * sqrt(0.2e1) * sqrt(pi) * sqrt((x(5) \symbol{94} 2 + sigma\_X \symbol{94} 2)) * tau0 * x(3) - 0.64e2 * exp(-((x(6) - tau0) \symbol{94} 2 / (x(5) \symbol{94} 2 + 2 * sigma\_X \symbol{94} 2))) * h * (sigma\_X \symbol{94} 2) * sqrt(0.2e1) * sqrt(pi) * sqrt((x(5) \symbol{94} 2 + sigma\_X \symbol{94} 2)) * (tau0 \symbol{94} 4) * x(4) + 0.42e2 * exp(-((x(6) - tau0) \symbol{94} 2 / (x(5) \symbol{94} 2 + sigma\_X \symbol{94} 2))) * h \symbol{94} 2 * sqrt(pi) * sqrt((x(5) \symbol{94} 2 + 2 *sigma\_X \symbol{94} 2)) * (sigma\_X \symbol{94} 4) - 0.16e2 * exp(-((x(6) - tau0) \symbol{94} 2 / (x(5) \symbol{94} 2 + sigma\_X \symbol{94} 2))) * h \symbol{94} 2 * sqrt(pi) * sqrt((x(5) \symbol{94} 2 + 2 * sigma\_X \symbol{94} 2)) * (tau0 \symbol{94} 2) * (sigma\_X \symbol{94} 2) + 0.256e3 * exp(-((x(6) - tau0) \symbol{94} 2 / (x(5) \symbol{94} 2 + 2 * sigma\_X \symbol{94} 2))) * h * (sigma\_X \symbol{94} 2) * sqrt(0.2e1) * sqrt(pi) * sqrt((x(5) \symbol{94} 2 + sigma\_X \symbol{94} 2)) * (tau0 \symbol{94} 3) * x(6) * x(4) - 0.1280e4 * exp(-((x(6) - tau0) \symbol{94} 2 / (x(5) \symbol{94} 2 + 2 * sigma\_X \symbol{94} 2))) * h * (sigma\_X \symbol{94} 4) * sqrt(0.2e1) * sqrt(pi) * sqrt((x(5) \symbol{94} 2 + sigma\_X \symbol{94} 2)) * x(6) * x(4) * tau0 - 0.384e3 * exp(-((x(6) - tau0) \symbol{94} 2 / (x(5) \symbol{94} 2 + 2 * sigma\_X \symbol{94} 2))) * h * (sigma\_X \symbol{94} 4) * sqrt(0.2e1) * sqrt(pi) * sqrt((x(5) \symbol{94} 2 + sigma\_X \symbol{94} 2)) * x(6) * x(3) + 0.256e3 * exp(-((x(6) - tau0) \symbol{94} 2 / (x(5) \symbol{94} 2 + 2 * sigma\_X \symbol{94} 2))) * h * (sigma\_X\symbol{94} 2) * sqrt(0.2e1) * sqrt(pi) * sqrt((x(5) \symbol{94} 2 + sigma\_X \symbol{94} 2)) * (x(6) \symbol{94} 3) * x(4) * tau0 + 0.48e2 * sqrt(pi) * sqrt((x(5) \symbol{94} 2 + sigma\_X \symbol{94} 2)) * (sigma\_X \symbol{94} 3) * sqrt((x(5) \symbol{94} 2 + 2 * sigma\_X \symbol{94} 2)) + 0.32e2 * sqrt((x(5) \symbol{94} 2 + 2 * sigma\_X \symbol{94} 2)) * exp(-((x(6)- tau0) \symbol{94} 2 / (x(5) \symbol{94} 2 + sigma\_X \symbol{94} 2))) * h \symbol{94} 2 * (sigma\_X \symbol{94} 2) * sqrt(pi) * tau0 * x(6) + 0.640e3 * exp(-((x(6) - tau0) \symbol{94} 2 / (x(5) \symbol{94} 2 + 2 * sigma\_X \symbol{94} 2))) * h * (sigma\_X \symbol{94} 4) * sqrt(0.2e1) * sqrt(pi) * sqrt((x(5) \symbol{94} 2 + sigma\_X \symbol{94} 2)) * x(4) * (x(6) \symbol{94} 2)- 0.32e2 * sqrt((x(5) \symbol{94} 2 + sigma\_X \symbol{94} 2)) * (sigma\_X \symbol{94} 2) * x(3) * sqrt(pi) * exp(-((x(6) - tau0) \symbol{94} 2 / (x(5) \symbol{94} 2 + 2 * sigma\_X \symbol{94} 2))) * sqrt(0.2e1) * h * (tau0 \symbol{94} 3) + 0.96e2 * exp(-((x(6) - tau0) \symbol{94} 2 / (x(5) \symbol{94} 2 + 2 * sigma\_X \symbol{94} 2))) * h * (sigma\_X \symbol{94} 6) *x(4) * sqrt(0.2e1) * sqrt(pi) * sqrt((x(5) \symbol{94} 2 + sigma\_X \symbol{94} 2))) * pi \symbol{94} (-0.1e1 / 0.2e1) * ((x(5) \symbol{94} 2 + sigma\_X \symbol{94} 2) \symbol{94} (-0.9e1 / 0.2e1)) / sigma\_X * ((x(5) \symbol{94} 2 + 2 * sigma\_X \symbol{94} 2) \symbol{94} (-0.9e1 / 0.2e1)) * (x(5) \symbol{94} 10) / 0.4e1 + (-0.256e3 * exp(-((x(6) - tau0) \symbol{94}2 / (x(5) \symbol{94} 2 + 2 * sigma\_X \symbol{94} 2))) * h * (sigma\_X \symbol{94} 4) * sqrt(0.2e1) * sqrt(pi) * sqrt((x(5) \symbol{94} 2 + sigma\_X \symbol{94} 2)) * (tau0 \symbol{94} 4) * x(4) + 0.192e3 * (sigma\_X \symbol{94} 4) * sqrt((x(5) \symbol{94} 2 + sigma\_X \symbol{94} 2)) * x(3) * sqrt(pi) * exp(-((x(6) - tau0) \symbol{94} 2 / (x(5) \symbol{94} 2 + 2 * sigma\_X \symbol{94} 2))) * sqrt(0.2e1) * h * (x(6) \symbol{94} 3) + 0.1024e4 * exp(-((x(6) - tau0) \symbol{94} 2 / (x(5) \symbol{94} 2 + 2 * sigma\_X \symbol{94} 2))) * h * (sigma\_X \symbol{94} 6) * sqrt(0.2e1) * sqrt(pi) * sqrt((x(5) \symbol{94} 2 + sigma\_X \symbol{94} 2)) * x(4) * (tau0 \symbol{94} 2) - 0.192e3 * sqrt((x(5) \symbol{94} 2 + sigma\_X \symbol{94} 2))* (sigma\_X \symbol{94} 4) * x(3) * sqrt(pi) * exp(-((x(6) - tau0) \symbol{94} 2 / (x(5) \symbol{94} 2 + 2 * sigma\_X \symbol{94} 2))) * sqrt(0.2e1) * h * (tau0 \symbol{94} 3) + 0.1248e4 * (sigma\_X \symbol{94} 6) * exp(-((x(6) - tau0) \symbol{94} 2 / (x(5) \symbol{94} 2 + 2 * sigma\_X \symbol{94} 2))) * h * sqrt(0.2e1) * sqrt(pi) * sqrt((x(5) \symbol{94} 2+ sigma\_X \symbol{94} 2)) * tau0 * x(3) - 0.1248e4 * (sigma\_X \symbol{94} 6) * exp(-((x(6) - tau0) \symbol{94} 2 / (x(5) \symbol{94} 2 + 2 * sigma\_X \symbol{94} 2))) * h * sqrt(0.2e1) * sqrt(pi) * sqrt((x(5) \symbol{94} 2 + sigma\_X \symbol{94} 2)) * x(6) * x(3) + 0.1024e4 * exp(-((x(6) - tau0) \symbol{94} 2 / (x(5) \symbol{94} 2 + 2 * sigma\_X \symbol{94}2))) * h * (sigma\_X \symbol{94} 6) * sqrt(0.2e1) * sqrt(pi) * sqrt((x(5) \symbol{94} 2 + sigma\_X \symbol{94} 2)) * x(4) * (x(6) \symbol{94} 2) - 0.256e3 * exp(-((x(6) - tau0) \symbol{94} 2 / (x(5) \symbol{94} 2 + 2 * sigma\_X \symbol{94} 2))) * h * (sigma\_X \symbol{94} 4) * sqrt(0.2e1) * sqrt(pi) * sqrt((x(5) \symbol{94} 2 + sigma\_X \symbol{94} 2)) * (x(6) \symbol{94} 4) * x(4) + 0.52e2 * exp(-((x(6) - tau0) \symbol{94} 2 / (x(5) \symbol{94} 2 + sigma\_X \symbol{94} 2))) * h \symbol{94} 2 * sqrt(pi) * sqrt((x(5) \symbol{94} 2 + 2 * sigma\_X \symbol{94} 2)) * (x(6) \symbol{94} 2) * (sigma\_X \symbol{94} 4) + 0.52e2 * exp(-((x(6) - tau0) \symbol{94} 2 / (x(5) \symbol{94} 2 + sigma\_X \symbol{94} 2))) * h \symbol{94} 2 * sqrt(pi) * sqrt((x(5) \symbol{94} 2 + 2 * sigma\_X \symbol{94} 2)) * (tau0 \symbol{94} 2) * (sigma\_X \symbol{94} 4) - 0.8e1 * exp(-((x(6) - tau0) \symbol{94} 2 / (x(5) \symbol{94} 2 + sigma\_X \symbol{94} 2))) * h \symbol{94} 2 * (sigma\_X \symbol{94} 2) * sqrt(pi) * sqrt((x(5) \symbol{94} 2 + 2 * sigma\_X \symbol{94} 2)) * (tau0 \symbol{94} 4) - 0.8e1 * exp(-((x(6) - tau0) \symbol{94} 2 / (x(5) \symbol{94} 2 + sigma\_X \symbol{94} 2))) * h \symbol{94} 2 * (sigma\_X \symbol{94} 2) * sqrt(pi) * sqrt((x(5) \symbol{94} 2 + 2 * sigma\_X \symbol{94} 2)) * (x(6) \symbol{94} 4) + 0.248e3 * sqrt(pi) * sqrt((x(5) \symbol{94} 2 + sigma\_X \symbol{94} 2)) * (sigma\_X \symbol{94} 5) * sqrt((x(5) \symbol{94} 2 + 2 * sigma\_X \symbol{94} 2)) + 0.12e2 * exp(-((x(6) - tau0) \symbol{94} 2 / (x(5) \symbol{94} 2 + sigma\_X \symbol{94} 2))) * h \symbol{94} 2 * sqrt(pi) * sqrt((x(5) \symbol{94} 2 + 2 * sigma\_X \symbol{94} 2)) * (sigma\_X \symbol{94} 6) + 0.1024e4 * exp(-((x(6) - tau0) \symbol{94} 2 / (x(5) \symbol{94} 2 + 2 * sigma\_X \symbol{94} 2))) * h * (sigma\_X \symbol{94} 4) * sqrt(0.2e1) * sqrt(pi) * sqrt((x(5) \symbol{94} 2 + sigma\_X \symbol{94} 2)) * (x(6) \symbol{94} 3) * x(4) * tau0 - 0.576e3 * sqrt((x(5) \symbol{94} 2 + sigma\_X \symbol{94} 2)) * (sigma\_X \symbol{94} 4) * x(3) * sqrt(pi) * exp(-((x(6) - tau0) \symbol{94} 2 / (x(5) \symbol{94} 2 + 2 * sigma\_X \symbol{94} 2))) * sqrt(0.2e1) * h * tau0 * (x(6) \symbol{94} 2) + 0.576e3 * (sigma\_X \symbol{94} 4) * sqrt((x(5) \symbol{94} 2 + sigma\_X \symbol{94} 2)) * x(3) * sqrt(pi) *exp(-((x(6) - tau0) \symbol{94} 2 / (x(5) \symbol{94} 2 + 2 * sigma\_X \symbol{94} 2))) * sqrt(0.2e1) * h * (tau0 \symbol{94} 2) * x(6) + 0.1024e4 * exp(-((x(6) - tau0) \symbol{94} 2 / (x(5) \symbol{94} 2 + 2 * sigma\_X \symbol{94} 2))) * h * (sigma\_X \symbol{94} 4) * sqrt(0.2e1) * sqrt(pi) * sqrt((x(5) \symbol{94} 2 + sigma\_X \symbol{94} 2)) * (tau0 \symbol{94} 3)* x(6) * x(4) - 0.1536e4 * exp(-((x(6) - tau0) \symbol{94} 2 / (x(5) \symbol{94} 2 + 2 * sigma\_X \symbol{94} 2))) * h * (sigma\_X \symbol{94} 4) * sqrt(0.2e1) * sqrt(pi) * sqrt((x(5) \symbol{94} 2 + sigma\_X \symbol{94} 2)) * (tau0 \symbol{94} 2) * (x(6) \symbol{94} 2) * x(4) - 0.2048e4 * exp(-((x(6) - tau0) \symbol{94} 2 / (x(5) \symbol{94} 2 + 2 * sigma\_X \symbol{94} 2))) * h * (sigma\_X \symbol{94} 6) * sqrt(0.2e1) * sqrt(pi) * sqrt((x(5) \symbol{94} 2 + sigma\_X \symbol{94} 2)) * x(6) * x(4) * tau0 - 0.5136e4 * (sigma\_X \symbol{94} 9) * (x(4) \symbol{94} 2) * sqrt(pi) * sqrt((x(5) \symbol{94} 2 + sigma\_X \symbol{94} 2)) * sqrt((x(5) \symbol{94} 2 + 2 * sigma\_X \symbol{94} 2)) + 0.32e2 * exp(-((x(6) - tau0) \symbol{94} 2 / (x(5) \symbol{94} 2 + sigma\_X \symbol{94} 2))) * h \symbol{94} 2 * (sigma\_X \symbol{94} 2) * sqrt(pi) * sqrt((x(5) \symbol{94} 2 + 2 * sigma\_X \symbol{94} 2)) * (x(6) \symbol{94} 3) * tau0 - 0.48e2 * exp(-((x(6) - tau0) \symbol{94} 2 / (x(5) \symbol{94} 2 + sigma\_X \symbol{94} 2))) * h \symbol{94} 2 * (sigma\_X \symbol{94} 2) * sqrt(pi) * sqrt((x(5) \symbol{94} 2 + 2 * sigma\_X \symbol{94} 2)) * (tau0 \symbol{94} 2) * (x(6) \symbol{94} 2) + 0.32e2 * exp(-((x(6) - tau0) \symbol{94} 2 / (x(5) \symbol{94} 2 + sigma\_X \symbol{94} 2))) * h \symbol{94} 2 * (sigma\_X \symbol{94} 2) * sqrt(pi) * sqrt((x(5) \symbol{94} 2 + 2 * sigma\_X \symbol{94} 2)) * (tau0 \symbol{94} 3) * x(6) - 0.104e3 * sqrt((x(5) \symbol{94} 2 + 2 * sigma\_X \symbol{94} 2)) * exp(-((x(6) - tau0) \symbol{94} 2 / (x(5) \symbol{94} 2 + sigma\_X \symbol{94} 2))) * h \symbol{94} 2 * (sigma\_X \symbol{94} 4) * sqrt(pi) * tau0 * x(6) + 0.960e3 * exp(-((x(6) - tau0) \symbol{94} 2 / (x(5) \symbol{94} 2 + 2 * sigma\_X \symbol{94} 2))) * h * (sigma\_X \symbol{94} 8) * x(4) * sqrt(0.2e1) * sqrt(pi) * sqrt((x(5) \symbol{94} 2 + sigma\_X \symbol{94} 2))) * pi \symbol{94} (-0.1e1/ 0.2e1) * ((x(5) \symbol{94} 2 + sigma\_X \symbol{94} 2) \symbol{94} (-0.9e1 / 0.2e1)) / sigma\_X * ((x(5) \symbol{94} 2 + 2 * sigma\_X \symbol{94} 2) \symbol{94} (-0.9e1 / 0.2e1)) * (x(5) \symbol{94} 8) / 0.4e1 + (-0.448e3 * sqrt((x(5) \symbol{94} 2 + sigma\_X \symbol{94} 2)) * (sigma\_X \symbol{94} 6) * x(3) * sqrt(pi) * exp(-((x(6) - tau0) \symbol{94} 2 / (x(5) \symbol{94} 2+ 2 * sigma\_X \symbol{94} 2))) * sqrt(0.2e1) * h * (tau0 \symbol{94} 3) + 0.256e3 * exp(-((x(6) - tau0) \symbol{94} 2 / (x(5) \symbol{94} 2 + 2 * sigma\_X \symbol{94} 2))) * h * (sigma\_X \symbol{94} 8) * sqrt(0.2e1) * sqrt(pi) * sqrt((x(5) \symbol{94} 2 + sigma\_X \symbol{94} 2)) * x(4) * (tau0 \symbol{94} 2) + 0.256e3 * exp(-((x(6) - tau0) \symbol{94} 2/ (x(5) \symbol{94} 2 + 2 * sigma\_X \symbol{94} 2))) * h * (sigma\_X \symbol{94} 8) * sqrt(0.2e1) * sqrt(pi) * sqrt((x(5) \symbol{94} 2 + sigma\_X \symbol{94} 2)) * x(4) * (x(6) \symbol{94} 2) - 0.384e3 * exp(-((x(6) - tau0) \symbol{94} 2 / (x(5) \symbol{94} 2 + 2 * sigma\_X \symbol{94} 2))) * h * (sigma\_X \symbol{94} 6) * sqrt(0.2e1) * sqrt(pi) * sqrt((x(5) \symbol{94} 2 + sigma\_X \symbol{94} 2)) * (x(6) \symbol{94} 4) * x(4) - 0.384e3 * exp(-((x(6) - tau0) \symbol{94} 2 / (x(5) \symbol{94} 2 + 2 * sigma\_X \symbol{94} 2))) * h * (sigma\_X \symbol{94} 6) * sqrt(0.2e1) * sqrt(pi) * sqrt((x(5) \symbol{94} 2 + sigma\_X \symbol{94} 2)) * (tau0 \symbol{94} 4) * x(4) + 0.448e3 * (sigma\_X \symbol{94} 6) * sqrt((x(5) \symbol{94} 2 + sigma\_X \symbol{94} 2)) * x(3) * sqrt(pi) * exp(-((x(6) - tau0) \symbol{94} 2 / (x(5) \symbol{94} 2 + 2 * sigma\_X \symbol{94} 2))) * sqrt(0.2e1) * h * (x(6) \symbol{94} 3) - 0.2112e4 * (sigma\_X \symbol{94} 8) * exp(-((x(6) - tau0) \symbol{94} 2 / (x(5) \symbol{94} 2 + 2 * sigma\_X \symbol{94} 2))) * h * sqrt(0.2e1) * sqrt(pi) * sqrt((x(5) \symbol{94} 2 + sigma\_X \symbol{94} 2)) * x(6) * x(3) + 0.2112e4 * (sigma\_X \symbol{94} 8) * exp(-((x(6) - tau0) \symbol{94} 2 / (x(5) \symbol{94} 2 + 2 * sigma\_X \symbol{94} 2))) * h * sqrt(0.2e1) * sqrt(pi) * sqrt((x(5) \symbol{94} 2 + sigma\_X \symbol{94} 2)) * tau0 * x(3) + 0.416e3 * exp(-((x(6) - tau0) \symbol{94} 2 / (x(5) \symbol{94} 2 + sigma\_X \symbol{94} 2))) * h\symbol{94} 2 * sqrt(pi) * sqrt((x(5) \symbol{94} 2 + 2 * sigma\_X \symbol{94} 2)) * (tau0 \symbol{94} 2) * (sigma\_X \symbol{94} 6) - 0.64e2 * exp(-((x(6) - tau0) \symbol{94} 2 / (x(5) \symbol{94} 2 + sigma\_X \symbol{94} 2))) * h \symbol{94} 2 * (sigma\_X \symbol{94} 4) * sqrt(pi) * sqrt((x(5) \symbol{94} 2 + 2 * sigma\_X \symbol{94} 2)) * (tau0 \symbol{94} 4) - 0.64e2 * exp(-((x(6) -tau0) \symbol{94} 2 / (x(5) \symbol{94} 2 + sigma\_X \symbol{94} 2))) * h \symbol{94} 2 * (sigma\_X \symbol{94} 4) * sqrt(pi) * sqrt((x(5) \symbol{94} 2 + 2 * sigma\_X \symbol{94} 2)) * (x(6) \symbol{94} 4) + 0.416e3 * exp(-((x(6) - tau0) \symbol{94} 2 / (x(5) \symbol{94} 2 + sigma\_X \symbol{94} 2))) * h \symbol{94} 2 * sqrt(pi) * sqrt((x(5) \symbol{94} 2 + 2 * sigma\_X \symbol{94} 2)) * (x(6) \symbol{94} 2) * (sigma\_X \symbol{94} 6) + 0.720e3 * (sigma\_X \symbol{94} 7) * sqrt(pi) * sqrt((x(5) \symbol{94} 2 + sigma\_X \symbol{94} 2)) * sqrt((x(5) \symbol{94} 2 + 2 * sigma\_X \symbol{94} 2)) - 0.2304e4 * exp(-((x(6) - tau0) \symbol{94} 2 / (x(5) \symbol{94} 2 + 2 * sigma\_X \symbol{94} 2))) * h * (sigma\_X \symbol{94} 6) * sqrt(0.2e1) * sqrt(pi) * sqrt((x(5) \symbol{94} 2+ sigma\_X \symbol{94} 2)) * (tau0 \symbol{94} 2) * (x(6) \symbol{94} 2) * x(4) - 0.512e3 * exp(-((x(6) - tau0) \symbol{94} 2 / (x(5) \symbol{94} 2 + 2 * sigma\_X \symbol{94} 2))) * h * (sigma\_X \symbol{94} 8) * sqrt(0.2e1) * sqrt(pi) * sqrt((x(5) \symbol{94} 2 + sigma\_X \symbol{94} 2)) * x(6) * x(4) * tau0 + 0.1536e4 * exp(-((x(6) - tau0) \symbol{94} 2 /(x(5) \symbol{94} 2 + 2 * sigma\_X \symbol{94} 2))) * h * (sigma\_X \symbol{94} 6) * sqrt(0.2e1) * sqrt(pi) * sqrt((x(5) \symbol{94} 2 + sigma\_X \symbol{94} 2)) * (x(6) \symbol{94} 3) * x(4) * tau0 + 0.1344e4 * (sigma\_X \symbol{94} 6) * sqrt((x(5) \symbol{94} 2 + sigma\_X \symbol{94} 2)) * x(3) * sqrt(pi) * exp(-((x(6) - tau0) \symbol{94} 2 / (x(5) \symbol{94} 2 + 2* sigma\_X \symbol{94} 2))) * sqrt(0.2e1) * h * (tau0 \symbol{94} 2) * x(6) + 0.1536e4 * exp(-((x(6) - tau0) \symbol{94} 2 / (x(5) \symbol{94} 2 + 2 * sigma\_X \symbol{94} 2))) * h * (sigma\_X \symbol{94} 6) * sqrt(0.2e1) * sqrt(pi) * sqrt((x(5) \symbol{94} 2 + sigma\_X \symbol{94} 2)) * (tau0 \symbol{94} 3) * x(6) * x(4) - 0.1344e4 * (sigma\_X \symbol{94} 6) * sqrt((x(5) \symbol{94} 2 + sigma\_X \symbol{94} 2)) * x(3) * sqrt(pi) * exp(-((x(6) - tau0) \symbol{94} 2 / (x(5) \symbol{94} 2 + 2 * sigma\_X \symbol{94} 2))) * sqrt(0.2e1) * h * tau0 * (x(6) \symbol{94} 2) - 0.144e3 * exp(-((x(6) - tau0) \symbol{94} 2 / (x(5) \symbol{94} 2 + sigma\_X \symbol{94} 2))) * h \symbol{94} 2 * sqrt(pi) * sqrt((x(5) \symbol{94} 2 + 2 *sigma\_X \symbol{94} 2)) * (sigma\_X \symbol{94} 8) - 0.5760e4 * (sigma\_X \symbol{94} 11) * (x(4) \symbol{94} 2) * sqrt(pi) * sqrt((x(5) \symbol{94} 2 + sigma\_X \symbol{94} 2)) * sqrt((x(5) \symbol{94} 2 + 2 * sigma\_X \symbol{94} 2)) + 0.2160e4 * exp(-((x(6) - tau0) \symbol{94} 2 / (x(5) \symbol{94} 2 + 2 * sigma\_X \symbol{94} 2))) * h * (sigma\_X \symbol{94} 10) * x(4) * sqrt(0.2e1) * sqrt(pi) * sqrt((x(5) \symbol{94} 2 + sigma\_X \symbol{94} 2)) - 0.832e3 * sqrt((x(5) \symbol{94} 2 + 2 * sigma\_X \symbol{94} 2)) * exp(-((x(6) - tau0) \symbol{94} 2 / (x(5) \symbol{94} 2 + sigma\_X \symbol{94} 2))) * h \symbol{94} 2 * (sigma\_X \symbol{94} 6) * sqrt(pi) * tau0 * x(6) - 0.384e3 * exp(-((x(6) - tau0) \symbol{94} 2 / (x(5) \symbol{94} 2 + sigma\_X \symbol{94} 2))) * h \symbol{94} 2 * (sigma\_X \symbol{94} 4) * sqrt(pi) * sqrt((x(5) \symbol{94} 2 + 2 * sigma\_X \symbol{94} 2)) * (tau0 \symbol{94} 2) * (x(6) \symbol{94} 2) + 0.256e3 * exp(-((x(6) - tau0) \symbol{94} 2 / (x(5) \symbol{94} 2 + sigma\_X \symbol{94} 2))) * h \symbol{94} 2 * (sigma\_X \symbol{94} 4) * sqrt(pi) * sqrt((x(5) \symbol{94} 2 + 2 * sigma\_X \symbol{94} 2)) * (tau0\symbol{94} 3) * x(6) + 0.256e3 * exp(-((x(6) - tau0) \symbol{94} 2 / (x(5) \symbol{94} 2 + sigma\_X \symbol{94} 2))) * h \symbol{94} 2 * (sigma\_X \symbol{94} 4) * sqrt(pi) * sqrt((x(5) \symbol{94} 2 + 2 * sigma\_X \symbol{94} 2)) * (x(6) \symbol{94} 3) * tau0) * pi \symbol{94} (-0.1e1 / 0.2e1) * ((x(5) \symbol{94} 2 + sigma\_X \symbol{94} 2) \symbol{94} (-0.9e1 / 0.2e1)) / sigma\_X * ((x(5) \symbol{94} 2 + 2 * sigma\_X \symbol{94} 2) \symbol{94} (-0.9e1 / 0.2e1)) * (x(5) \symbol{94} 6) / 0.4e1 + (-0.512e3 * (sigma\_X \symbol{94} 8) * sqrt((x(5) \symbol{94} 2 + sigma\_X \symbol{94} 2)) * x(3) * sqrt(pi) * exp(-((x(6) - tau0) \symbol{94} 2 / (x(5) \symbol{94} 2 + 2 * sigma\_X \symbol{94} 2))) * sqrt(0.2e1) * h * (tau0 \symbol{94} 3) - 0.256e3 * exp(-((x(6) - tau0) \symbol{94} 2 / (x(5) \symbol{94} 2 + 2 * sigma\_X \symbol{94} 2))) * h * (sigma\_X \symbol{94} 8) * sqrt(0.2e1) * sqrt(pi) * sqrt((x(5) \symbol{94} 2 + sigma\_X \symbol{94} 2)) * (x(6) \symbol{94} 4) * x(4) - 0.256e3 * exp(-((x(6) - tau0) \symbol{94} 2 / (x(5) \symbol{94} 2 + 2 * sigma\_X \symbol{94} 2))) * h * (sigma\_X \symbol{94} 8) * sqrt(0.2e1) * sqrt(pi) * sqrt((x(5) \symbol{94} 2 + sigma\_X \symbol{94} 2)) * (tau0 \symbol{94} 4) * x(4) + 0.1968e4 * (sigma\_X \symbol{94} 10) * exp(-((x(6) - tau0) \symbol{94} 2 / (x(5) \symbol{94} 2 + 2 * sigma\_X \symbol{94} 2))) * h * sqrt(0.2e1) * sqrt(pi) * sqrt((x(5) \symbol{94} 2 + sigma\_X \symbol{94} 2)) * tau0 * x(3) + 0.512e3 * (sigma\_X \symbol{94} 8) * sqrt((x(5) \symbol{94} 2 + sigma\_X \symbol{94} 2)) * x(3) * sqrt(pi) * exp(-((x(6) - tau0) \symbol{94} 2 / (x(5) \symbol{94} 2 + 2 * sigma\_X \symbol{94} 2))) * sqrt(0.2e1) * h * (x(6) \symbol{94} 3) - 0.1968e4 * (sigma\_X \symbol{94} 10) * exp(-((x(6) - tau0) \symbol{94} 2 / (x(5) \symbol{94} 2 + 2 * sigma\_X \symbol{94} 2))) * h * sqrt(0.2e1) * sqrt(pi) * sqrt((x(5) \symbol{94} 2 + sigma\_X \symbol{94} 2)) * x(6) * x(3) - 0.896e3 * exp(-((x(6) - tau0) \symbol{94} 2 / (x(5) \symbol{94} 2 + 2 * sigma\_X \symbol{94} 2))) * h * (sigma\_X \symbol{94} 10) * sqrt(0.2e1) * sqrt(pi) * sqrt((x(5) \symbol{94} 2 + sigma\_X \symbol{94} 2)) * x(4) * (x(6) \symbol{94} 2) - 0.896e3 * exp(-((x(6) - tau0) \symbol{94} 2 / (x(5) \symbol{94} 2+ 2 * sigma\_X \symbol{94} 2))) * h * (sigma\_X \symbol{94} 10) * sqrt(0.2e1) * sqrt(pi) * sqrt((x(5) \symbol{94} 2 + sigma\_X \symbol{94} 2)) * x(4) * (tau0 \symbol{94} 2) - 0.192e3 * exp(-((x(6) - tau0) \symbol{94} 2 / (x(5) \symbol{94} 2 + sigma\_X \symbol{94} 2))) * h \symbol{94} 2 * (sigma\_X \symbol{94} 6) * sqrt(pi) * sqrt((x(5) \symbol{94} 2 + 2 * sigma\_X \symbol{94} 2)) * (tau0 \symbol{94} 4) - 0.192e3 * exp(-((x(6) - tau0) \symbol{94} 2 / (x(5) \symbol{94} 2 + sigma\_X \symbol{94} 2))) * h \symbol{94} 2 * (sigma\_X \symbol{94} 6) * sqrt(pi) * sqrt((x(5) \symbol{94} 2 + 2 * sigma\_X \symbol{94} 2)) * (x(6) \symbol{94} 4) + 0.928e3 * exp(-((x(6) - tau0) \symbol{94} 2 / (x(5) \symbol{94} 2 + sigma\_X \symbol{94} 2))) * h \symbol{94} 2 * sqrt(pi) * sqrt((x(5) \symbol{94} 2 + 2 * sigma\_X \symbol{94} 2)) * (tau0 \symbol{94} 2) * (sigma\_X \symbol{94} 8) + 0.928e3 * exp(-((x(6) - tau0) \symbol{94} 2 / (x(5) \symbol{94} 2 + sigma\_X \symbol{94} 2))) * h \symbol{94} 2 * sqrt(pi) * sqrt((x(5) \symbol{94} 2 + 2 * sigma\_X \symbol{94} 2)) * (x(6) \symbol{94} 2) * (sigma\_X \symbol{94} 8) + 0.1284e4 * (sigma\_X \symbol{94} 9) * sqrt(pi) * sqrt((x(5) \symbol{94} 2 + sigma\_X \symbol{94} 2)) * sqrt((x(5) \symbol{94} 2 + 2 * sigma\_X \symbol{94} 2)) - 0.1536e4 * exp(-((x(6) - tau0) \symbol{94} 2 / (x(5) \symbol{94} 2 + 2 * sigma\_X \symbol{94} 2))) * h * (sigma\_X \symbol{94} 8) * sqrt(0.2e1) * sqrt(pi) * sqrt((x(5) \symbol{94} 2 + sigma\_X \symbol{94} 2)) * (tau0 \symbol{94} 2) * (x(6) \symbol{94} 2) * x(4) + 0.1792e4 * (sigma\_X \symbol{94} 10) * exp(-((x(6) - tau0) \symbol{94} 2 / (x(5) \symbol{94} 2 + 2 * sigma\_X \symbol{94} 2))) * h * sqrt(0.2e1) * sqrt(pi) * sqrt((x(5) \symbol{94} 2 + sigma\_X \symbol{94} 2)) * x(6) * x(4) * tau0 + 0.1024e4 * exp(-((x(6) - tau0) \symbol{94} 2 / (x(5) \symbol{94} 2 + 2 * sigma\_X \symbol{94} 2))) * h * (sigma\_X \symbol{94} 8) * sqrt(0.2e1) * sqrt(pi) * sqrt((x(5) \symbol{94} 2 + sigma\_X \symbol{94} 2)) * (tau0 \symbol{94} 3) * x(6) * x(4) + 0.1024e4 * exp(-((x(6) - tau0) \symbol{94} 2 / (x(5) \symbol{94} 2 + 2 * sigma\_X \symbol{94} 2))) * h * (sigma\_X \symbol{94} 8) * sqrt(0.2e1) * sqrt(pi) * sqrt((x(5) \symbol{94} 2 + sigma\_X \symbol{94} 2)) * (x(6) \symbol{94} 3) * x(4) * tau0 + 0.1536e4 * (sigma\_X \symbol{94} 8) * sqrt((x(5) \symbol{94} 2 + sigma\_X \symbol{94} 2)) * x(3) * sqrt(pi) * exp(-((x(6) - tau0) \symbol{94} 2 / (x(5) \symbol{94} 2 + 2 * sigma\_X \symbol{94} 2))) * sqrt(0.2e1) * h * (tau0 \symbol{94} 2) * x(6) - 0.1536e4 * (sigma\_X \symbol{94} 8) * sqrt((x(5) \symbol{94} 2 + sigma\_X \symbol{94} 2)) * x(3) * sqrt(pi) * exp(-((x(6) - tau0) \symbol{94} 2 / (x(5) \symbol{94} 2 + 2 * sigma\_X \symbol{94} 2))) * sqrt(0.2e1) * h * tau0 * (x(6) \symbol{94} 2) - 0.288e3 * (sigma\_X \symbol{94} 10) * exp(-((x(6) - tau0) \symbol{94} 2 / (x(5) \symbol{94} 2 + sigma\_X \symbol{94} 2))) * h \symbol{94} 2 * sqrt(pi) * sqrt((x(5) \symbol{94} 2 + 2 * sigma\_X \symbol{94} 2)) - 0.3968e4 * (sigma\_X \symbol{94} 13) *(x(4) \symbol{94} 2) * sqrt(pi) * sqrt((x(5) \symbol{94} 2 + sigma\_X \symbol{94} 2)) * sqrt((x(5) \symbol{94} 2 + 2 * sigma\_X \symbol{94} 2)) + 0.768e3 * exp(-((x(6) - tau0) \symbol{94} 2 / (x(5) \symbol{94} 2 + sigma\_X \symbol{94} 2))) * h \symbol{94} 2 * (sigma\_X \symbol{94} 6) * sqrt(pi) * sqrt((x(5) \symbol{94} 2 + 2 * sigma\_X \symbol{94} 2)) * (tau0 \symbol{94} 3) * x(6) + 0.768e3 * exp(-((x(6) - tau0) \symbol{94} 2 / (x(5) \symbol{94} 2 + sigma\_X \symbol{94} 2))) * h \symbol{94} 2 * (sigma\_X \symbol{94} 6) * sqrt(pi) * sqrt((x(5) \symbol{94} 2 + 2 * sigma\_X \symbol{94} 2)) * (x(6) \symbol{94} 3) * tau0 - 0.1856e4 * sqrt((x(5) \symbol{94} 2 + 2 * sigma\_X \symbol{94} 2)) * exp(-((x(6) - tau0) \symbol{94} 2 / (x(5) \symbol{94} 2 + sigma\_X \symbol{94} 2))) * h\symbol{94} 2 * (sigma\_X \symbol{94} 8) * sqrt(pi) * tau0 * x(6) + 0.2304e4 * (sigma\_X \symbol{94} 12) * exp(-((x(6) - tau0) \symbol{94} 2 / (x(5) \symbol{94} 2 + 2 * sigma\_X \symbol{94} 2))) * h * x(4) * sqrt(0.2e1) * sqrt(pi) * sqrt((x(5) \symbol{94} 2 + sigma\_X \symbol{94} 2)) - 0.1152e4 * exp(-((x(6) - tau0) \symbol{94} 2 / (x(5) \symbol{94} 2 + sigma\_X \symbol{94} 2))) * h \symbol{94} 2 * (sigma\_X \symbol{94} 6) * sqrt(pi) * sqrt((x(5) \symbol{94} 2 + 2 * sigma\_X \symbol{94} 2)) * (tau0 \symbol{94} 2) * (x(6) \symbol{94} 2)) * pi \symbol{94} (-0.1e1 / 0.2e1) * ((x(5) \symbol{94} 2 + sigma\_X \symbol{94} 2) \symbol{94} (-0.9e1 / 0.2e1)) / sigma\_X * ((x(5) \symbol{94} 2 + 2 * sigma\_X \symbol{94} 2) \symbol{94} (-0.9e1 / 0.2e1)) * (x(5) \symbol{94} 4) / 0.4e1 + (-0.896e3 * exp(-((x(6) - tau0) \symbol{94} 2 / (x(5) \symbol{94} 2 + 2 * sigma\_X \symbol{94} 2))) * h * (sigma\_X \symbol{94} 12) * sqrt(0.2e1) * sqrt(pi) * sqrt((x(5) \symbol{94} 2 + sigma\_X \symbol{94} 2)) * x(4) * (x(6) \symbol{94} 2) - 0.896e3 * exp(-((x(6) - tau0) \symbol{94} 2 / (x(5) \symbol{94} 2 + 2 * sigma\_X \symbol{94} 2))) * h * (sigma\_X \symbol{94} 12) * sqrt(0.2e1) * sqrt(pi) * sqrt((x(5) \symbol{94} 2 + sigma\_X \symbol{94} 2)) * x(4) * (tau0 \symbol{94} 2) - 0.64e2 * exp(-((x(6) - tau0) \symbol{94} 2 / (x(5) \symbol{94} 2 + 2 * sigma\_X \symbol{94} 2))) * h * (sigma\_X \symbol{94} 10) * sqrt(0.2e1) * sqrt(pi) * sqrt((x(5) \symbol{94} 2 + sigma\_X \symbol{94} 2)) * (x(6) \symbol{94} 4) * x(4) - 0.64e2 * exp(-((x(6) - tau0) \symbol{94} 2 / (x(5) \symbol{94} 2 + 2 * sigma\_X \symbol{94} 2))) * h * (sigma\_X \symbol{94} 10) * sqrt(0.2e1) * sqrt(pi) * sqrt((x(5) \symbol{94} 2 + sigma\_X \symbol{94} 2)) * (tau0 \symbol{94} 4) * x(4) - 0.960e3 * (sigma\_X \symbol{94} 12) * exp(-((x(6) - tau0) \symbol{94} 2 / (x(5) \symbol{94} 2 + 2 * sigma\_X \symbol{94} 2)))* h * sqrt(0.2e1) * sqrt(pi) * sqrt((x(5) \symbol{94} 2 + sigma\_X \symbol{94} 2)) * x(6) * x(3) + 0.960e3 * (sigma\_X \symbol{94} 12) * exp(-((x(6) - tau0) \symbol{94} 2 / (x(5) \symbol{94} 2 + 2 * sigma\_X \symbol{94} 2))) * h * sqrt(0.2e1) * sqrt(pi) * sqrt((x(5) \symbol{94} 2 + sigma\_X \symbol{94} 2)) * tau0 * x(3) - 0.288e3 * (sigma\_X \symbol{94} 10) * sqrt((x(5) \symbol{94} 2 + sigma\_X \symbol{94} 2)) * x(3) * sqrt(pi) * exp(-((x(6) - tau0) \symbol{94} 2 / (x(5) \symbol{94} 2 + 2 * sigma\_X \symbol{94} 2))) * sqrt(0.2e1) * h * (tau0 \symbol{94} 3) + 0.288e3 * (sigma\_X \symbol{94} 10) * sqrt((x(5) \symbol{94} 2 + sigma\_X \symbol{94} 2)) * x(3) * sqrt(pi) * exp(-((x(6) - tau0) \symbol{94} 2 /(x(5) \symbol{94} 2 + 2 * sigma\_X \symbol{94} 2))) * sqrt(0.2e1) * h * (x(6) \symbol{94} 3) - 0.256e3 * exp(-((x(6) - tau0) \symbol{94} 2 / (x(5) \symbol{94} 2 + sigma\_X \symbol{94} 2))) * h \symbol{94} 2 * (sigma\_X \symbol{94} 8) * sqrt(pi) * sqrt((x(5) \symbol{94} 2 + 2 * sigma\_X \symbol{94} 2)) * (tau0 \symbol{94} 4) - 0.256e3 * exp(-((x(6) - tau0) \symbol{94} 2 / (x(5)\symbol{94} 2 + sigma\_X \symbol{94} 2))) * h \symbol{94} 2 * (sigma\_X \symbol{94} 8) * sqrt(pi) * sqrt((x(5) \symbol{94} 2 + 2 * sigma\_X \symbol{94} 2)) * (x(6) \symbol{94} 4) + 0.896e3 * sqrt((x(5) \symbol{94} 2 + 2 * sigma\_X \symbol{94} 2)) * exp(-((x(6) - tau0) \symbol{94} 2 / (x(5) \symbol{94} 2 + sigma\_X \symbol{94} 2))) * h \symbol{94} 2 * (sigma\_X \symbol{94} 10) * sqrt(pi) * (tau0 \symbol{94} 2)+ 0.896e3 * sqrt((x(5) \symbol{94} 2 + 2 * sigma\_X \symbol{94} 2)) * exp(-((x(6) - tau0) \symbol{94} 2 / (x(5) \symbol{94} 2 + sigma\_X \symbol{94} 2))) * h \symbol{94} 2 * (sigma\_X \symbol{94} 10) * sqrt(pi) * (x(6) \symbol{94} 2) + 0.1440e4 * (sigma\_X \symbol{94} 11) * sqrt(pi) * sqrt((x(5) \symbol{94} 2 + sigma\_X \symbol{94} 2)) * sqrt((x(5) \symbol{94} 2 + 2 * sigma\_X \symbol{94}2)) - 0.384e3 * exp(-((x(6) - tau0) \symbol{94} 2 / (x(5) \symbol{94} 2 + 2 * sigma\_X \symbol{94} 2))) * h * (sigma\_X \symbol{94} 10) * sqrt(0.2e1) * sqrt(pi) * sqrt((x(5) \symbol{94} 2 + sigma\_X \symbol{94} 2)) * (tau0 \symbol{94} 2) * (x(6) \symbol{94} 2) * x(4) + 0.864e3 * (sigma\_X \symbol{94} 10) * sqrt((x(5) \symbol{94} 2 + sigma\_X \symbol{94} 2)) * x(3) * sqrt(pi) * exp(-((x(6) - tau0) \symbol{94} 2 / (x(5) \symbol{94} 2 + 2 * sigma\_X \symbol{94} 2))) * sqrt(0.2e1) * h * (tau0 \symbol{94} 2) * x(6) - 0.864e3 * (sigma\_X \symbol{94} 10) * sqrt((x(5) \symbol{94} 2 + sigma\_X \symbol{94} 2)) * x(3) * sqrt(pi) * exp(-((x(6) - tau0) \symbol{94} 2 / (x(5) \symbol{94} 2 + 2 * sigma\_X \symbol{94} 2))) * sqrt(0.2e1)* h * tau0 * (x(6) \symbol{94} 2) + 0.1792e4 * (sigma\_X \symbol{94} 12) * exp(-((x(6) - tau0) \symbol{94} 2 / (x(5) \symbol{94} 2 + 2 * sigma\_X \symbol{94} 2))) * h * sqrt(0.2e1) * sqrt(pi) * sqrt((x(5) \symbol{94} 2 + sigma\_X \symbol{94} 2)) * x(6) * x(4) * tau0 + 0.256e3 * exp(-((x(6) - tau0) \symbol{94} 2 / (x(5) \symbol{94} 2 + 2 * sigma\_X\symbol{94} 2))) * h * (sigma\_X \symbol{94} 10) * sqrt(0.2e1) * sqrt(pi) * sqrt((x(5) \symbol{94} 2 + sigma\_X \symbol{94} 2)) * (tau0 \symbol{94} 3) * x(6) * x(4) + 0.256e3 * exp(-((x(6) - tau0) \symbol{94} 2 / (x(5) \symbol{94} 2 + 2 * sigma\_X \symbol{94} 2))) * h * (sigma\_X \symbol{94} 10) * sqrt(0.2e1) * sqrt(pi) * sqrt((x(5) \symbol{94} 2 + sigma\_X \symbol{94}2)) * (x(6) \symbol{94} 3) * x(4) * tau0 - 0.224e3 * (sigma\_X \symbol{94} 12) * exp(-((x(6) - tau0) \symbol{94} 2 / (x(5) \symbol{94} 2 + sigma\_X \symbol{94} 2))) * h \symbol{94} 2 * sqrt(pi) * sqrt((x(5) \symbol{94} 2 + 2 * sigma\_X \symbol{94} 2)) - 0.1536e4 * (sigma\_X \symbol{94} 15) * (x(4) \symbol{94} 2) * sqrt(pi) * sqrt((x(5) \symbol{94} 2 + sigma\_X \symbol{94} 2)) *sqrt((x(5) \symbol{94} 2 + 2 * sigma\_X \symbol{94} 2)) + 0.1024e4 * exp(-((x(6) - tau0) \symbol{94} 2 / (x(5) \symbol{94} 2 + sigma\_X \symbol{94} 2))) * h \symbol{94} 2 * (sigma\_X \symbol{94} 8) * sqrt(pi) * sqrt((x(5) \symbol{94} 2 + 2 * sigma\_X \symbol{94} 2)) * (x(6) \symbol{94} 3) * tau0 + 0.1024e4 * exp(-((x(6) - tau0) \symbol{94} 2 / (x(5) \symbol{94} 2 + sigma\_X \symbol{94} 2))) * h \symbol{94} 2 * (sigma\_X \symbol{94} 8) * sqrt(pi) * sqrt((x(5) \symbol{94} 2 + 2 * sigma\_X \symbol{94} 2)) * (tau0 \symbol{94} 3) * x(6) - 0.1536e4 * exp(-((x(6) - tau0) \symbol{94} 2 / (x(5) \symbol{94} 2 + sigma\_X \symbol{94} 2))) * h \symbol{94} 2 * (sigma\_X \symbol{94} 8) * sqrt(pi) * sqrt((x(5) \symbol{94} 2 + 2 * sigma\_X \symbol{94} 2)) * (tau0 \symbol{94} 2) * (x(6) \symbol{94}2) + 0.1216e4 * (sigma\_X \symbol{94} 14) * exp(-((x(6) - tau0) \symbol{94} 2 / (x(5) \symbol{94} 2 + 2 * sigma\_X \symbol{94} 2))) * h * x(4) * sqrt(0.2e1) * sqrt(pi) * sqrt((x(5) \symbol{94} 2 + sigma\_X \symbol{94} 2)) - 0.1792e4 * sqrt((x(5) \symbol{94} 2 + 2 * sigma\_X \symbol{94} 2)) * exp(-((x(6) - tau0) \symbol{94} 2 / (x(5) \symbol{94} 2 + sigma\_X\symbol{94} 2))) * h \symbol{94} 2 * (sigma\_X \symbol{94} 10) * sqrt(pi) * tau0 * x(6)) * pi \symbol{94} (-0.1e1 / 0.2e1) * ((x(5) \symbol{94} 2 + sigma\_X \symbol{94} 2) \symbol{94} (-0.9e1 / 0.2e1)) / sigma\_X * ((x(5) \symbol{94} 2 + 2 * sigma\_X \symbol{94} 2) \symbol{94} (-0.9e1 / 0.2e1)) * (x(5) \symbol{94} 2) / 0.4e1 + (-0.192e3 * (sigma\_X \symbol{94} 12) * sqrt((x(5) \symbol{94}2 + sigma\_X \symbol{94} 2)) * x(3) * sqrt(pi) * exp(-((x(6) - tau0) \symbol{94} 2 / (x(5) \symbol{94} 2 + 2 * sigma\_X \symbol{94} 2))) * sqrt(0.2e1) * h * tau0 * (x(6) \symbol{94} 2) - 0.256e3 * (sigma\_X \symbol{94} 17) * (x(4) \symbol{94} 2) * sqrt(pi) * sqrt((x(5) \symbol{94} 2 + sigma\_X \symbol{94} 2)) * sqrt((x(5) \symbol{94} 2 + 2 * sigma\_X \symbol{94} 2)) +0.320e3 * (sigma\_X \symbol{94} 12) * exp(-((x(6) - tau0) \symbol{94} 2 / (x(5) \symbol{94} 2 + sigma\_X \symbol{94} 2))) * h \symbol{94} 2 * sqrt(pi) * sqrt((x(5) \symbol{94} 2 + 2 * sigma\_X \symbol{94} 2)) * (x(6) \symbol{94} 2) + 0.192e3 * (sigma\_X \symbol{94} 14) * exp(-((x(6) - tau0) \symbol{94} 2 / (x(5) \symbol{94} 2 + 2 * sigma\_X \symbol{94} 2))) * h * sqrt(0.2e1) *sqrt(pi) * sqrt((x(5) \symbol{94} 2 + sigma\_X \symbol{94} 2)) * tau0 * x(3) - 0.64e2 * (sigma\_X \symbol{94} 14) * exp(-((x(6) - tau0) \symbol{94} 2 / (x(5) \symbol{94} 2 + sigma\_X \symbol{94} 2))) * h \symbol{94} 2 * sqrt(pi) * sqrt((x(5) \symbol{94} 2 + 2 * sigma\_X \symbol{94} 2)) - 0.64e2 * (sigma\_X \symbol{94} 12) * sqrt((x(5) \symbol{94} 2 + sigma\_X \symbol{94} 2)) * x(3) * sqrt(pi) * exp(-((x(6) - tau0) \symbol{94} 2 / (x(5) \symbol{94} 2 + 2 * sigma\_X \symbol{94} 2))) * sqrt(0.2e1) * h * (tau0 \symbol{94} 3) + 0.256e3 * exp(-((x(6) - tau0) \symbol{94} 2 / (x(5) \symbol{94} 2 + 2 * sigma\_X \symbol{94} 2))) * h * (sigma\_X \symbol{94} 16) * x(4) * sqrt(0.2e1) * sqrt(pi) * sqrt((x(5) \symbol{94} 2 + sigma\_X \symbol{94} 2)) - 0.128e3 * exp(-((x(6) - tau0) \symbol{94} 2 / (x(5) \symbol{94} 2 + sigma\_X \symbol{94} 2))) * h \symbol{94} 2 * (sigma\_X \symbol{94} 10) * sqrt(pi) * sqrt((x(5) \symbol{94} 2 + 2 * sigma\_X \symbol{94} 2)) * (tau0 \symbol{94} 4) - 0.256e3 * exp(-((x(6) - tau0) \symbol{94} 2 / (x(5) \symbol{94} 2 + 2 * sigma\_X \symbol{94} 2))) * h * (sigma\_X \symbol{94} 14) * sqrt(0.2e1) * sqrt(pi) * sqrt((x(5) \symbol{94} 2 + sigma\_X \symbol{94} 2)) * x(4) * (x(6) \symbol{94} 2) - 0.256e3 * exp(-((x(6) - tau0) \symbol{94} 2 / (x(5) \symbol{94} 2 + 2 * sigma\_X \symbol{94} 2))) * h * (sigma\_X \symbol{94} 14) * sqrt(0.2e1) * sqrt(pi) * sqrt((x(5) \symbol{94} 2 + sigma\_X \symbol{94} 2)) * x(4) * (tau0 \symbol{94} 2) + 0.320e3 * sqrt((x(5)\symbol{94} 2 + 2 * sigma\_X \symbol{94} 2)) * exp(-((x(6) - tau0) \symbol{94} 2 / (x(5) \symbol{94} 2 + sigma\_X \symbol{94} 2))) * h \symbol{94} 2 * (sigma\_X \symbol{94} 12) * sqrt(pi) * (tau0 \symbol{94} 2) + 0.192e3 * (sigma\_X \symbol{94} 12) * sqrt((x(5) \symbol{94} 2 + sigma\_X \symbol{94} 2)) * x(3) * sqrt(pi) * exp(-((x(6) - tau0) \symbol{94} 2 / (x(5) \symbol{94} 2 + 2 * sigma\_X \symbol{94} 2))) * sqrt(0.2e1) * h * (tau0 \symbol{94} 2) * x(6) + 0.512e3 * exp(-((x(6) - tau0) \symbol{94} 2 / (x(5) \symbol{94} 2 + sigma\_X \symbol{94} 2))) * h \symbol{94} 2 * (sigma\_X \symbol{94} 10) * sqrt(pi) * sqrt((x(5) \symbol{94} 2 + 2 * sigma\_X \symbol{94} 2)) * (x(6) \symbol{94} 3) * tau0 - 0.768e3 * exp(-((x(6) - tau0) \symbol{94} 2 / (x(5) \symbol{94} 2 +sigma\_X \symbol{94} 2))) * h \symbol{94} 2 * (sigma\_X \symbol{94} 10) * sqrt(pi) * sqrt((x(5) \symbol{94} 2 + 2 * sigma\_X \symbol{94} 2)) * (tau0 \symbol{94} 2) * (x(6) \symbol{94} 2) + 0.512e3 * exp(-((x(6) - tau0) \symbol{94} 2 / (x(5) \symbol{94} 2 + sigma\_X \symbol{94} 2))) * h \symbol{94} 2 * (sigma\_X \symbol{94} 10) * sqrt(pi) * sqrt((x(5) \symbol{94} 2 + 2 * sigma\_X \symbol{94} 2)) * (tau0 \symbol{94} 3) * x(6) + 0.512e3 * (sigma\_X \symbol{94} 14) * exp(-((x(6) - tau0) \symbol{94} 2 / (x(5) \symbol{94} 2 + 2 * sigma\_X \symbol{94} 2))) * h * sqrt(0.2e1) * sqrt(pi) * sqrt((x(5) \symbol{94} 2 + sigma\_X \symbol{94} 2)) * x(6) * x(4) * tau0 - 0.192e3 * (sigma\_X \symbol{94} 14) * exp(-((x(6) - tau0) \symbol{94} 2 / (x(5) \symbol{94} 2 + 2 *sigma\_X \symbol{94} 2))) * h * sqrt(0.2e1) * sqrt(pi) * sqrt((x(5) \symbol{94} 2 + sigma\_X \symbol{94} 2)) * x(6) * x(3) + 0.64e2 * (sigma\_X \symbol{94} 12) * sqrt((x(5) \symbol{94} 2 + sigma\_X \symbol{94} 2)) * x(3) * sqrt(pi) * exp(-((x(6) - tau0) \symbol{94} 2 / (x(5) \symbol{94} 2 + 2 * sigma\_X \symbol{94} 2))) * sqrt(0.2e1) * h * (x(6) \symbol{94} 3) + 0.992e3 * (sigma\_X \symbol{94} 13) * sqrt(pi) * sqrt((x(5) \symbol{94} 2 + sigma\_X \symbol{94} 2)) * sqrt((x(5) \symbol{94} 2 + 2 * sigma\_X \symbol{94} 2)) - 0.640e3 * sqrt((x(5) \symbol{94} 2 + 2 * sigma\_X \symbol{94} 2)) * exp(-((x(6) - tau0) \symbol{94} 2 / (x(5) \symbol{94} 2 + sigma\_X \symbol{94} 2))) * h \symbol{94} 2 * (sigma\_X \symbol{94} 12) * sqrt(pi) * tau0 *x(6) - 0.128e3 * exp(-((x(6) - tau0) \symbol{94} 2 / (x(5) \symbol{94} 2 + sigma\_X \symbol{94} 2))) * h \symbol{94} 2 * (sigma\_X \symbol{94} 10) * sqrt(pi) * sqrt((x(5) \symbol{94} 2 + 2 * sigma\_X \symbol{94} 2)) * (x(6) \symbol{94} 4)) * pi \symbol{94} (-0.1e1 / 0.2e1) * ((x(5) \symbol{94} 2 + sigma\_X \symbol{94} 2) \symbol{94} (-0.9e1 / 0.2e1)) / sigma\_X * ((x(5) \symbol{94} 2 + 2* sigma\_X \symbol{94} 2) \symbol{94} (-0.9e1 / 0.2e1)) / 0.4e1 + (96 * sigma\_X \symbol{94} 14 / (x(5) \symbol{94} 2 + sigma\_X \symbol{94} 2) \symbol{94} 4 / (x(5) \symbol{94} 2 + 2 * sigma\_X \symbol{94} 2) \symbol{94} 4 / x(5) \symbol{94} 2) + (16 / (x(5) \symbol{94} 2 + sigma\_X \symbol{94} 2) \symbol{94} 4 / (x(5) \symbol{94} 2 + 2 * sigma\_X \symbol{94} 2) \symbol{94} 4 * sigma\_X \symbol{94} 16 / x(5) \symbol{94} 4) + (Ia / ((x(5)\symbol{94} 2 + sigma\_X \symbol{94} 2) \symbol{94} 4) / ((x(5) \symbol{94} 2 + 2 * sigma\_X \symbol{94} 2) \symbol{94} 4) * pi \symbol{94} (-0.1e1 / 0.2e1) * (x(5) \symbol{94} 13) / 0.2e1 + 0.6e1 * Ia / ((x(5) \symbol{94} 2 + sigma\_X \symbol{94} 2) \symbol{94} 4) * (sigma\_X \symbol{94} 2) / ((x(5) \symbol{94} 2 + 2 * sigma\_X \symbol{94} 2) \symbol{94} 4) * pi \symbol{94} (-0.1e1 / 0.2e1) * (x(5) \symbol{94} 11) + 0.31e2 *Ia / ((x(5) \symbol{94} 2 + sigma\_X \symbol{94} 2) \symbol{94} 4) * (sigma\_X \symbol{94} 4) / ((x(5) \symbol{94} 2 + 2 * sigma\_X \symbol{94} 2) \symbol{94} 4) * pi \symbol{94} (-0.1e1 / 0.2e1) * (x(5) \symbol{94} 9) + 0.90e2 * (sigma\_X \symbol{94} 6) * Ia / ((x(5) \symbol{94} 2 + sigma\_X \symbol{94} 2) \symbol{94} 4) / ((x(5) \symbol{94} 2 + 2 * sigma\_X \symbol{94} 2) \symbol{94} 4) * pi \symbol{94} (-0.1e1 / 0.2e1) * (x(5) \symbol{94} 7) + 0.321e3 / 0.2e1 * (sigma\_X \symbol{94} 8) * Ia / ((x(5) \symbol{94} 2 + sigma\_X \symbol{94} 2) \symbol{94} 4) / ((x(5) \symbol{94} 2 + 2 * sigma\_X \symbol{94} 2) \symbol{94} 4) * pi \symbol{94} (-0.1e1 / 0.2e1) * (x(5) \symbol{94} 5) + 0.180e3 * (sigma\_X \symbol{94} 10) * Ia / ((x(5) \symbol{94} 2 + sigma\_X \symbol{94} 2) \symbol{94} 4) / ((x(5) \symbol{94} 2 + 2 * sigma\_X \symbol{94} 2) \symbol{94} 4) *pi \symbol{94} (-0.1e1 / 0.2e1) * (x(5) \symbol{94} 3) + 0.124e3 * (sigma\_X \symbol{94} 12) * Ia / ((x(5) \symbol{94} 2 + sigma\_X \symbol{94} 2) \symbol{94} 4) / ((x(5) \symbol{94} 2 + 2 * sigma\_X \symbol{94} 2) \symbol{94} 4) * pi \symbol{94} (-0.1e1 / 0.2e1) * x(5) + 0.48e2 * (sigma\_X \symbol{94} 14) * Ia / ((x(5) \symbol{94} 2 + sigma\_X \symbol{94} 2) \symbol{94} 4) / ((x(5) \symbol{94} 2 + 2 * sigma\_X \symbol{94} 2) \symbol{94} 4) * pi \symbol{94} (-0.1e1 / 0.2e1) / x(5) + 0.8e1 * (sigma\_X \symbol{94} 16) * Ia / ((x(5) \symbol{94} 2 + sigma\_X \symbol{94} 2) \symbol{94} 4) / ((x(5) \symbol{94} 2 + 2 * sigma\_X \symbol{94} 2) \symbol{94} 4) * pi \symbol{94} (-0.1e1 / 0.2e1) / (x(5) \symbol{94} 3)) / x(1) + (-Is / ((x(5) \symbol{94} 2 + sigma\_X \symbol{94} 2) \symbol{94} 4) / ((x(5) \symbol{94} 2 + 2 * sigma\_X\symbol{94} 2) \symbol{94} 4) * pi \symbol{94} (-0.1e1 / 0.2e1) * (x(5) \symbol{94} 14) - 0.12e2 * Is / ((x(5) \symbol{94} 2 + sigma\_X \symbol{94} 2) \symbol{94} 4) * (sigma\_X \symbol{94} 2) / ((x(5) \symbol{94} 2 + 2 * sigma\_X \symbol{94} 2) \symbol{94} 4) * pi \symbol{94} (-0.1e1 / 0.2e1) * (x(5) \symbol{94} 12) - 0.62e2 * Is / ((x(5) \symbol{94} 2 + sigma\_X \symbol{94} 2) \symbol{94} 4) * (sigma\_X \symbol{94} 4) / ((x(5) \symbol{94} 2 + 2 * sigma\_X \symbol{94} 2) \symbol{94} 4) * pi \symbol{94} (-0.1e1 / 0.2e1) * (x(5) \symbol{94} 10) - 0.180e3 * Is / ((x(5) \symbol{94} 2 + sigma\_X \symbol{94} 2) \symbol{94} 4) * (sigma\_X \symbol{94} 6) / ((x(5) \symbol{94} 2 + 2 * sigma\_X \symbol{94} 2) \symbol{94} 4) * pi \symbol{94} (-0.1e1 / 0.2e1) * (x(5) \symbol{94} 8) - 0.321e3 * Is / ((x(5) \symbol{94} 2 + sigma\_X \symbol{94} 2) \symbol{94} 4) *(sigma\_X \symbol{94} 8) / ((x(5) \symbol{94} 2 + 2 * sigma\_X \symbol{94} 2) \symbol{94} 4) * pi \symbol{94} (-0.1e1 / 0.2e1) * (x(5) \symbol{94} 6) - 0.360e3 * Is / ((x(5) \symbol{94} 2 + sigma\_X \symbol{94} 2) \symbol{94} 4) * (sigma\_X \symbol{94} 10) / ((x(5) \symbol{94} 2 + 2 * sigma\_X \symbol{94} 2) \symbol{94} 4) * pi \symbol{94} (-0.1e1 / 0.2e1) * (x(5) \symbol{94} 4) - 0.248e3 * Is / ((x(5) \symbol{94} 2+ sigma\_X \symbol{94} 2) \symbol{94} 4) * (sigma\_X \symbol{94} 12) / ((x(5) \symbol{94} 2 + 2 * sigma\_X \symbol{94} 2) \symbol{94} 4) * pi \symbol{94} (-0.1e1 / 0.2e1) * (x(5) \symbol{94} 2) - 0.96e2 * Is / ((x(5) \symbol{94} 2 + sigma\_X \symbol{94} 2) \symbol{94} 4) * (sigma\_X \symbol{94} 14) / ((x(5) \symbol{94} 2 + 2 * sigma\_X \symbol{94} 2) \symbol{94} 4) * pi \symbol{94} (-0.1e1 / 0.2e1) - 0.16e2 * Is / ((x(5) \symbol{94} 2 + sigma\_X \symbol{94} 2) \symbol{94} 4) * (sigma\_X \symbol{94} 16) / ((x(5) \symbol{94} 2 + 2 * sigma\_X \symbol{94} 2) \symbol{94} 4) * pi \symbol{94} (-0.1e1 / 0.2e1) / (x(5) \symbol{94} 2)) / x(1) \symbol{94} 2;
\underline{}Warning, the function names \{x\} are not recognized in the target language\underline{}\mapleresult
cg3 = 0.0e0 == -(-0.192e3 * x(1) \symbol{94} 2 * x(4) * (x(5) \symbol{94} 7) * sqrt(pi) * sqrt((x(5) \symbol{94} 2 + 2 * sigma\_X \symbol{94} 2)) * (sigma\_X \symbol{94} 4) - 0.256e3 * x(1) \symbol{94} 2 * x(4) * (x(5) \symbol{94} 5) * sqrt(pi) * sqrt((x(5) \symbol{94} 2 + 2 * sigma\_X \symbol{94} 2)) * (sigma\_X \symbol{94} 6) - 0.128e3 * x(1) \symbol{94} 2 * x(4) *(x(5) \symbol{94} 3) * sqrt(pi) * sqrt((x(5) \symbol{94} 2 + 2 * sigma\_X \symbol{94} 2)) * (sigma\_X \symbol{94} 8) - 0.64e2 * x(1) \symbol{94} 2 * x(4) * (x(5) \symbol{94} 9) * sqrt(pi) * sqrt((x(5) \symbol{94} 2 + 2 * sigma\_X \symbol{94} 2)) * (sigma\_X \symbol{94} 2) + 0.128e3 * exp(-((x(6) - tau0) \symbol{94} 2 / (x(5) \symbol{94} 2 + 2 * sigma\_X \symbol{94} 2))) * h * sigma\_X * (x(5) \symbol{94} 7) * x(1) \symbol{94} 2 * sqrt(0.2e1) * sqrt(pi) * tau0 * x(6) + 0.320e3 * exp(-((x(6) - tau0) \symbol{94} 2 / (x(5) \symbol{94} 2 + 2 * sigma\_X \symbol{94} 2))) * h * (sigma\_X \symbol{94} 3) * (x(5) \symbol{94} 5) * x(1) \symbol{94} 2 * sqrt(0.2e1) * sqrt(pi) * tau0 * x(6) + 0.128e3 * exp(-((x(6) - tau0) \symbol{94} 2/ (x(5) \symbol{94} 2 + 2 * sigma\_X \symbol{94} 2))) * h * (sigma\_X \symbol{94} 5) * (x(5) \symbol{94} 3) * x(1) \symbol{94} 2 * sqrt(0.2e1) * sqrt(pi) * tau0 * x(6) - 0.64e2 * sqrt(pi) * (x(5) \symbol{94} 5) * x(1) \symbol{94} 2 * exp(-((x(6) - tau0) \symbol{94} 2 / (x(5) \symbol{94} 2 + 2 * sigma\_X \symbol{94} 2))) * sqrt(0.2e1) * h * tau0 * (x(6) \symbol{94} 3) * sigma\_X - 0.64e2 * sqrt(pi) * (x(5) \symbol{94} 5) * x(1) \symbol{94} 2 * exp(-((x(6) - tau0) \symbol{94} 2 / (x(5) \symbol{94} 2 + 2 * sigma\_X \symbol{94} 2))) * sqrt(0.2e1) * h * (tau0 \symbol{94} 3) * x(6) * sigma\_X + 0.96e2 * sqrt(pi) * (x(5) \symbol{94} 5) * x(1) \symbol{94} 2 * exp(-((x(6) - tau0) \symbol{94} 2 / (x(5) \symbol{94} 2 + 2 * sigma\_X \symbol{94} 2))) * sqrt(0.2e1) * h * (tau0 \symbol{94} 2) * (x(6) \symbol{94} 2) * sigma\_X + 0.20e2 * sigma\_X * (x(5) \symbol{94} 9) * sqrt(pi) * exp(-((x(6) - tau0) \symbol{94} 2 / (x(5) \symbol{94} 2 + 2 * sigma\_X \symbol{94} 2))) * sqrt(0.2e1) * h * x(1) \symbol{94} 2 + 0.96e2 * (sigma\_X \symbol{94} 3) * (x(5) \symbol{94} 7) * sqrt(pi) * exp(-((x(6) - tau0) \symbol{94} 2 / (x(5) \symbol{94} 2 + 2 * sigma\_X \symbol{94} 2))) * sqrt(0.2e1) * h * x(1) \symbol{94} 2 + 0.144e3 * (sigma\_X \symbol{94} 5) * (x(5) \symbol{94} 5) * sqrt(pi) * exp(-((x(6) - tau0) \symbol{94} 2 / (x(5) \symbol{94} 2 + 2 * sigma\_X \symbol{94} 2))) * sqrt(0.2e1) * h * x(1) \symbol{94} 2 + 0.64e2 * (sigma\_X \symbol{94} 7) * (x(5) \symbol{94} 3) * sqrt(pi) * exp(-((x(6) - tau0) \symbol{94} 2 / (x(5) \symbol{94} 2 + 2 * sigma\_X \symbol{94} 2))) * sqrt(0.2e1) * h * x(1) \symbol{94} 2 - 0.2e1 * Id * sqrt((x(5) \symbol{94} 2 + 2 * sigma\_X \symbol{94} 2)) * (x(5) \symbol{94} 8) - 0.32e2 * Id * sqrt((x(5) \symbol{94} 2 + 2 * sigma\_X \symbol{94} 2)) * (sigma\_X \symbol{94} 8) + (x(5) \symbol{94} 10) * Ib * sqrt((x(5)\symbol{94} 2 + 2 * sigma\_X \symbol{94} 2)) - 0.64e2 * Id * sqrt((x(5) \symbol{94} 2 + 2 * sigma\_X \symbol{94} 2)) * (x(5) \symbol{94} 2) * (sigma\_X \symbol{94} 6) + 0.32e2 * (x(5) \symbol{94} 4) * Ib * sqrt((x(5) \symbol{94} 2 + 2 * sigma\_X \symbol{94} 2)) * (sigma\_X \symbol{94} 6) + 0.16e2 * (x(5) \symbol{94} 2) * Ib * sqrt((x(5) \symbol{94} 2 + 2 * sigma\_X \symbol{94} 2)) * (sigma\_X \symbol{94} 8) + 0.8e1 * (x(5) \symbol{94} 8) * Ib * sqrt((x(5) \symbol{94} 2 + 2 * sigma\_X \symbol{94} 2)) * (sigma\_X \symbol{94} 2) + 0.24e2 * (x(5) \symbol{94} 6) * Ib * sqrt((x(5) \symbol{94} 2 + 2 * sigma\_X \symbol{94} 2)) * (sigma\_X \symbol{94} 4) - 0.16e2 * Id * sqrt((x(5) \symbol{94} 2 + 2 * sigma\_X \symbol{94} 2)) * (x(5) \symbol{94} 6) * (sigma\_X \symbol{94} 2) - 0.48e2* Id * sqrt((x(5) \symbol{94} 2 + 2 * sigma\_X \symbol{94} 2)) * (x(5) \symbol{94} 4) * (sigma\_X \symbol{94} 4) - 0.8e1 * x(1) \symbol{94} 2 * x(4) * (x(5) \symbol{94} 11) * sqrt(pi) * sqrt((x(5) \symbol{94} 2 + 2 * sigma\_X \symbol{94} 2)) - 0.64e2 * exp(-((x(6) - tau0) \symbol{94} 2 / (x(5) \symbol{94} 2 + 2 * sigma\_X \symbol{94} 2))) * h * sigma\_X * (x(5) \symbol{94} 7) *x(1) \symbol{94} 2 * sqrt(0.2e1) * sqrt(pi) * (tau0 \symbol{94} 2) + 0.16e2 * sqrt(pi) * (x(5) \symbol{94} 5) * x(1) \symbol{94} 2 * exp(-((x(6) - tau0) \symbol{94} 2 / (x(5) \symbol{94} 2 + 2 * sigma\_X \symbol{94} 2))) * sqrt(0.2e1) * h * (x(6) \symbol{94} 4) * sigma\_X + 0.16e2 * sqrt(pi) * (x(5) \symbol{94} 5) * x(1) \symbol{94} 2 * exp(-((x(6) - tau0) \symbol{94} 2 / (x(5) \symbol{94} 2 + 2 * sigma\_X \symbol{94} 2))) * sqrt(0.2e1) * h * (tau0 \symbol{94} 4) * sigma\_X - 0.160e3 * exp(-((x(6) - tau0) \symbol{94} 2 / (x(5) \symbol{94} 2 + 2 * sigma\_X \symbol{94} 2))) * h * (sigma\_X \symbol{94} 3) * (x(5) \symbol{94} 5) * x(1) \symbol{94} 2 * sqrt(0.2e1) * sqrt(pi) * (tau0 \symbol{94} 2) - 0.160e3 * exp(-((x(6) -tau0) \symbol{94} 2 / (x(5) \symbol{94} 2 + 2 * sigma\_X \symbol{94} 2))) * h * (sigma\_X \symbol{94} 3) * (x(5) \symbol{94} 5) * x(1) \symbol{94} 2 * sqrt(0.2e1) * sqrt(pi) * (x(6) \symbol{94} 2) - 0.64e2 * exp(-((x(6) - tau0) \symbol{94} 2 / (x(5) \symbol{94} 2 + 2 * sigma\_X \symbol{94} 2))) * h * (sigma\_X \symbol{94} 5) * (x(5) \symbol{94} 3) * x(1) \symbol{94} 2 * sqrt(0.2e1) * sqrt(pi) * (tau0 \symbol{94} 2) - 0.64e2 * exp(-((x(6) - tau0) \symbol{94} 2 / (x(5) \symbol{94} 2 + 2 * sigma\_X \symbol{94} 2))) * h * sigma\_X * (x(5) \symbol{94} 7) * x(1) \symbol{94} 2 * sqrt(0.2e1) * sqrt(pi) * (x(6) \symbol{94} 2) - 0.64e2 * exp(-((x(6) - tau0) \symbol{94} 2 / (x(5) \symbol{94} 2 + 2 * sigma\_X \symbol{94} 2))) * h * (sigma\_X \symbol{94} 5) * (x(5) \symbol{94} 3) * x(1) \symbol{94} 2 * sqrt(0.2e1) * sqrt(pi) * (x(6) \symbol{94} 2)) / x(1) \symbol{94} 2 / (x(5) \symbol{94} 2) * pi \symbol{94} (-0.1e1 / 0.2e1) * ((x(5) \symbol{94} 2 + 2 * sigma\_X \symbol{94} 2) \symbol{94} (-0.9e1 / 0.2e1)) / 0.2e1;
\underline{}Warning, the function names \{x\} are not recognized in the target language\underline{}\mapleresult
cg4 = 0.0e0 == (0.24e2 * sigma\_X * (x(5) \symbol{94} 3) * sqrt(pi) * exp(-((x(6) - tau0) \symbol{94} 2 / (x(5) \symbol{94} 2 + 2 * sigma\_X \symbol{94} 2))) * sqrt(0.2e1) * h * x(1) \symbol{94} 2 * (tau0 \symbol{94} 2) * x(6) + 0.2e1 * x(1) \symbol{94} 2 * x(3) * (x(5) \symbol{94} 7) * sqrt(pi) * sqrt((x(5) \symbol{94} 2 + 2 * sigma\_X \symbol{94} 2)) + 0.12e2 * x(1) \symbol{94} 2 * x(3) * (x(5) \symbol{94} 5) * sqrt(pi) * sqrt((x(5) \symbol{94} 2 + 2 * sigma\_X \symbol{94} 2)) * (sigma\_X \symbol{94} 2) + 0.24e2 * x(1) \symbol{94} 2 * x(3) * (x(5) \symbol{94} 3) * sqrt(pi) * sqrt((x(5) \symbol{94} 2 + 2 * sigma\_X \symbol{94} 2)) * (sigma\_X \symbol{94} 4) + 0.16e2 * x(1) \symbol{94} 2 * x(3) * x(5) * sqrt(pi) * sqrt((x(5) \symbol{94} 2 + 2 * sigma\_X \symbol{94} 2)) * (sigma\_X \symbol{94} 6) - 0.16e2 * sigma\_X * (x(5) \symbol{94} 5) * sqrt(pi) * exp(-((x(6) - tau0) \symbol{94} 2 / (x(5) \symbol{94} 2 + 2 * sigma\_X \symbol{94} 2))) * sqrt(0.2e1) * h * x(1) \symbol{94} 2 * x(6) - 0.40e2 * (sigma\_X \symbol{94} 3) * (x(5) \symbol{94} 3) * sqrt(pi) * exp(-((x(6) - tau0) \symbol{94}2 / (x(5) \symbol{94} 2 + 2 * sigma\_X \symbol{94} 2))) * sqrt(0.2e1) * h * x(1) \symbol{94} 2 * x(6) - 0.16e2 * exp(-((x(6) - tau0) \symbol{94} 2 / (x(5) \symbol{94} 2 + 2 * sigma\_X \symbol{94} 2))) * h * (sigma\_X \symbol{94} 5) * x(5) * x(1) \symbol{94} 2 * sqrt(0.2e1) * sqrt(pi) * x(6) + 0.16e2 * sigma\_X * (x(5) \symbol{94} 5) * sqrt(pi) * exp(-((x(6) - tau0) \symbol{94} 2 / (x(5) \symbol{94} 2 + 2 * sigma\_X \symbol{94} 2))) * sqrt(0.2e1) * h * x(1) \symbol{94} 2 * tau0 + 0.40e2 * (sigma\_X \symbol{94} 3) * (x(5) \symbol{94} 3) * sqrt(pi) * exp(-((x(6) - tau0) \symbol{94} 2 / (x(5) \symbol{94} 2 + 2 * sigma\_X \symbol{94} 2))) * sqrt(0.2e1) * h * x(1) \symbol{94} 2 * tau0 + 0.16e2 * exp(-((x(6) - tau0) \symbol{94} 2 / (x(5) \symbol{94} 2 + 2 * sigma\_X \symbol{94} 2))) * h * (sigma\_X \symbol{94} 5) * x(5) * x(1) \symbol{94} 2 * sqrt(0.2e1) * sqrt(pi) * tau0 - 0.24e2 * sigma\_X * (x(5) \symbol{94} 3) * sqrt(pi) * exp(-((x(6) - tau0) \symbol{94} 2 / (x(5) \symbol{94} 2 + 2 * sigma\_X \symbol{94} 2))) * sqrt(0.2e1) * h * x(1) \symbol{94} 2 * tau0* (x(6) \symbol{94} 2) + 0.8e1 * sigma\_X * (x(5) \symbol{94} 3) * sqrt(pi) * exp(-((x(6) - tau0) \symbol{94} 2 / (x(5) \symbol{94} 2 + 2 * sigma\_X \symbol{94} 2))) * sqrt(0.2e1) * h * x(1) \symbol{94} 2 * (x(6) \symbol{94} 3) - 0.8e1 * sigma\_X * (x(5) \symbol{94} 3) * sqrt(pi) * exp(-((x(6) - tau0) \symbol{94} 2 / (x(5) \symbol{94} 2 + 2 * sigma\_X \symbol{94} 2)))* sqrt(0.2e1) * h * x(1) \symbol{94} 2 * (tau0 \symbol{94} 3) + Ic * sqrt((x(5) \symbol{94} 2 + 2 * sigma\_X \symbol{94} 2)) * (x(5) \symbol{94} 6) + 0.6e1 * Ic * sqrt((x(5) \symbol{94} 2 + 2 * sigma\_X \symbol{94} 2)) * (x(5) \symbol{94} 4) * (sigma\_X \symbol{94} 2) + 0.12e2 * Ic * sqrt((x(5) \symbol{94} 2 + 2 * sigma\_X \symbol{94} 2)) * (x(5) \symbol{94} 2) * (sigma\_X \symbol{94} 4)+ 0.8e1 * Ic * sqrt((x(5) \symbol{94} 2 + 2 * sigma\_X \symbol{94} 2)) * (sigma\_X \symbol{94} 6)) / x(1) \symbol{94} 2 / x(5) * pi \symbol{94} (-0.1e1 / 0.2e1) * ((x(5) \symbol{94} 2 + 2 * sigma\_X \symbol{94} 2) \symbol{94} (-0.7e1 / 0.2e1));
\end{maplegroup}
\begin{maplegroup}
\begin{mapleinput}
\mapleinline{active}{2d}{Matlab(F1); -1; Matlab(F2); -1; Matlab(F3); -1; Matlab(F4); -1; Matlab(F5); -1; Matlab(F6); -1}{\[\]}
\end{mapleinput}
\mapleresult
\underline{}Warning, the function names \{imag, real, x\} are not recognized in the target language\underline{}\mapleresult
cg5 = (h \symbol{94} 2 * (t - tau0) \symbol{94} 2 / sigma\_X \symbol{94} 4 * exp(-(t - tau0) \symbol{94} 2 / sigma\_X \symbol{94} 2 / 0.2e1) \symbol{94} 2 + h / sigma\_X \symbol{94} 2 * exp(-(t - tau0) \symbol{94} 2 / sigma\_X \symbol{94} 2 / 0.2e1) - h * (t - tau0) \symbol{94} 2 / sigma\_X \symbol{94} 4 * exp(-(t - tau0) \symbol{94} 2 / sigma\_X \symbol{94} 2 / 0.2e1) - 0.3e1 * x(1) \symbol{94} 2 *exp(-(t - x(6)) \symbol{94} 2 / x(5) \symbol{94} 2 / 0.2e1) \symbol{94} 2) * exp(-(t - x(6)) \symbol{94} 2 / x(5) \symbol{94} 2 / 0.2e1) * (0.2e1 * real(uS) * cos(x(2) + x(3) * (t - x(6)) + x(4) * (t - x(6)) \symbol{94} 2) + 0.2e1 * imag(uS) * sin(x(2) + x(3) * (t - x(6)) + x(4) * (t - x(6)) \symbol{94} 2)) - x(1) * exp(-(t- x(6)) \symbol{94} 2 / x(5) \symbol{94} 2 / 0.2e1) \symbol{94} 2 * (0.2e1 * real(uS) \symbol{94} 2 * cos(0.2e1 * x(2) + 0.2e1 * x(3) * (t - x(6)) + 0.2e1 * x(4) * (t - x(6)) \symbol{94} 2) - 0.2e1 * imag(uS) \symbol{94} 2 * cos(0.2e1 * x(2) + 0.2e1 * x(3) * (t - x(6)) + 0.2e1 * x(4) * (t - x(6)) \symbol{94} 2) + 0.4e1 * imag(uS) * real(uS) * sin(0.2e1 * x(2) + 0.2e1 * x(3) * (t - x(6)) + 0.2e1 * x(4) * (t - x(6)) \symbol{94} 2));
\underline{}Warning, the function names \{imag, real, x\} are not recognized in the target language\underline{}\mapleresult
cg6 = (h \symbol{94} 2 * (t - tau0) \symbol{94} 2 / sigma\_X \symbol{94} 4 * exp(-(t - tau0) \symbol{94} 2 / sigma\_X \symbol{94} 2 / 0.2e1) \symbol{94} 2 + h / sigma\_X \symbol{94} 2 * exp(-(t - tau0) \symbol{94} 2 / sigma\_X \symbol{94} 2 / 0.2e1) - h * (t - tau0) \symbol{94} 2 / sigma\_X \symbol{94} 4 * exp(-(t - tau0) \symbol{94} 2 / sigma\_X \symbol{94} 2 / 0.2e1) - x(1) \symbol{94} 2 * exp(-(t- x(6)) \symbol{94} 2 / x(5) \symbol{94} 2 / 0.2e1) \symbol{94} 2) * x(1) * exp(-(t - x(6)) \symbol{94} 2 / x(5) \symbol{94} 2 / 0.2e1) * (0.2e1 * real(uS) * cos(x(2) + x(3) * (t - x(6)) + x(4) * (t - x(6)) \symbol{94} 2) - 0.2e1 * imag(uS) * sin(x(2) + x(3) * (t - x(6)) + x(4) * (t - x(6)) \symbol{94} 2)) - x(1) \symbol{94} 2 * exp(-(t - x(6)) \symbol{94} 2 / x(5) \symbol{94} 2 / 0.2e1) \symbol{94} 2 * (0.2e1 * imag(uS) \symbol{94} 2 * sin(0.2e1 * x(2) + 0.2e1 * x(3) * (t - x(6)) + 0.2e1 * x(4) * (t - x(6)) \symbol{94} 2) - 0.2e1 * real(uS) \symbol{94} 2 * sin(0.2e1 * x(2) + 0.2e1 * x(3) * (t - x(6)) + 0.2e1 * x(4) * (t - x(6)) \symbol{94} 2) + 0.4e1 *imag(uS) * real(uS) * cos(0.2e1 * x(2) + 0.2e1 * x(3) * (t - x(6)) + 0.2e1 * x(4) * (t - x(6)) \symbol{94} 2));
\underline{}Warning, the function names \{imag, real, x\} are not recognized in the target language\underline{}\mapleresult
cg7 = (h \symbol{94} 2 * (t - tau0) \symbol{94} 2 / sigma\_X \symbol{94} 4 * exp(-(t - tau0) \symbol{94} 2 / sigma\_X \symbol{94} 2 / 0.2e1) \symbol{94} 2 + h / sigma\_X \symbol{94} 2 * exp(-(t - tau0) \symbol{94} 2 / sigma\_X \symbol{94} 2 / 0.2e1) - h * (t - tau0) \symbol{94} 2 / sigma\_X \symbol{94} 4 * exp(-(t - tau0) \symbol{94} 2 / sigma\_X \symbol{94} 2 / 0.2e1) - x(1) \symbol{94} 2 * exp(-(t- x(6)) \symbol{94} 2 / x(5) \symbol{94} 2 / 0.2e1) \symbol{94} 2) * x(1) * exp(-(t - x(6)) \symbol{94} 2 / x(5) \symbol{94} 2 / 0.2e1) * (t - x(6)) * (0.2e1 * real(uS) * cos(x(2) + x(3) * (t - x(6)) + x(4) * (t - x(6)) \symbol{94} 2) - 0.2e1 * imag(uS) * sin(x(2) + x(3) * (t - x(6)) + x(4) * (t - x(6)) \symbol{94} 2)) - x(1) \symbol{94} 2 * exp(-(t - x(6)) \symbol{94} 2 / x(5) \symbol{94} 2 / 0.2e1) \symbol{94} 2 * (t - x(6)) * (0.2e1 * imag(uS) \symbol{94} 2 * sin(0.2e1 * x(2) + 0.2e1 * x(3) * (t - x(6)) + 0.2e1 * x(4) * (t - x(6)) \symbol{94} 2) - 0.2e1 * real(uS) \symbol{94} 2 * sin(0.2e1 * x(2) + 0.2e1 * x(3) * (t - x(6)) + 0.2e1 * x(4) *(t - x(6)) \symbol{94} 2) + 0.4e1 * imag(uS) * real(uS) * cos(0.2e1 * x(2) + 0.2e1 * x(3) * (t - x(6)) + 0.2e1 * x(4) * (t - x(6)) \symbol{94} 2));
\underline{}Warning, the function names \{imag, real, x\} are not recognized in the target language\underline{}\mapleresult
cg8 = (h \symbol{94} 2 * (t - tau0) \symbol{94} 2 / sigma\_X \symbol{94} 4 * exp(-(t - tau0) \symbol{94} 2 / sigma\_X \symbol{94} 2 / 0.2e1) \symbol{94} 2 + h / sigma\_X \symbol{94} 2 * exp(-(t - tau0) \symbol{94} 2 / sigma\_X \symbol{94} 2 / 0.2e1) - h * (t - tau0) \symbol{94} 2 / sigma\_X \symbol{94} 4 * exp(-(t - tau0) \symbol{94} 2 / sigma\_X \symbol{94} 2 / 0.2e1) - x(1) \symbol{94} 2 * exp(-(t- x(6)) \symbol{94} 2 / x(5) \symbol{94} 2 / 0.2e1) \symbol{94} 2) * x(1) * exp(-(t - x(6)) \symbol{94} 2 / x(5) \symbol{94} 2 / 0.2e1) * (t - x(6)) \symbol{94} 2 * (0.2e1 * real(uS) * cos(x(2) + x(3) * (t - x(6)) + x(4) * (t - x(6)) \symbol{94} 2) - 0.2e1 * imag(uS) * sin(x(2) + x(3) * (t - x(6)) + x(4) * (t - x(6)) \symbol{94} 2))- x(1) \symbol{94} 2 * exp(-(t - x(6)) \symbol{94} 2 / x(5) \symbol{94} 2 / 0.2e1) \symbol{94} 2 * (t - x(6)) \symbol{94} 2 * (0.2e1 * imag(uS) \symbol{94} 2 * sin(0.2e1 * x(2) + 0.2e1 * x(3) * (t - x(6)) + 0.2e1 * x(4) * (t - x(6)) \symbol{94} 2) - 0.2e1 * real(uS) \symbol{94} 2 * sin(0.2e1 * x(2) + 0.2e1 * x(3) * (t - x(6)) + 0.2e1* x(4) * (t - x(6)) \symbol{94} 2) + 0.4e1 * imag(uS) * real(uS) * cos(0.2e1 * x(2) + 0.2e1 * x(3) * (t - x(6)) + 0.2e1 * x(4) * (t - x(6)) \symbol{94} 2));
\underline{}Warning, the function names \{imag, real, x\} are not recognized in the target language\underline{}\mapleresult
cg9 = (h \symbol{94} 2 * (t - tau0) \symbol{94} 2 / sigma\_X \symbol{94} 4 * exp(-(t - tau0) \symbol{94} 2 / sigma\_X \symbol{94} 2 / 0.2e1) \symbol{94} 2 + h / sigma\_X \symbol{94} 2 * exp(-(t - tau0) \symbol{94} 2 / sigma\_X \symbol{94} 2 / 0.2e1) - h * (t - tau0) \symbol{94} 2 / sigma\_X \symbol{94} 4 * exp(-(t - tau0) \symbol{94} 2 / sigma\_X \symbol{94} 2 / 0.2e1) - 0.3e1 * x(1) \symbol{94} 2 *exp(-(t - x(6)) \symbol{94} 2 / x(5) \symbol{94} 2 / 0.2e1) \symbol{94} 2) * x(1) * exp(-(t - x(6)) \symbol{94} 2 / x(5) \symbol{94} 2 / 0.2e1) * (t - x(6)) \symbol{94} 2 / x(5) \symbol{94} 3 * (0.2e1 * real(uS) * cos(x(2) + x(3) * (t - x(6)) + x(4) * (t - x(6)) \symbol{94} 2) + 0.2e1 * imag(uS) * sin(x(2) + x(3) * (t - x(6)) + x(4)* (t - x(6)) \symbol{94} 2)) - x(1) \symbol{94} 2 * exp(-(t - x(6)) \symbol{94} 2 / x(5) \symbol{94} 2 / 0.2e1) \symbol{94} 2 * (t - x(6)) \symbol{94} 2 / x(5) \symbol{94} 3 * (0.2e1 * real(uS) \symbol{94} 2 * cos(0.2e1 * x(2) + 0.2e1 * x(3) * (t - x(6)) + 0.2e1 * x(4) * (t - x(6)) \symbol{94} 2) - 0.2e1 * imag(uS) \symbol{94} 2 * cos(0.2e1 * x(2) + 0.2e1 * x(3) * (t - x(6)) + 0.2e1 * x(4) * (t - x(6)) \symbol{94} 2) + 0.4e1 * imag(uS) * real(uS) * sin(0.2e1 * x(2) + 0.2e1 * x(3) * (t - x(6)) + 0.2e1 * x(4) * (t - x(6)) \symbol{94} 2));
\underline{}Warning, the function names \{imag, real, x\} are not recognized in the target language\underline{}\mapleresult
cg10 = (h \symbol{94} 2 * (t - tau0) \symbol{94} 2 / sigma\_X \symbol{94} 4 * exp(-(t - tau0) \symbol{94} 2 / sigma\_X \symbol{94} 2 / 0.2e1) \symbol{94} 2 + h / sigma\_X \symbol{94} 2 * exp(-(t - tau0) \symbol{94} 2 / sigma\_X \symbol{94} 2 / 0.2e1) - h * (t - tau0) \symbol{94} 2 / sigma\_X \symbol{94} 4 * exp(-(t - tau0) \symbol{94} 2 / sigma\_X \symbol{94} 2 / 0.2e1) - x(1) \symbol{94} 2 * exp(-(t - x(6)) \symbol{94} 2 / x(5) \symbol{94} 2 / 0.2e1) \symbol{94} 2) * x(1) * exp(-(t - x(6)) \symbol{94} 2 / x(5) \symbol{94} 2 / 0.2e1) * (-x(3) - 0.2e1 * x(4) * (t - x(6))) * (0.2e1 * real(uS) * cos(x(2) + x(3) * (t - x(6)) + x(4) * (t - x(6)) \symbol{94} 2) - 0.2e1 * imag(uS) * sin(x(2) + x(3) * (t - x(6)) + x(4) * (t - x(6)) \symbol{94} 2)) - x(1) \symbol{94} 2 * exp(-(t - x(6)) \symbol{94} 2 / x(5) \symbol{94} 2 / 0.2e1) \symbol{94} 2 * (-x(3) - 0.2e1 * x(4) * (t - x(6))) * (0.2e1 * imag(uS) \symbol{94} 2 * sin(0.2e1 * x(2) + 0.2e1 * x(3) * (t - x(6)) + 0.2e1 * x(4) * (t - x(6)) \symbol{94} 2) - 0.2e1 * real(uS) \symbol{94} 2 * sin(0.2e1* x(2) + 0.2e1 * x(3) * (t - x(6)) + 0.2e1 * x(4) * (t - x(6)) \symbol{94} 2) + 0.4e1 * imag(uS) * real(uS) * cos(0.2e1 * x(2) + 0.2e1 * x(3) * (t - x(6)) + 0.2e1 * x(4) * (t - x(6)) \symbol{94} 2)) + (h \symbol{94} 2 * (t - tau0) \symbol{94} 2 / sigma\_X \symbol{94} 4 * exp(-(t - tau0) \symbol{94} 2 / sigma\_X \symbol{94} 2 /0.2e1) \symbol{94} 2 + h / sigma\_X \symbol{94} 2 * exp(-(t - tau0) \symbol{94} 2 / sigma\_X \symbol{94} 2 / 0.2e1) - h * (t - tau0) \symbol{94} 2 / sigma\_X \symbol{94} 4 * exp(-(t - tau0) \symbol{94} 2 / sigma\_X \symbol{94} 2 / 0.2e1) - 0.3e1 * x(1) \symbol{94} 2 * exp(-(t - x(6)) \symbol{94} 2 / x(5) \symbol{94} 2 / 0.2e1) \symbol{94} 2) * x(1) * exp(-(t - x(6)) \symbol{94} 2 / x(5)\symbol{94} 2 / 0.2e1) * (t - x(6)) * (0.2e1 * real(uS) * cos(x(2) + x(3) * (t - x(6)) + x(4) * (t - x(6)) \symbol{94} 2) + 0.2e1 * imag(uS) * sin(x(2) + x(3) * (t - x(6)) + x(4) * (t - x(6)) \symbol{94} 2)) / x(5) \symbol{94} 2 - x(1) \symbol{94} 2 * exp(-(t - x(6)) \symbol{94} 2 / x(5) \symbol{94} 2 / 0.2e1) \symbol{94} 2 * (t - x(6)) * (0.2e1 * real(uS) \symbol{94} 2 * cos(0.2e1 * x(2) + 0.2e1 * x(3) * (t - x(6)) + 0.2e1 * x(4) * (t - x(6)) \symbol{94} 2) - 0.2e1 * imag(uS) \symbol{94} 2 * cos(0.2e1 * x(2) + 0.2e1 * x(3) * (t - x(6)) + 0.2e1 * x(4) * (t - x(6)) \symbol{94} 2) + 0.4e1 * imag(uS) * real(uS) * sin(0.2e1 * x(2) + 0.2e1 * x(3) * (t - x(6)) + 0.2e1 * x(4) * (t - x(6)) \symbol{94} 2)) / x(5) \symbol{94} 2;
\end{maplegroup}
\begin{maplegroup}
\begin{mapleinput}
\mapleinline{active}{1d}{subLatex := a(z) = a, b(z) = b, c(z)=c, d(z) = d, s(z)=sigma, xi(z) = xi, Ui = imag(uS), Ur=real(uS), alpha = h_phi, beta = sigma_phi: 
}{}
\end{mapleinput}
\end{maplegroup}
\begin{maplegroup}
\begin{mapleinput}
\mapleinline{active}{2d}{F1 := subs({subLatex}, eFIa); 1; F2 := subs({subLatex}, eFIb); 1; F3 := subs({subLatex}, eFIc); 1; F4 := subs({subLatex}, eFId); 1; F5 := subs({subLatex}, eFIs); 1; F6 := subs({subLatex}, eFIxi); 1}{\[\]}
\end{mapleinput}
\mapleresult
\begin{maplelatex}
\mapleinline{inert}{2d}{F1 := (h_phi^2*(t-tau0)^2*(exp(-(1/2)*(t-tau0)^2/sigma_phi^2))^2/sigma_phi^4+h_phi*exp(-(1/2)*(t-tau0)^2/sigma_phi^2)/sigma_phi^2-h_phi*(t-tau0)^2*exp(-(1/2)*(t-tau0)^2/sigma_phi^2)/sigma_phi^4-3*a^2*(exp(-(1/2)*(t-`&xi;`)^2/sigma^2))^2)*exp(-(1/2)*(t-`&xi;`)^2/sigma^2)*(2*real(uS)*cos(b+c*(t-`&xi;`)+d*(t-`&xi;`)^2)+2*imag(uS)*sin(b+c*(t-`&xi;`)+d*(t-`&xi;`)^2))-a*(exp(-(1/2)*(t-`&xi;`)^2/sigma^2))^2*(2*real(uS)^2*cos(2*b+2*c*(t-`&xi;`)+2*d*(t-`&xi;`)^2)-2*imag(uS)^2*cos(2*b+2*c*(t-`&xi;`)+2*d*(t-`&xi;`)^2)+4*imag(uS)*real(uS)*sin(2*b+2*c*(t-`&xi;`)+2*d*(t-`&xi;`)^2))}{\[\displaystyle {\it F1}\, := \, \left( {{\it h\_phi}}^{2} \left( t-{\it tau0} \right) ^{2} \left( {{\rm e}^{-1/2\,{\frac { \left( t-{\it tau0} \right) ^{2}}{{{\it sigma\_phi}}^{2}}}}} \right) ^{2}\\
\mbox{}{{\it sigma\_phi}}^{-4}+{\it h\_phi}\,{{\rm e}^{-1/2\,{\frac { \left( t-{\it tau0} \right) ^{2}}{{{\it sigma\_phi}}^{2}}}}}{{\it sigma\_phi}}^{-2}-{\it h\_phi}\, \left( t-{\it tau0} \right) ^{2}{{\rm e}^{-1/2\,{\frac { \left( t-{\it tau0} \right) ^{2}}{{{\it sigma\_phi}}^{2}}}}}{{\it sigma\_phi}}^{-4}-3\,{a}^{2} \left( {{\rm e}^{-1/2\,{\frac { \left( t-\xi  \right) ^{2}}{{\sigma}^{2}}}}} \right) ^{2}\\
\mbox{} \right) {{\rm e}^{-1/2\,{\frac { \left( t-\xi  \right) ^{2}}{{\sigma}^{2}}}}} \left( 2\,{\it real} \left( {\it uS} \right) \cos \left( b+c \left( t-\xi  \right) +d \left( t-\xi  \right) ^{2} \right) +2\,{\it imag} \left( {\it uS} \right) \sin \left( b+c \left( t-\xi  \right) +d \left( t-\xi  \right) ^{2} \right) \\
\mbox{} \right) -a \left( {{\rm e}^{-1/2\,{\frac { \left( t-\xi  \right) ^{2}}{{\sigma}^{2}}}}} \right) ^{2} \left( 2\, \left( {\it real} \left( {\it uS} \right)  \right) ^{2}\cos \left( 2\,b+2\,c \left( t-\xi  \right) +2\,d \left( t-\xi  \right) ^{2} \right) -2\, \left( {\it imag} \left( {\it uS} \right)  \right) ^{2}\cos \left( 2\,b+2\,c \left( t-\xi  \right) +2\,d \left( t-\xi  \right) ^{2} \right) \\
\mbox{}+4\,{\it imag} \left( {\it uS} \right) {\it real} \left( {\it uS} \right) \sin \left( 2\,b+2\,c \left( t-\xi  \right) +2\,d \left( t-\xi  \right) ^{2} \right)  \right) \]}
\end{maplelatex}
\mapleresult
\begin{maplelatex}
\mapleinline{inert}{2d}{F2 := (h_phi^2*(t-tau0)^2*(exp(-(1/2)*(t-tau0)^2/sigma_phi^2))^2/sigma_phi^4+h_phi*exp(-(1/2)*(t-tau0)^2/sigma_phi^2)/sigma_phi^2-h_phi*(t-tau0)^2*exp(-(1/2)*(t-tau0)^2/sigma_phi^2)/sigma_phi^4-a^2*(exp(-(1/2)*(t-`&xi;`)^2/sigma^2))^2)*a*exp(-(1/2)*(t-`&xi;`)^2/sigma^2)*(2*real(uS)*cos(b+c*(t-`&xi;`)+d*(t-`&xi;`)^2)-2*imag(uS)*sin(b+c*(t-`&xi;`)+d*(t-`&xi;`)^2))-a^2*(exp(-(1/2)*(t-`&xi;`)^2/sigma^2))^2*(2*imag(uS)^2*sin(2*b+2*c*(t-`&xi;`)+2*d*(t-`&xi;`)^2)-2*real(uS)^2*sin(2*b+2*c*(t-`&xi;`)+2*d*(t-`&xi;`)^2)+4*imag(uS)*real(uS)*cos(2*b+2*c*(t-`&xi;`)+2*d*(t-`&xi;`)^2))}{\[\displaystyle {\it F2}\, := \, \left( {{\it h\_phi}}^{2} \left( t-{\it tau0} \right) ^{2} \left( {{\rm e}^{-1/2\,{\frac { \left( t-{\it tau0} \right) ^{2}}{{{\it sigma\_phi}}^{2}}}}} \right) ^{2}\\
\mbox{}{{\it sigma\_phi}}^{-4}+{\it h\_phi}\,{{\rm e}^{-1/2\,{\frac { \left( t-{\it tau0} \right) ^{2}}{{{\it sigma\_phi}}^{2}}}}}{{\it sigma\_phi}}^{-2}-{\it h\_phi}\, \left( t-{\it tau0} \right) ^{2}{{\rm e}^{-1/2\,{\frac { \left( t-{\it tau0} \right) ^{2}}{{{\it sigma\_phi}}^{2}}}}}{{\it sigma\_phi}}^{-4}-{a}^{2} \left( {{\rm e}^{-1/2\,{\frac { \left( t-\xi  \right) ^{2}}{{\sigma}^{2}}}}} \right) ^{2}\\
\mbox{} \right) a{{\rm e}^{-1/2\,{\frac { \left( t-\xi  \right) ^{2}}{{\sigma}^{2}}}}} \left( 2\,{\it real} \left( {\it uS} \right) \cos \left( b+c \left( t-\xi  \right) +d \left( t-\xi  \right) ^{2} \right) -2\,{\it imag} \left( {\it uS} \right) \sin \left( b+c \left( t-\xi  \right) +d \left( t-\xi  \right) ^{2} \right) \\
\mbox{} \right) -{a}^{2} \left( {{\rm e}^{-1/2\,{\frac { \left( t-\xi  \right) ^{2}}{{\sigma}^{2}}}}} \right) ^{2} \left( 2\, \left( {\it imag} \left( {\it uS} \right)  \right) ^{2}\sin \left( 2\,b+2\,c \left( t-\xi  \right) +2\,d \left( t-\xi  \right) ^{2} \right) -2\, \left( {\it real} \left( {\it uS} \right)  \right) ^{2}\sin \left( 2\,b+2\,c \left( t-\xi  \right) +2\,d \left( t-\xi  \right) ^{2} \right) \\
\mbox{}+4\,{\it imag} \left( {\it uS} \right) {\it real} \left( {\it uS} \right) \cos \left( 2\,b+2\,c \left( t-\xi  \right) +2\,d \left( t-\xi  \right) ^{2} \right)  \right) \]}
\end{maplelatex}
\mapleresult
\begin{maplelatex}
\mapleinline{inert}{2d}{F3 := (h_phi^2*(t-tau0)^2*(exp(-(1/2)*(t-tau0)^2/sigma_phi^2))^2/sigma_phi^4+h_phi*exp(-(1/2)*(t-tau0)^2/sigma_phi^2)/sigma_phi^2-h_phi*(t-tau0)^2*exp(-(1/2)*(t-tau0)^2/sigma_phi^2)/sigma_phi^4-a^2*(exp(-(1/2)*(t-`&xi;`)^2/sigma^2))^2)*a*exp(-(1/2)*(t-`&xi;`)^2/sigma^2)*(t-`&xi;`)*(2*real(uS)*cos(b+c*(t-`&xi;`)+d*(t-`&xi;`)^2)-2*imag(uS)*sin(b+c*(t-`&xi;`)+d*(t-`&xi;`)^2))-a^2*(exp(-(1/2)*(t-`&xi;`)^2/sigma^2))^2*(t-`&xi;`)*(2*imag(uS)^2*sin(2*b+2*c*(t-`&xi;`)+2*d*(t-`&xi;`)^2)-2*real(uS)^2*sin(2*b+2*c*(t-`&xi;`)+2*d*(t-`&xi;`)^2)+4*imag(uS)*real(uS)*cos(2*b+2*c*(t-`&xi;`)+2*d*(t-`&xi;`)^2))}{\[\displaystyle {\it F3}\, := \, \left( {{\it h\_phi}}^{2} \left( t-{\it tau0} \right) ^{2} \left( {{\rm e}^{-1/2\,{\frac { \left( t-{\it tau0} \right) ^{2}}{{{\it sigma\_phi}}^{2}}}}} \right) ^{2}\\
\mbox{}{{\it sigma\_phi}}^{-4}+{\it h\_phi}\,{{\rm e}^{-1/2\,{\frac { \left( t-{\it tau0} \right) ^{2}}{{{\it sigma\_phi}}^{2}}}}}{{\it sigma\_phi}}^{-2}-{\it h\_phi}\, \left( t-{\it tau0} \right) ^{2}{{\rm e}^{-1/2\,{\frac { \left( t-{\it tau0} \right) ^{2}}{{{\it sigma\_phi}}^{2}}}}}{{\it sigma\_phi}}^{-4}-{a}^{2} \left( {{\rm e}^{-1/2\,{\frac { \left( t-\xi  \right) ^{2}}{{\sigma}^{2}}}}} \right) ^{2}\\
\mbox{} \right) a{{\rm e}^{-1/2\,{\frac { \left( t-\xi  \right) ^{2}}{{\sigma}^{2}}}}} \left( t-\xi  \right)  \left( 2\,{\it real} \left( {\it uS} \right) \cos \left( b+c \left( t-\xi  \right) +d \left( t-\xi  \right) ^{2} \right) -2\,{\it imag} \left( {\it uS} \right) \sin \left( b+c \left( t-\xi  \right) +d \left( t-\xi  \right) ^{2} \right) \\
\mbox{} \right) -{a}^{2} \left( {{\rm e}^{-1/2\,{\frac { \left( t-\xi  \right) ^{2}}{{\sigma}^{2}}}}} \right) ^{2} \left( t-\xi  \right)  \left( 2\, \left( {\it imag} \left( {\it uS} \right)  \right) ^{2}\sin \left( 2\,b+2\,c \left( t-\xi  \right) +2\,d \left( t-\xi  \right) ^{2} \right) -2\, \left( {\it real} \left( {\it uS} \right)  \right) ^{2}\sin \left( 2\,b+2\,c \left( t-\xi  \right) +2\,d \left( t-\xi  \right) ^{2} \right) \\
\mbox{}+4\,{\it imag} \left( {\it uS} \right) {\it real} \left( {\it uS} \right) \cos \left( 2\,b+2\,c \left( t-\xi  \right) +2\,d \left( t-\xi  \right) ^{2} \right)  \right) \]}
\end{maplelatex}
\mapleresult
\begin{maplelatex}
\mapleinline{inert}{2d}{F4 := (h_phi^2*(t-tau0)^2*(exp(-(1/2)*(t-tau0)^2/sigma_phi^2))^2/sigma_phi^4+h_phi*exp(-(1/2)*(t-tau0)^2/sigma_phi^2)/sigma_phi^2-h_phi*(t-tau0)^2*exp(-(1/2)*(t-tau0)^2/sigma_phi^2)/sigma_phi^4-a^2*(exp(-(1/2)*(t-`&xi;`)^2/sigma^2))^2)*a*exp(-(1/2)*(t-`&xi;`)^2/sigma^2)*(t-`&xi;`)^2*(2*real(uS)*cos(b+c*(t-`&xi;`)+d*(t-`&xi;`)^2)-2*imag(uS)*sin(b+c*(t-`&xi;`)+d*(t-`&xi;`)^2))-a^2*(exp(-(1/2)*(t-`&xi;`)^2/sigma^2))^2*(t-`&xi;`)^2*(2*imag(uS)^2*sin(2*b+2*c*(t-`&xi;`)+2*d*(t-`&xi;`)^2)-2*real(uS)^2*sin(2*b+2*c*(t-`&xi;`)+2*d*(t-`&xi;`)^2)+4*imag(uS)*real(uS)*cos(2*b+2*c*(t-`&xi;`)+2*d*(t-`&xi;`)^2))}{\[\displaystyle {\it F4}\, := \, \left( {{\it h\_phi}}^{2} \left( t-{\it tau0} \right) ^{2} \left( {{\rm e}^{-1/2\,{\frac { \left( t-{\it tau0} \right) ^{2}}{{{\it sigma\_phi}}^{2}}}}} \right) ^{2}\\
\mbox{}{{\it sigma\_phi}}^{-4}+{\it h\_phi}\,{{\rm e}^{-1/2\,{\frac { \left( t-{\it tau0} \right) ^{2}}{{{\it sigma\_phi}}^{2}}}}}{{\it sigma\_phi}}^{-2}-{\it h\_phi}\, \left( t-{\it tau0} \right) ^{2}{{\rm e}^{-1/2\,{\frac { \left( t-{\it tau0} \right) ^{2}}{{{\it sigma\_phi}}^{2}}}}}{{\it sigma\_phi}}^{-4}-{a}^{2} \left( {{\rm e}^{-1/2\,{\frac { \left( t-\xi  \right) ^{2}}{{\sigma}^{2}}}}} \right) ^{2}\\
\mbox{} \right) a{{\rm e}^{-1/2\,{\frac { \left( t-\xi  \right) ^{2}}{{\sigma}^{2}}}}} \left( t-\xi  \right) ^{2} \left( 2\,{\it real} \left( {\it uS} \right) \cos \left( b+c \left( t-\xi  \right) +d \left( t-\xi  \right) ^{2} \right) -2\,{\it imag} \left( {\it uS} \right) \sin \left( b+c \left( t-\xi  \right) +d \left( t-\xi  \right) ^{2} \right) \\
\mbox{} \right) -{a}^{2} \left( {{\rm e}^{-1/2\,{\frac { \left( t-\xi  \right) ^{2}}{{\sigma}^{2}}}}} \right) ^{2} \left( t-\xi  \right) ^{2} \left( 2\, \left( {\it imag} \left( {\it uS} \right)  \right) ^{2}\sin \left( 2\,b+2\,c \left( t-\xi  \right) +2\,d \left( t-\xi  \right) ^{2} \right) -2\, \left( {\it real} \left( {\it uS} \right)  \right) ^{2}\sin \left( 2\,b+2\,c \left( t-\xi  \right) +2\,d \left( t-\xi  \right) ^{2} \right) \\
\mbox{}+4\,{\it imag} \left( {\it uS} \right) {\it real} \left( {\it uS} \right) \cos \left( 2\,b+2\,c \left( t-\xi  \right) +2\,d \left( t-\xi  \right) ^{2} \right)  \right) \]}
\end{maplelatex}
\mapleresult
\begin{maplelatex}
\mapleinline{inert}{2d}{F5 := (h_phi^2*(t-tau0)^2*(exp(-(1/2)*(t-tau0)^2/sigma_phi^2))^2/sigma_phi^4+h_phi*exp(-(1/2)*(t-tau0)^2/sigma_phi^2)/sigma_phi^2-h_phi*(t-tau0)^2*exp(-(1/2)*(t-tau0)^2/sigma_phi^2)/sigma_phi^4-3*a^2*(exp(-(1/2)*(t-`&xi;`)^2/sigma^2))^2)*a*exp(-(1/2)*(t-`&xi;`)^2/sigma^2)*(t-`&xi;`)^2*(2*real(uS)*cos(b+c*(t-`&xi;`)+d*(t-`&xi;`)^2)+2*imag(uS)*sin(b+c*(t-`&xi;`)+d*(t-`&xi;`)^2))/sigma^3-a^2*(exp(-(1/2)*(t-`&xi;`)^2/sigma^2))^2*(t-`&xi;`)^2*(2*real(uS)^2*cos(2*b+2*c*(t-`&xi;`)+2*d*(t-`&xi;`)^2)-2*imag(uS)^2*cos(2*b+2*c*(t-`&xi;`)+2*d*(t-`&xi;`)^2)+4*imag(uS)*real(uS)*sin(2*b+2*c*(t-`&xi;`)+2*d*(t-`&xi;`)^2))/sigma^3}{\[\displaystyle {\it F5}\, := \, \left( {{\it h\_phi}}^{2} \left( t-{\it tau0} \right) ^{2} \left( {{\rm e}^{-1/2\,{\frac { \left( t-{\it tau0} \right) ^{2}}{{{\it sigma\_phi}}^{2}}}}} \right) ^{2}\\
\mbox{}{{\it sigma\_phi}}^{-4}+{\it h\_phi}\,{{\rm e}^{-1/2\,{\frac { \left( t-{\it tau0} \right) ^{2}}{{{\it sigma\_phi}}^{2}}}}}{{\it sigma\_phi}}^{-2}-{\it h\_phi}\, \left( t-{\it tau0} \right) ^{2}{{\rm e}^{-1/2\,{\frac { \left( t-{\it tau0} \right) ^{2}}{{{\it sigma\_phi}}^{2}}}}}{{\it sigma\_phi}}^{-4}-3\,{a}^{2} \left( {{\rm e}^{-1/2\,{\frac { \left( t-\xi  \right) ^{2}}{{\sigma}^{2}}}}} \right) ^{2}\\
\mbox{} \right) a{{\rm e}^{-1/2\,{\frac { \left( t-\xi  \right) ^{2}}{{\sigma}^{2}}}}} \left( t-\xi  \right) ^{2} \left( 2\,{\it real} \left( {\it uS} \right) \cos \left( b+c \left( t-\xi  \right) +d \left( t-\xi  \right) ^{2} \right) +2\,{\it imag} \left( {\it uS} \right) \sin \left( b+c \left( t-\xi  \right) +d \left( t-\xi  \right) ^{2} \right) \\
\mbox{} \right) {\sigma}^{-3}-{a}^{2} \left( {{\rm e}^{-1/2\,{\frac { \left( t-\xi  \right) ^{2}}{{\sigma}^{2}}}}} \right) ^{2} \left( t-\xi  \right) ^{2} \left( 2\, \left( {\it real} \left( {\it uS} \right)  \right) ^{2}\cos \left( 2\,b+2\,c \left( t-\xi  \right) +2\,d \left( t-\xi  \right) ^{2} \right) -2\, \left( {\it imag} \left( {\it uS} \right)  \right) ^{2}\cos \left( 2\,b+2\,c \left( t-\xi  \right) +2\,d \left( t-\xi  \right) ^{2} \right) \\
\mbox{}+4\,{\it imag} \left( {\it uS} \right) {\it real} \left( {\it uS} \right) \sin \left( 2\,b+2\,c \left( t-\xi  \right) +2\,d \left( t-\xi  \right) ^{2} \right)  \right) \\
\mbox{}{\sigma}^{-3}\]}
\end{maplelatex}
\mapleresult
\begin{maplelatex}
\mapleinline{inert}{2d}{F6 := (h_phi^2*(t-tau0)^2*(exp(-(1/2)*(t-tau0)^2/sigma_phi^2))^2/sigma_phi^4+h_phi*exp(-(1/2)*(t-tau0)^2/sigma_phi^2)/sigma_phi^2-h_phi*(t-tau0)^2*exp(-(1/2)*(t-tau0)^2/sigma_phi^2)/sigma_phi^4-a^2*(exp(-(1/2)*(t-`&xi;`)^2/sigma^2))^2)*a*exp(-(1/2)*(t-`&xi;`)^2/sigma^2)*(-c-2*d*(t-`&xi;`))*(2*real(uS)*cos(b+c*(t-`&xi;`)+d*(t-`&xi;`)^2)-2*imag(uS)*sin(b+c*(t-`&xi;`)+d*(t-`&xi;`)^2))-a^2*(exp(-(1/2)*(t-`&xi;`)^2/sigma^2))^2*(-c-2*d*(t-`&xi;`))*(2*imag(uS)^2*sin(2*b+2*c*(t-`&xi;`)+2*d*(t-`&xi;`)^2)-2*real(uS)^2*sin(2*b+2*c*(t-`&xi;`)+2*d*(t-`&xi;`)^2)+4*imag(uS)*real(uS)*cos(2*b+2*c*(t-`&xi;`)+2*d*(t-`&xi;`)^2))+(h_phi^2*(t-tau0)^2*(exp(-(1/2)*(t-tau0)^2/sigma_phi^2))^2/sigma_phi^4+h_phi*exp(-(1/2)*(t-tau0)^2/sigma_phi^2)/sigma_phi^2-h_phi*(t-tau0)^2*exp(-(1/2)*(t-tau0)^2/sigma_phi^2)/sigma_phi^4-3*a^2*(exp(-(1/2)*(t-`&xi;`)^2/sigma^2))^2)*a*exp(-(1/2)*(t-`&xi;`)^2/sigma^2)*(t-`&xi;`)*(2*real(uS)*cos(b+c*(t-`&xi;`)+d*(t-`&xi;`)^2)+2*imag(uS)*sin(b+c*(t-`&xi;`)+d*(t-`&xi;`)^2))/sigma^2-a^2*(exp(-(1/2)*(t-`&xi;`)^2/sigma^2))^2*(t-`&xi;`)*(2*real(uS)^2*cos(2*b+2*c*(t-`&xi;`)+2*d*(t-`&xi;`)^2)-2*imag(uS)^2*cos(2*b+2*c*(t-`&xi;`)+2*d*(t-`&xi;`)^2)+4*imag(uS)*real(uS)*sin(2*b+2*c*(t-`&xi;`)+2*d*(t-`&xi;`)^2))/sigma^2}{\[\displaystyle {\it F6}\, := \, \left( {{\it h\_phi}}^{2} \left( t-{\it tau0} \right) ^{2} \left( {{\rm e}^{-1/2\,{\frac { \left( t-{\it tau0} \right) ^{2}}{{{\it sigma\_phi}}^{2}}}}} \right) ^{2}\\
\mbox{}{{\it sigma\_phi}}^{-4}+{\it h\_phi}\,{{\rm e}^{-1/2\,{\frac { \left( t-{\it tau0} \right) ^{2}}{{{\it sigma\_phi}}^{2}}}}}{{\it sigma\_phi}}^{-2}-{\it h\_phi}\, \left( t-{\it tau0} \right) ^{2}{{\rm e}^{-1/2\,{\frac { \left( t-{\it tau0} \right) ^{2}}{{{\it sigma\_phi}}^{2}}}}}{{\it sigma\_phi}}^{-4}-{a}^{2} \left( {{\rm e}^{-1/2\,{\frac { \left( t-\xi  \right) ^{2}}{{\sigma}^{2}}}}} \right) ^{2}\\
\mbox{} \right) a{{\rm e}^{-1/2\,{\frac { \left( t-\xi  \right) ^{2}}{{\sigma}^{2}}}}} \left( -c-2\,d \left( t-\xi  \right)  \right) \\
\mbox{} \left( 2\,{\it real} \left( {\it uS} \right) \cos \left( b+c \left( t-\xi  \right) +d \left( t-\xi  \right) ^{2} \right) -2\,{\it imag} \left( {\it uS} \right) \sin \left( b+c \left( t-\xi  \right) +d \left( t-\xi  \right) ^{2} \right) \\
\mbox{} \right) -{a}^{2} \left( {{\rm e}^{-1/2\,{\frac { \left( t-\xi  \right) ^{2}}{{\sigma}^{2}}}}} \right) ^{2} \left( -c-2\,d \left( t-\xi  \right)  \right)  \left( 2\, \left( {\it imag} \left( {\it uS} \right)  \right) ^{2}\sin \left( 2\,b+2\,c \left( t-\xi  \right) +2\,d \left( t-\xi  \right) ^{2} \right) -2\, \left( {\it real} \left( {\it uS} \right)  \right) ^{2}\sin \left( 2\,b+2\,c \left( t-\xi  \right) +2\,d \left( t-\xi  \right) ^{2} \right) \\
\mbox{}+4\,{\it imag} \left( {\it uS} \right) {\it real} \left( {\it uS} \right) \cos \left( 2\,b+2\,c \left( t-\xi  \right) +2\,d \left( t-\xi  \right) ^{2} \right)  \right) \\
\mbox{}+ \left( {{\it h\_phi}}^{2} \left( t-{\it tau0} \right) ^{2} \left( {{\rm e}^{-1/2\,{\frac { \left( t-{\it tau0} \right) ^{2}}{{{\it sigma\_phi}}^{2}}}}} \right) ^{2}{{\it sigma\_phi}}^{-4}+{\it h\_phi}\,{{\rm e}^{-1/2\,{\frac { \left( t-{\it tau0} \right) ^{2}}{{{\it sigma\_phi}}^{2}}}}}{{\it sigma\_phi}}^{-2}-{\it h\_phi}\, \left( t-{\it tau0} \right) ^{2}{{\rm e}^{-1/2\,{\frac { \left( t-{\it tau0} \right) ^{2}}{{{\it sigma\_phi}}^{2}}}}}{{\it sigma\_phi}}^{-4}\\
\mbox{}-3\,{a}^{2} \left( {{\rm e}^{-1/2\,{\frac { \left( t-\xi  \right) ^{2}}{{\sigma}^{2}}}}} \right) ^{2} \right) a{{\rm e}^{-1/2\,{\frac { \left( t-\xi  \right) ^{2}}{{\sigma}^{2}}}}}\\
\mbox{} \left( t-\xi  \right)  \left( 2\,{\it real} \left( {\it uS} \right) \cos \left( b+c \left( t-\xi  \right) +d \left( t-\xi  \right) ^{2} \right) +2\,{\it imag} \left( {\it uS} \right) \sin \left( b+c \left( t-\xi  \right) +d \left( t-\xi  \right) ^{2} \right)  \right) {\sigma}^{-2}-{a}^{2} \left( {{\rm e}^{-1/2\,{\frac { \left( t-\xi  \right) ^{2}}{{\sigma}^{2}}}}} \right) ^{2} \left( t-\xi  \right)  \left( 2\, \left( {\it real} \left( {\it uS} \right)  \right) ^{2}\cos \left( 2\,b+2\,c \left( t-\xi  \right) +2\,d \left( t-\xi  \right) ^{2} \right) -2\, \left( {\it imag} \left( {\it uS} \right)  \right) ^{2}\cos \left( 2\,b+2\,c \left( t-\xi  \right) +2\,d \left( t-\xi  \right) ^{2} \right) \\
\mbox{}+4\,{\it imag} \left( {\it uS} \right) {\it real} \left( {\it uS} \right) \sin \left( 2\,b+2\,c \left( t-\xi  \right) +2\,d \left( t-\xi  \right) ^{2} \right)  \right) \\
\mbox{}{\sigma}^{-2}\]}
\end{maplelatex}
\end{maplegroup}
\begin{maplegroup}
\begin{mapleinput}
\mapleinline{active}{2d}{latex(F1); 1}{\[\]}
\end{mapleinput}
\mapleresult
\symbol{92}left( \{\{\symbol{92}it h\symbol{92}\_phi\}\}\symbol{94}\{2\} \symbol{92}left( t-\{\symbol{92}it tau0\} \symbol{92}right) \symbol{94}\{2\} \symbol{92}left( \{
\{\symbol{92}rm e\}\symbol{94}\{-1/2\symbol{92},\{\symbol{92}frac \{ \symbol{92}left( t-\{\symbol{92}it tau0\} \symbol{92}right) \symbol{94}\{2\}\}\{\{\{\symbol{92}itsigma\symbol{92}\_phi\}\}\symbol{94}\{2\}\}\}\}\} \symbol{92}right) \symbol{94}\{2\}\{\{\symbol{92}it sigma\symbol{92}\_phi\}\}\symbol{94}\{-4\}+\{\symbol{92}it h\symbol{92}\_phi\}\symbol{92},\{\{\symbol{92}rm e\}\symbol{94}\{-1/2\symbol{92},\{\symbol{92}frac \{ \symbol{92}left( t-\{\symbol{92}it tau0\} \symbol{92}right) \symbol{94}\{2\}\}\{\{\{\symbol{92}itsigma\symbol{92}\_phi\}\}\symbol{94}\{2\}\}\}\}\}\{\{\symbol{92}it sigma\symbol{92}\_phi\}\}\symbol{94}\{-2\}-\{\symbol{92}it h\symbol{92}\_phi\}\symbol{92}, \symbol{92}left( t-\{\symbol{92}it tau0\} \symbol{92}right) \symbol{94}\{2\}\{\{\symbol{92}rm e\}\symbol{94}\{-1/2\symbol{92},\{\symbol{92}frac \{ \symbol{92}left( t-\{\symbol{92}it tau0\}\symbol{92}right) \symbol{94}\{2\}\}\{\{\{\symbol{92}it sigma\symbol{92}\_phi\}\}\symbol{94}\{2\}\}\}\}\}\{\{\symbol{92}it sigma\symbol{92}\_phi\}\}\symbol{94}\{-4\}-3\symbol{92},\{a\}\symbol{94}\{2\} \symbol{92}left( \{\{\symbol{92}rm e\}\symbol{94}\{-1/2\symbol{92},\{\symbol{92}frac \{ \symbol{92}left( t-\{\symbol{92}it xi\} \symbol{92}right) \symbol{94}\{2\}\}\{\{\symbol{92}sigma\}\symbol{94}\{2\}\}\}\}\} \symbol{92}right) \symbol{94}\{2\} \symbol{92}right) \{\{\symbol{92}rm e\}\symbol{94}\{-1/2\symbol{92},\{\symbol{92}frac \{ \symbol{92}left(t-\{\symbol{92}it xi\} \symbol{92}right) \symbol{94}\{2\}\}\{\{\symbol{92}sigma\}\symbol{94}\{2\}\}\}\}\} \symbol{92}left( 2\symbol{92},\{\symbol{92}it real\} \symbol{92}left(\{\symbol{92}it uS\} \symbol{92}right) \symbol{92}cos \symbol{92}left( b+c\symbol{92}, \symbol{92}left( t-\{\symbol{92}it xi\} \symbol{92}right) +d\symbol{92},\symbol{92}left( t-\{\symbol{92}it xi\} \symbol{92}right) \symbol{94}\{2\} \symbol{92}right) +2\symbol{92},\{\symbol{92}it imag\} \symbol{92}left( \{\symbol{92}it uS\}\symbol{92}right) \symbol{92}sin \symbol{92}left( b+c\symbol{92}, \symbol{92}left( t-\{\symbol{92}it xi\} \symbol{92}right) +d\symbol{92}, \symbol{92}left( t-\{\symbol{92}it xi\} \symbol{92}right) \symbol{94}\{2\} \symbol{92}right)  \symbol{92}right) -a \symbol{92}left( \{\{\symbol{92}rm e\}\symbol{94}\{-1/2\symbol{92},\{\symbol{92}frac \{ \symbol{92}left( t-\{\symbol{92}it xi\} \symbol{92}right) \symbol{94}\{2\}\}\{\{\symbol{92}sigma\}\symbol{94}\{2\}\}\}\}\} \symbol{92}right) \symbol{94}\{2\}\symbol{92}left( 2\symbol{92}, \symbol{92}left( \{\symbol{92}it real\} \symbol{92}left( \{\symbol{92}it uS\} \symbol{92}right)  \symbol{92}right) \symbol{94}\{2\}\symbol{92}cos \symbol{92}left( 2\symbol{92},b+2\symbol{92},c\symbol{92}, \symbol{92}left( t-\{\symbol{92}it xi\} \symbol{92}right) +2\symbol{92},d\symbol{92}, \symbol{92}left( t-\{\symbol{92}it xi\} \symbol{92}right) \symbol{94}\{2\} \symbol{92}right) -2\symbol{92}, \symbol{92}left( \{\symbol{92}it imag\} \symbol{92}left( \{\symbol{92}it uS\}\symbol{92}right)  \symbol{92}right) \symbol{94}\{2\}\symbol{92}cos \symbol{92}left( 2\symbol{92},b+2\symbol{92},c\symbol{92}, \symbol{92}left( t-\{\symbol{92}it xi\}\symbol{92}right) +2\symbol{92},d\symbol{92}, \symbol{92}left( t-\{\symbol{92}it xi\} \symbol{92}right) \symbol{94}\{2\} \symbol{92}right) +4\symbol{92},\{\symbol{92}it imag\}\symbol{92}left( \{\symbol{92}it uS\} \symbol{92}right) \{\symbol{92}it real\} \symbol{92}left( \{\symbol{92}it uS\} \symbol{92}right) \symbol{92}sin\symbol{92}left( 2\symbol{92},b+2\symbol{92},c\symbol{92}, \symbol{92}left( t-\{\symbol{92}it xi\} \symbol{92}right) +2\symbol{92},d\symbol{92}, \symbol{92}left( t-\{\symbol{92}it xi\} \symbol{92}right) \symbol{94}\{2\} \symbol{92}right)  \symbol{92}right)\end{maplegroup}
\begin{maplegroup}
\begin{mapleinput}
\mapleinline{active}{2d}{latex(F2); 1}{\[\]}
\end{mapleinput}
\mapleresult
\symbol{92}left( \{\{\symbol{92}it h\symbol{92}\_phi\}\}\symbol{94}\{2\} \symbol{92}left( t-\{\symbol{92}it tau0\} \symbol{92}right) \symbol{94}\{2\} \symbol{92}left( \{
\{\symbol{92}rm e\}\symbol{94}\{-1/2\symbol{92},\{\symbol{92}frac \{ \symbol{92}left( t-\{\symbol{92}it tau0\} \symbol{92}right) \symbol{94}\{2\}\}\{\{\{\symbol{92}itsigma\symbol{92}\_phi\}\}\symbol{94}\{2\}\}\}\}\} \symbol{92}right) \symbol{94}\{2\}\{\{\symbol{92}it sigma\symbol{92}\_phi\}\}\symbol{94}\{-4\}+\{\symbol{92}it h\symbol{92}\_phi\}\symbol{92},\{\{\symbol{92}rm e\}\symbol{94}\{-1/2\symbol{92},\{\symbol{92}frac \{ \symbol{92}left( t-\{\symbol{92}it tau0\} \symbol{92}right) \symbol{94}\{2\}\}\{\{\{\symbol{92}itsigma\symbol{92}\_phi\}\}\symbol{94}\{2\}\}\}\}\}\{\{\symbol{92}it sigma\symbol{92}\_phi\}\}\symbol{94}\{-2\}-\{\symbol{92}it h\symbol{92}\_phi\}\symbol{92}, \symbol{92}left( t-\{\symbol{92}it tau0\} \symbol{92}right) \symbol{94}\{2\}\{\{\symbol{92}rm e\}\symbol{94}\{-1/2\symbol{92},\{\symbol{92}frac \{ \symbol{92}left( t-\{\symbol{92}it tau0\}\symbol{92}right) \symbol{94}\{2\}\}\{\{\{\symbol{92}it sigma\symbol{92}\_phi\}\}\symbol{94}\{2\}\}\}\}\}\{\{\symbol{92}it sigma\symbol{92}\_phi\}\}\symbol{94}\{-4\}-\{a\}\symbol{94}\{2\} \symbol{92}left( \{\{\symbol{92}rm e\}\symbol{94}\{-1/2\symbol{92},\{\symbol{92}frac \{ \symbol{92}left( t-\{\symbol{92}it xi\} \symbol{92}right) \symbol{94}\{2\}\}\{\{\symbol{92}sigma\}\symbol{94}\{2\}\}\}\}\} \symbol{92}right) \symbol{94}\{2\} \symbol{92}right) a\{\{\symbol{92}rm e\}\symbol{94}\{-1/2\symbol{92},\{\symbol{92}frac \{ \symbol{92}left(t-\{\symbol{92}it xi\} \symbol{92}right) \symbol{94}\{2\}\}\{\{\symbol{92}sigma\}\symbol{94}\{2\}\}\}\}\} \symbol{92}left( 2\symbol{92},\{\symbol{92}it real\} \symbol{92}left(\{\symbol{92}it uS\} \symbol{92}right) \symbol{92}cos \symbol{92}left( b+c\symbol{92}, \symbol{92}left( t-\{\symbol{92}it xi\} \symbol{92}right) +d\symbol{92},\symbol{92}left( t-\{\symbol{92}it xi\} \symbol{92}right) \symbol{94}\{2\} \symbol{92}right) -2\symbol{92},\{\symbol{92}it imag\} \symbol{92}left( \{\symbol{92}it uS\}\symbol{92}right) \symbol{92}sin \symbol{92}left( b+c\symbol{92}, \symbol{92}left( t-\{\symbol{92}it xi\} \symbol{92}right) +d\symbol{92}, \symbol{92}left( t-\{\symbol{92}it xi\} \symbol{92}right) \symbol{94}\{2\} \symbol{92}right)  \symbol{92}right) -\{a\}\symbol{94}\{2\} \symbol{92}left( \{\{\symbol{92}rm e\}\symbol{94}\{-1/2\symbol{92},\{\symbol{92}frac \{ \symbol{92}left( t-\{\symbol{92}it xi\} \symbol{92}right) \symbol{94}\{2\}\}\{\{\symbol{92}sigma\}\symbol{94}\{2\}\}\}\}\} \symbol{92}right) \symbol{94}\{2\}\symbol{92}left( 2\symbol{92}, \symbol{92}left( \{\symbol{92}it imag\} \symbol{92}left( \{\symbol{92}it uS\} \symbol{92}right)  \symbol{92}right) \symbol{94}\{2\}\symbol{92}sin \symbol{92}left( 2\symbol{92},b+2\symbol{92},c\symbol{92}, \symbol{92}left( t-\{\symbol{92}it xi\} \symbol{92}right) +2\symbol{92},d\symbol{92}, \symbol{92}left( t-\{\symbol{92}it xi\} \symbol{92}right) \symbol{94}\{2\} \symbol{92}right) -2\symbol{92}, \symbol{92}left( \{\symbol{92}it real\} \symbol{92}left( \{\symbol{92}it uS\}\symbol{92}right)  \symbol{92}right) \symbol{94}\{2\}\symbol{92}sin \symbol{92}left( 2\symbol{92},b+2\symbol{92},c\symbol{92}, \symbol{92}left( t-\{\symbol{92}it xi\}\symbol{92}right) +2\symbol{92},d\symbol{92}, \symbol{92}left( t-\{\symbol{92}it xi\} \symbol{92}right) \symbol{94}\{2\} \symbol{92}right) +4\symbol{92},\{\symbol{92}it imag\}\symbol{92}left( \{\symbol{92}it uS\} \symbol{92}right) \{\symbol{92}it real\} \symbol{92}left( \{\symbol{92}it uS\} \symbol{92}right) \symbol{92}cos\symbol{92}left( 2\symbol{92},b+2\symbol{92},c\symbol{92}, \symbol{92}left( t-\{\symbol{92}it xi\} \symbol{92}right) +2\symbol{92},d\symbol{92}, \symbol{92}left( t-\{\symbol{92}it xi\} \symbol{92}right) \symbol{94}\{2\} \symbol{92}right)  \symbol{92}right)\end{maplegroup}
\begin{maplegroup}
\begin{mapleinput}
\mapleinline{active}{2d}{latex(F3); 1}{\[\]}
\end{mapleinput}
\mapleresult
\symbol{92}left( \{\{\symbol{92}it h\symbol{92}\_phi\}\}\symbol{94}\{2\} \symbol{92}left( t-\{\symbol{92}it tau0\} \symbol{92}right) \symbol{94}\{2\} \symbol{92}left( \{
\{\symbol{92}rm e\}\symbol{94}\{-1/2\symbol{92},\{\symbol{92}frac \{ \symbol{92}left( t-\{\symbol{92}it tau0\} \symbol{92}right) \symbol{94}\{2\}\}\{\{\{\symbol{92}itsigma\symbol{92}\_phi\}\}\symbol{94}\{2\}\}\}\}\} \symbol{92}right) \symbol{94}\{2\}\{\{\symbol{92}it sigma\symbol{92}\_phi\}\}\symbol{94}\{-4\}+\{\symbol{92}it h\symbol{92}\_phi\}\symbol{92},\{\{\symbol{92}rm e\}\symbol{94}\{-1/2\symbol{92},\{\symbol{92}frac \{ \symbol{92}left( t-\{\symbol{92}it tau0\} \symbol{92}right) \symbol{94}\{2\}\}\{\{\{\symbol{92}itsigma\symbol{92}\_phi\}\}\symbol{94}\{2\}\}\}\}\}\{\{\symbol{92}it sigma\symbol{92}\_phi\}\}\symbol{94}\{-2\}-\{\symbol{92}it h\symbol{92}\_phi\}\symbol{92}, \symbol{92}left( t-\{\symbol{92}it tau0\} \symbol{92}right) \symbol{94}\{2\}\{\{\symbol{92}rm e\}\symbol{94}\{-1/2\symbol{92},\{\symbol{92}frac \{ \symbol{92}left( t-\{\symbol{92}it tau0\}\symbol{92}right) \symbol{94}\{2\}\}\{\{\{\symbol{92}it sigma\symbol{92}\_phi\}\}\symbol{94}\{2\}\}\}\}\}\{\{\symbol{92}it sigma\symbol{92}\_phi\}\}\symbol{94}\{-4\}-\{a\}\symbol{94}\{2\} \symbol{92}left( \{\{\symbol{92}rm e\}\symbol{94}\{-1/2\symbol{92},\{\symbol{92}frac \{ \symbol{92}left( t-\{\symbol{92}it xi\} \symbol{92}right) \symbol{94}\{2\}\}\{\{\symbol{92}sigma\}\symbol{94}\{2\}\}\}\}\} \symbol{92}right) \symbol{94}\{2\} \symbol{92}right) a\{\{\symbol{92}rm e\}\symbol{94}\{-1/2\symbol{92},\{\symbol{92}frac \{ \symbol{92}left(t-\{\symbol{92}it xi\} \symbol{92}right) \symbol{94}\{2\}\}\{\{\symbol{92}sigma\}\symbol{94}\{2\}\}\}\}\} \symbol{92}left( t-\{\symbol{92}it xi\} \symbol{92}right)\symbol{92}left( 2\symbol{92},\{\symbol{92}it real\} \symbol{92}left( \{\symbol{92}it uS\} \symbol{92}right) \symbol{92}cos \symbol{92}left( b+c\symbol{92},\symbol{92}left( t-\{\symbol{92}it xi\} \symbol{92}right) +d\symbol{92}, \symbol{92}left( t-\{\symbol{92}it xi\} \symbol{92}right) \symbol{94}\{2\}\symbol{92}right) -2\symbol{92},\{\symbol{92}it imag\} \symbol{92}left( \{\symbol{92}it uS\} \symbol{92}right) \symbol{92}sin \symbol{92}left( b+c\symbol{92},\symbol{92}left( t-\{\symbol{92}it xi\} \symbol{92}right) +d\symbol{92}, \symbol{92}left( t-\{\symbol{92}it xi\} \symbol{92}right) \symbol{94}\{2\}\symbol{92}right)  \symbol{92}right) -\{a\}\symbol{94}\{2\} \symbol{92}left( \{\{\symbol{92}rm e\}\symbol{94}\{-1/2\symbol{92},\{\symbol{92}frac \{ \symbol{92}left( t-\{\symbol{92}it xi\} \symbol{92}right) \symbol{94}\{2\}\}\{\{\symbol{92}sigma\}\symbol{94}\{2\}\}\}\}\} \symbol{92}right) \symbol{94}\{2\} \symbol{92}left( t-\{\symbol{92}it xi\}\symbol{92}right)  \symbol{92}left( 2\symbol{92}, \symbol{92}left( \{\symbol{92}it imag\} \symbol{92}left( \{\symbol{92}it uS\} \symbol{92}right)\symbol{92}right) \symbol{94}\{2\}\symbol{92}sin \symbol{92}left( 2\symbol{92},b+2\symbol{92},c\symbol{92}, \symbol{92}left( t-\{\symbol{92}it xi\} \symbol{92}right) +2\symbol{92},d\symbol{92},\symbol{92}left( t-\{\symbol{92}it xi\} \symbol{92}right) \symbol{94}\{2\} \symbol{92}right) -2\symbol{92}, \symbol{92}left( \{\symbol{92}it real\} \symbol{92}left(\{\symbol{92}it uS\} \symbol{92}right)  \symbol{92}right) \symbol{94}\{2\}\symbol{92}sin \symbol{92}left( 2\symbol{92},b+2\symbol{92},c\symbol{92}, \symbol{92}left( t-\{\symbol{92}it xi\} \symbol{92}right) +2\symbol{92},d\symbol{92}, \symbol{92}left( t-\{\symbol{92}it xi\} \symbol{92}right) \symbol{94}\{2\} \symbol{92}right) +4\symbol{92},\{\symbol{92}it imag\} \symbol{92}left( \{\symbol{92}it uS\} \symbol{92}right) \{\symbol{92}it real\} \symbol{92}left( \{\symbol{92}it uS\} \symbol{92}right) \symbol{92}cos\symbol{92}left( 2\symbol{92},b+2\symbol{92},c\symbol{92}, \symbol{92}left( t-\{\symbol{92}it xi\} \symbol{92}right) +2\symbol{92},d\symbol{92}, \symbol{92}left( t-\{\symbol{92}it xi\} \symbol{92}right) \symbol{94}\{2\} \symbol{92}right)  \symbol{92}right)\end{maplegroup}
\begin{maplegroup}
\begin{mapleinput}
\mapleinline{active}{2d}{latex(F4); 1}{\[\]}
\end{mapleinput}
\mapleresult
\symbol{92}left( \{\{\symbol{92}it h\symbol{92}\_phi\}\}\symbol{94}\{2\} \symbol{92}left( t-\{\symbol{92}it tau0\} \symbol{92}right) \symbol{94}\{2\} \symbol{92}left( \{\{\symbol{92}rm e\}\symbol{94}\{-1/2\symbol{92},\{\symbol{92}frac \{ \symbol{92}left( t-\{\symbol{92}it tau0\} \symbol{92}right) \symbol{94}\{2\}\}\{\{\{\symbol{92}itsigma\symbol{92}\_phi\}\}\symbol{94}\{2\}\}\}\}\} \symbol{92}right) \symbol{94}\{2\}\{\{\symbol{92}it sigma\symbol{92}\_phi\}\}\symbol{94}\{-4\}+\{\symbol{92}it h\symbol{92}\_phi\}\symbol{92},\{\{\symbol{92}rm e\}\symbol{94}\{-1/2\symbol{92},\{\symbol{92}frac \{ \symbol{92}left( t-\{\symbol{92}it tau0\} \symbol{92}right) \symbol{94}\{2\}\}\{\{\{\symbol{92}itsigma\symbol{92}\_phi\}\}\symbol{94}\{2\}\}\}\}\}\{\{\symbol{92}it sigma\symbol{92}\_phi\}\}\symbol{94}\{-2\}-\{\symbol{92}it h\symbol{92}\_phi\}\symbol{92}, \symbol{92}left( t-\{\symbol{92}it tau0\} \symbol{92}right) \symbol{94}\{2\}\{\{\symbol{92}rm e\}\symbol{94}\{-1/2\symbol{92},\{\symbol{92}frac \{ \symbol{92}left( t-\{\symbol{92}it tau0\}\symbol{92}right) \symbol{94}\{2\}\}\{\{\{\symbol{92}it sigma\symbol{92}\_phi\}\}\symbol{94}\{2\}\}\}\}\}\{\{\symbol{92}it sigma\symbol{92}\_phi\}\}\symbol{94}\{-4\}-\{a\}\symbol{94}\{2\} \symbol{92}left( \{\{\symbol{92}rm e\}\symbol{94}\{-1/2\symbol{92},\{\symbol{92}frac \{ \symbol{92}left( t-\{\symbol{92}it xi\} \symbol{92}right) \symbol{94}\{2\}\}\{\{\symbol{92}sigma\}\symbol{94}\{2\}\}\}\}\} \symbol{92}right) \symbol{94}\{2\} \symbol{92}right) a\{\{\symbol{92}rm e\}\symbol{94}\{-1/2\symbol{92},\{\symbol{92}frac \{ \symbol{92}left(t-\{\symbol{92}it xi\} \symbol{92}right) \symbol{94}\{2\}\}\{\{\symbol{92}sigma\}\symbol{94}\{2\}\}\}\}\} \symbol{92}left( t-\{\symbol{92}it xi\} \symbol{92}right) \symbol{94}\{2\} \symbol{92}left( 2\symbol{92},\{\symbol{92}it real\} \symbol{92}left( \{\symbol{92}it uS\} \symbol{92}right) \symbol{92}cos \symbol{92}left( b+c\symbol{92},\symbol{92}left( t-\{\symbol{92}it xi\} \symbol{92}right) +d\symbol{92}, \symbol{92}left( t-\{\symbol{92}it xi\} \symbol{92}right) \symbol{94}\{2\}\symbol{92}right) -2\symbol{92},\{\symbol{92}it imag\} \symbol{92}left( \{\symbol{92}it uS\} \symbol{92}right) \symbol{92}sin \symbol{92}left( b+c\symbol{92},\symbol{92}left( t-\{\symbol{92}it xi\} \symbol{92}right) +d\symbol{92}, \symbol{92}left( t-\{\symbol{92}it xi\} \symbol{92}right) \symbol{94}\{2\}\symbol{92}right)  \symbol{92}right) -\{a\}\symbol{94}\{2\} \symbol{92}left( \{\{\symbol{92}rm e\}\symbol{94}\{-1/2\symbol{92},\{\symbol{92}frac \{ \symbol{92}left( t-\{\symbol{92}it xi\} \symbol{92}right) \symbol{94}\{2\}\}\{\{\symbol{92}sigma\}\symbol{94}\{2\}\}\}\}\} \symbol{92}right) \symbol{94}\{2\} \symbol{92}left( t-\{\symbol{92}it xi\}\symbol{92}right) \symbol{94}\{2\} \symbol{92}left( 2\symbol{92}, \symbol{92}left( \{\symbol{92}it imag\} \symbol{92}left( \{\symbol{92}it uS\} \symbol{92}right)\symbol{92}right) \symbol{94}\{2\}\symbol{92}sin \symbol{92}left( 2\symbol{92},b+2\symbol{92},c\symbol{92}, \symbol{92}left( t-\{\symbol{92}it xi\} \symbol{92}right) +2\symbol{92},d\symbol{92},\symbol{92}left( t-\{\symbol{92}it xi\} \symbol{92}right) \symbol{94}\{2\} \symbol{92}right) -2\symbol{92}, \symbol{92}left( \{\symbol{92}it real\} \symbol{92}left(\{\symbol{92}it uS\} \symbol{92}right)  \symbol{92}right) \symbol{94}\{2\}\symbol{92}sin \symbol{92}left( 2\symbol{92},b+2\symbol{92},c\symbol{92}, \symbol{92}left( t-\{\symbol{92}it xi\} \symbol{92}right) +2\symbol{92},d\symbol{92}, \symbol{92}left( t-\{\symbol{92}it xi\} \symbol{92}right) \symbol{94}\{2\} \symbol{92}right) +4\symbol{92},\{\symbol{92}it imag
\} \symbol{92}left( \{\symbol{92}it uS\} \symbol{92}right) \{\symbol{92}it real\} \symbol{92}left( \{\symbol{92}it uS\} \symbol{92}right) \symbol{92}cos\symbol{92}left( 2\symbol{92},b+2\symbol{92},c\symbol{92}, \symbol{92}left( t-\{\symbol{92}it xi\} \symbol{92}right) +2\symbol{92},d\symbol{92}, \symbol{92}left( t-\{\symbol{92}it xi\} \symbol{92}right) \symbol{94}\{2\} \symbol{92}right)  \symbol{92}right)\end{maplegroup}
\begin{maplegroup}
\begin{mapleinput}
\mapleinline{active}{2d}{latex(F5); 1}{\[\]}
\end{mapleinput}
\mapleresult
\symbol{92}left( \{\{\symbol{92}it h\symbol{92}\_phi\}\}\symbol{94}\{2\} \symbol{92}left( t-\{\symbol{92}it tau0\} \symbol{92}right) \symbol{94}\{2\} \symbol{92}left( \{
\{\symbol{92}rm e\}\symbol{94}\{-1/2\symbol{92},\{\symbol{92}frac \{ \symbol{92}left( t-\{\symbol{92}it tau0\} \symbol{92}right) \symbol{94}\{2\}\}\{\{\{\symbol{92}itsigma\symbol{92}\_phi\}\}\symbol{94}\{2\}\}\}\}\} \symbol{92}right) \symbol{94}\{2\}\{\{\symbol{92}it sigma\symbol{92}\_phi\}\}\symbol{94}\{-4\}+\{\symbol{92}it h\symbol{92}\_phi\}\symbol{92},\{\{\symbol{92}rm e\}\symbol{94}\{-1/2\symbol{92},\{\symbol{92}frac \{ \symbol{92}left( t-\{\symbol{92}it tau0\} \symbol{92}right) \symbol{94}\{2\}\}\{\{\{\symbol{92}itsigma\symbol{92}\_phi\}\}\symbol{94}\{2\}\}\}\}\}\{\{\symbol{92}it sigma\symbol{92}\_phi\}\}\symbol{94}\{-2\}-\{\symbol{92}it h\symbol{92}\_phi\}\symbol{92}, \symbol{92}left( t-\{\symbol{92}it tau0\} \symbol{92}right) \symbol{94}\{2\}\{\{\symbol{92}rm e\}\symbol{94}\{-1/2\symbol{92},\{\symbol{92}frac \{ \symbol{92}left( t-\{\symbol{92}it tau0\}\symbol{92}right) \symbol{94}\{2\}\}\{\{\{\symbol{92}it sigma\symbol{92}\_phi\}\}\symbol{94}\{2\}\}\}\}\}\{\{\symbol{92}it sigma\symbol{92}\_phi\}\}\symbol{94}\{-4\}-3\symbol{92},\{a\}\symbol{94}\{2\} \symbol{92}left( \{\{\symbol{92}rm e\}\symbol{94}\{-1/2\symbol{92},\{\symbol{92}frac \{ \symbol{92}left( t-\{\symbol{92}it xi\} \symbol{92}right) \symbol{94}\{2\}\}\{\{\symbol{92}sigma\}\symbol{94}\{2\}\}\}\}\} \symbol{92}right) \symbol{94}\{2\} \symbol{92}right) a\{\{\symbol{92}rm e\}\symbol{94}\{-1/2\symbol{92},\{\symbol{92}frac \{\symbol{92}left( t-\{\symbol{92}it xi\} \symbol{92}right) \symbol{94}\{2\}\}\{\{\symbol{92}sigma\}\symbol{94}\{2\}\}\}\}\} \symbol{92}left( t-\{\symbol{92}it xi\}\symbol{92}right) \symbol{94}\{2\} \symbol{92}left( 2\symbol{92},\{\symbol{92}it real\} \symbol{92}left( \{\symbol{92}it uS\} \symbol{92}right) \symbol{92}cos\symbol{92}left( b+c\symbol{92}, \symbol{92}left( t-\{\symbol{92}it xi\} \symbol{92}right) +d\symbol{92}, \symbol{92}left( t-\{\symbol{92}it xi\}\symbol{92}right) \symbol{94}\{2\} \symbol{92}right) +2\symbol{92},\{\symbol{92}it imag\} \symbol{92}left( \{\symbol{92}it uS\} \symbol{92}right) \symbol{92}sin\symbol{92}left( b+c\symbol{92}, \symbol{92}left( t-\{\symbol{92}it xi\} \symbol{92}right) +d\symbol{92}, \symbol{92}left( t-\{\symbol{92}it xi\}\symbol{92}right) \symbol{94}\{2\} \symbol{92}right)  \symbol{92}right) \{\symbol{92}sigma\}\symbol{94}\{-3\}-\{a\}\symbol{94}\{2\} \symbol{92}left( \{\{\symbol{92}rm e\}\symbol{94}\{-1/2\symbol{92},\{\symbol{92}frac \{ \symbol{92}left( t-\{\symbol{92}it xi\} \symbol{92}right) \symbol{94}\{2\}\}\{\{\symbol{92}sigma\}\symbol{94}\{2\}\}\}\}\}\symbol{92}right) \symbol{94}\{2\} \symbol{92}left( t-\{\symbol{92}it xi\} \symbol{92}right) \symbol{94}\{2\} \symbol{92}left( 2\symbol{92}, \symbol{92}left( \{\symbol{92}itreal\} \symbol{92}left( \{\symbol{92}it uS\} \symbol{92}right)  \symbol{92}right) \symbol{94}\{2\}\symbol{92}cos \symbol{92}left( 2\symbol{92},b+2\symbol{92},c\symbol{92},\symbol{92}left( t-\{\symbol{92}it xi\} \symbol{92}right) +2\symbol{92},d\symbol{92}, \symbol{92}left( t-\{\symbol{92}it xi\} \symbol{92}right) \symbol{94}\{2\}\symbol{92}right) -2\symbol{92}, \symbol{92}left( \{\symbol{92}it imag\} \symbol{92}left( \{\symbol{92}it uS\} \symbol{92}right)  \symbol{92}right) \symbol{94}\{2\}\symbol{92}cos \symbol{92}left( 2\symbol{92},b+2\symbol{92},c\symbol{92}, \symbol{92}left( t-\{\symbol{92}it xi\} \symbol{92}right) +2\symbol{92},d\symbol{92}, \symbol{92}left( t-\{\symbol{92}it xi\} \symbol{92}right) \symbol{94}\{2\} \symbol{92}right) +4\symbol{92},\{\symbol{92}it imag\} \symbol{92}left( \{\symbol{92}it uS\} \symbol{92}right) \{\symbol{92}it real\} \symbol{92}left( \{\symbol{92}it uS\} \symbol{92}right) \symbol{92}sin \symbol{92}left( 2\symbol{92},b+2\symbol{92},c\symbol{92}, \symbol{92}left( t-\{\symbol{92}it xi\} \symbol{92}right) +2\symbol{92},d\symbol{92}, \symbol{92}left( t-\{\symbol{92}it xi\} \symbol{92}right) \symbol{94}\{2\} \symbol{92}right)\symbol{92}right) \{\symbol{92}sigma\}\symbol{94}\{-3\}\end{maplegroup}
\begin{maplegroup}
\begin{mapleinput}
\mapleinline{active}{2d}{latex(F6); 1}{\[\]}
\end{mapleinput}
\mapleresult
\symbol{92}left( \{\{\symbol{92}it h\symbol{92}\_phi\}\}\symbol{94}\{2\} \symbol{92}left( t-\{\symbol{92}it tau0\} \symbol{92}right) \symbol{94}\{2\} \symbol{92}left( \{
\{\symbol{92}rm e\}\symbol{94}\{-1/2\symbol{92},\{\symbol{92}frac \{ \symbol{92}left( t-\{\symbol{92}it tau0\} \symbol{92}right) \symbol{94}\{2\}\}\{\{\{\symbol{92}itsigma\symbol{92}\_phi\}\}\symbol{94}\{2\}\}\}\}\} \symbol{92}right) \symbol{94}\{2\}\{\{\symbol{92}it sigma\symbol{92}\_phi\}\}\symbol{94}\{-4\}+\{\symbol{92}it h\symbol{92}\_phi\}\symbol{92},\{\{\symbol{92}rm e\}\symbol{94}\{-1/2\symbol{92},\{\symbol{92}frac \{ \symbol{92}left( t-\{\symbol{92}it tau0\} \symbol{92}right) \symbol{94}\{2\}\}\{\{\{\symbol{92}itsigma\symbol{92}\_phi\}\}\symbol{94}\{2\}\}\}\}\}\{\{\symbol{92}it sigma\symbol{92}\_phi\}\}\symbol{94}\{-2\}-\{\symbol{92}it h\symbol{92}\_phi\}\symbol{92}, \symbol{92}left( t-\{\symbol{92}it tau0\} \symbol{92}right) \symbol{94}\{2\}\{\{\symbol{92}rm e\}\symbol{94}\{-1/2\symbol{92},\{\symbol{92}frac \{ \symbol{92}left( t-\{\symbol{92}it tau0\}\symbol{92}right) \symbol{94}\{2\}\}\{\{\{\symbol{92}it sigma\symbol{92}\_phi\}\}\symbol{94}\{2\}\}\}\}\}\{\{\symbol{92}it sigma\symbol{92}\_phi\}\}\symbol{94}\{-4\}-\{a\}\symbol{94}\{2\} \symbol{92}left( \{\{\symbol{92}rm e\}\symbol{94}\{-1/2\symbol{92},\{\symbol{92}frac \{ \symbol{92}left( t-\{\symbol{92}it xi\} \symbol{92}right) \symbol{94}\{2\}\}\{\{\symbol{92}sigma\}\symbol{94}\{2\}\}\}\}\} \symbol{92}right) \symbol{94}\{2\} \symbol{92}right) a\{\{\symbol{92}rm e\}\symbol{94}\{-1/2\symbol{92},\{\symbol{92}frac \{ \symbol{92}left(t-\{\symbol{92}it xi\} \symbol{92}right) \symbol{94}\{2\}\}\{\{\symbol{92}sigma\}\symbol{94}\{2\}\}\}\}\} \symbol{92}left( -c-2\symbol{92},d\symbol{92}, \symbol{92}left( t-\{\symbol{92}it xi\} \symbol{92}right)  \symbol{92}right)  \symbol{92}left( 2\symbol{92},\{\symbol{92}it real\} \symbol{92}left( \{\symbol{92}it uS\}\symbol{92}right) \symbol{92}cos \symbol{92}left( b+c\symbol{92}, \symbol{92}left( t-\{\symbol{92}it xi\} \symbol{92}right) +d\symbol{92}, \symbol{92}left( t-\{\symbol{92}it xi\} \symbol{92}right) \symbol{94}\{2\} \symbol{92}right) -2\symbol{92},\{\symbol{92}it imag\} \symbol{92}left( \{\symbol{92}it uS\} \symbol{92}right)\symbol{92}sin \symbol{92}left( b+c\symbol{92}, \symbol{92}left( t-\{\symbol{92}it xi\} \symbol{92}right) +d\symbol{92}, \symbol{92}left( t-\{\symbol{92}it xi\}\symbol{92}right) \symbol{94}\{2\} \symbol{92}right)  \symbol{92}right) -\{a\}\symbol{94}\{2\} \symbol{92}left( \{\{\symbol{92}rm e\}\symbol{94}\{-1/2\symbol{92},\{\symbol{92}frac\{ \symbol{92}left( t-\{\symbol{92}it xi\} \symbol{92}right) \symbol{94}\{2\}\}\{\{\symbol{92}sigma\}\symbol{94}\{2\}\}\}\}\} \symbol{92}right) \symbol{94}\{2\}\symbol{92}left( -c-2\symbol{92},d\symbol{92}, \symbol{92}left( t-\{\symbol{92}it xi\} \symbol{92}right)  \symbol{92}right)  \symbol{92}left( 2\symbol{92},\symbol{92}left( \{\symbol{92}it imag\} \symbol{92}left( \{\symbol{92}it uS\} \symbol{92}right)  \symbol{92}right) \symbol{94}\{2\}\symbol{92}sin \symbol{92}left( 2\symbol{92},b+2\symbol{92},c\symbol{92}, \symbol{92}left( t-\{\symbol{92}it xi\} \symbol{92}right) +2\symbol{92},d\symbol{92}, \symbol{92}left( t-\{\symbol{92}it xi\}\symbol{92}right) \symbol{94}\{2\} \symbol{92}right) -2\symbol{92}, \symbol{92}left( \{\symbol{92}it real\} \symbol{92}left( \{\symbol{92}it uS\} \symbol{92}right)\symbol{92}right) \symbol{94}\{2\}\symbol{92}sin \symbol{92}left( 2\symbol{92},b+2\symbol{92},c\symbol{92}, \symbol{92}left( t-\{\symbol{92}it xi\} \symbol{92}right) +2\symbol{92},d\symbol{92},\symbol{92}left( t-\{\symbol{92}it xi\} \symbol{92}right) \symbol{94}\{2\} \symbol{92}right) +4\symbol{92},\{\symbol{92}it imag\} \symbol{92}left( \{\symbol{92}it uS\}\symbol{92}right) \{\symbol{92}it real\} \symbol{92}left( \{\symbol{92}it uS\} \symbol{92}right) \symbol{92}cos \symbol{92}left( 2\symbol{92},b+2\symbol{92},c\symbol{92},\symbol{92}left( t-\{\symbol{92}it xi\} \symbol{92}right) +2\symbol{92},d\symbol{92}, \symbol{92}left( t-\{\symbol{92}it xi\} \symbol{92}right) \symbol{94}\{2\}\symbol{92}right)  \symbol{92}right) + \symbol{92}left( \{\{\symbol{92}it h\symbol{92}\_phi\}\}\symbol{94}\{2\} \symbol{92}left( t-\{\symbol{92}it tau0\}\symbol{92}right) \symbol{94}\{2\} \symbol{92}left( \{\{\symbol{92}rm e\}\symbol{94}\{-1/2\symbol{92},\{\symbol{92}frac \{ \symbol{92}left( t-\{\symbol{92}it tau0\}\symbol{92}right) \symbol{94}\{2\}\}\{\{\{\symbol{92}it sigma\symbol{92}\_phi\}\}\symbol{94}\{2\}\}\}\}\} \symbol{92}right) \symbol{94}\{2\}\{\{\symbol{92}it sigma\symbol{92}\_phi\}\}\symbol{94}\{-4\}+\{\symbol{92}it h\symbol{92}\_phi\}\symbol{92},\{\{\symbol{92}rm e\}\symbol{94}\{-1/2\symbol{92},\{\symbol{92}frac \{ \symbol{92}left( t-\{\symbol{92}it tau0\}\symbol{92}right) \symbol{94}\{2\}\}\{\{\{\symbol{92}it sigma\symbol{92}\_phi\}\}\symbol{94}\{2\}\}\}\}\}\{\{\symbol{92}it sigma\symbol{92}\_phi\}\}\symbol{94}\{-2\}-\{\symbol{92}ith\symbol{92}\_phi\}\symbol{92}, \symbol{92}left( t-\{\symbol{92}it tau0\} \symbol{92}right) \symbol{94}\{2\}\{\{\symbol{92}rm e\}\symbol{94}\{-1/2\symbol{92},\{\symbol{92}frac \{\symbol{92}left( t-\{\symbol{92}it tau0\} \symbol{92}right) \symbol{94}\{2\}\}\{\{\{\symbol{92}it sigma\symbol{92}\_phi\}\}\symbol{94}\{2\}\}\}\}\}\{\{\symbol{92}itsigma\symbol{92}\_phi\}\}\symbol{94}\{-4\}-3\symbol{92},\{a\}\symbol{94}\{2\} \symbol{92}left( \{\{\symbol{92}rm e\}\symbol{94}\{-1/2\symbol{92},\{\symbol{92}frac \{ \symbol{92}left( t-\{\symbol{92}it xi\} \symbol{92}right) \symbol{94}\{2\}\}\{\{\symbol{92}sigma\}\symbol{94}\{2\}\}\}\}\} \symbol{92}right) \symbol{94}\{2\} \symbol{92}right) a\{\{\symbol{92}rm e\}\symbol{94}\{-1/2\symbol{92},\{\symbol{92}frac \{ \symbol{92}left( t-\{\symbol{92}it xi\} \symbol{92}right) \symbol{94}\{2\}\}\{\{\symbol{92}sigma\}\symbol{94}\{2\}\}\}\}\}\symbol{92}left( t-\{\symbol{92}it xi\} \symbol{92}right)  \symbol{92}left( 2\symbol{92},\{\symbol{92}it real\} \symbol{92}left( \{\symbol{92}it uS\}\symbol{92}right) \symbol{92}cos \symbol{92}left( b+c\symbol{92}, \symbol{92}left( t-\{\symbol{92}it xi\} \symbol{92}right) +d\symbol{92}, \symbol{92}left( t-\{\symbol{92}it xi\} \symbol{92}right) \symbol{94}\{2\} \symbol{92}right) +2\symbol{92},\{\symbol{92}it imag\} \symbol{92}left( \{\symbol{92}it uS\} \symbol{92}right)\symbol{92}sin \symbol{92}left( b+c\symbol{92}, \symbol{92}left( t-\{\symbol{92}it xi\} \symbol{92}right) +d\symbol{92}, \symbol{92}left( t-\{\symbol{92}it xi\}\symbol{92}right) \symbol{94}\{2\} \symbol{92}right)  \symbol{92}right) \{\symbol{92}sigma\}\symbol{94}\{-2\}-\{a\}\symbol{94}\{2\} \symbol{92}left( \{\{\symbol{92}rm e\}\symbol{94}\{-1/2\symbol{92},\{\symbol{92}frac \{ \symbol{92}left( t-\{\symbol{92}it xi\} \symbol{92}right) \symbol{94}\{2\}\}\{\{\symbol{92}sigma\}\symbol{94}\{2\}\}\}\}\}\symbol{92}right) \symbol{94}\{2\} \symbol{92}left( t-\{\symbol{92}it xi\} \symbol{92}right)  \symbol{92}left( 2\symbol{92}, \symbol{92}left( \{\symbol{92}it real\}\symbol{92}left( \{\symbol{92}it uS\} \symbol{92}right)  \symbol{92}right) \symbol{94}\{2\}\symbol{92}cos \symbol{92}left( 2\symbol{92},b+2\symbol{92},c\symbol{92}, \symbol{92}left( t-\{\symbol{92}it xi\} \symbol{92}right) +2\symbol{92},d\symbol{92}, \symbol{92}left( t-\{\symbol{92}it xi\} \symbol{92}right) \symbol{94}\{2\} \symbol{92}right) -2\symbol{92},\symbol{92}left( \{\symbol{92}it imag\} \symbol{92}left( \{\symbol{92}it uS\} \symbol{92}right)  \symbol{92}right) \symbol{94}\{2\}\symbol{92}cos \symbol{92}left( 2\symbol{92},b+2\symbol{92},c\symbol{92}, \symbol{92}left( t-\{\symbol{92}it xi\} \symbol{92}right) +2\symbol{92},d\symbol{92}, \symbol{92}left( t-\{\symbol{92}it xi\}\symbol{92}right) \symbol{94}\{2\} \symbol{92}right) +4\symbol{92},\{\symbol{92}it imag\} \symbol{92}left( \{\symbol{92}it uS\} \symbol{92}right) \{\symbol{92}it real\} \symbol{92}left( \{\symbol{92}it uS\} \symbol{92}right) \symbol{92}sin \symbol{92}left( 2\symbol{92},b+2\symbol{92},c\symbol{92}, \symbol{92}left( t-\{\symbol{92}it xi\}\symbol{92}right) +2\symbol{92},d\symbol{92}, \symbol{92}left( t-\{\symbol{92}it xi\} \symbol{92}right) \symbol{94}\{2\} \symbol{92}right)  \symbol{92}right) \{\symbol{92}sigma\}\symbol{94}\{-2\}\end{maplegroup}
\begin{maplegroup}
\begin{mapleinput}
\mapleinline{active}{2d}{}{\[\]}
\end{mapleinput}
\end{maplegroup}
%\end{document}

%\end{landscape}

%\begin{landscape}
%\small{\begin{code}{} 
\vskip18pt\hrule\vskip-10pt\hskip0pt
\begin{lstlisting}
#NCVA for LLE (Temporal Tweezing) - 6-parameter Gaussian Ansatz + Phi
#Ansatz of the form u(z,t) = A(z,t) * exp(I*theta(z,t)) - z is fast time (tau=t is slow time) - 
restart; 
interface(showassumed=0): assume(Ur, real): assume(Ui, real): 
Ustar := Ur+I*Ui; UStarc := Ur-I*Ui; assume(u0, complex);
A := a(z)*exp(-(t-xi(z))^2/(2*(s(z))^2)):   
theta := b(z)+c(z)*(t-xi(z))+d(z)*(t-xi(z))^2;
phi := alpha*exp(-(t-tau0)^2/(2*beta^2));
phidot1 := diff(phi, t);
V:= Delta + phidot1^2 - 2*abs(u0)^2:

#New Lagrangian:
LA := A^2*(diff(theta, z))+(diff(A, t))^2+A^2*(diff(theta, t))^2-(1/2)*A^4+Delta*A^2+phidot1^2*A^2-2*abs(u0)^2*A^2+2*(diff(phi, t))*A^2*(diff(theta, t));
sub1 := diff(a(z),z)=ap,diff(s(z),z)=sp,diff(c(z),z)=cp,diff(xi(z),z)=xip,diff(d(z),z)=dp,diff(b(z),z)=bp:
sub2 := xi(z)=xi,a(z)=a,c(z)=c,s(z)=s,d(z)=d, b(z)=b:
LA2 := subs({sub1,sub2},LA):

#Leff = integral of Lag, we need to assume that a>0 and xi real to be able to evaluate integrals
assume(s>0): assume(xi,real):  assume(alpha, real): assume(beta>0): asume(s,real): assume(d,real): assume(c,real); assume(b, real):
Leff := int(LA2,t=-infinity..infinity):
uAs := subs({sub1,sub2},A*exp(I*theta)):    
uAsC := subs({sub1, sub2}, A*exp(-I*theta));
AA := subs({sub1, sub2}, A); theta2 := subs({sub1, sub2}, theta); phi2 := subs({sub1, sub2}, phi);
phidot := subs({sub1, sub2}, diff(phi, t)); phidot2 := subs({sub1, sub2}, diff(phi, `$`(t, 2)));
ut := subs({sub1, sub2}, diff(A*exp(I*theta), t)); utC := subs({sub1, sub2}, diff(A*exp(-I*theta), t));
duda := diff(uAs, a); dudac := diff(uAsC, a);
dudb := diff(uAs,b): dudbc := diff(uAsC,b):
dudd := diff(uAs, d); duddc := diff(uAsC, d);
dudc := diff(uAs,c): dudcc := diff(uAsC,c):
duds := diff(uAs, s): dudsc := diff(uAsC, s):
dudxi := diff(uAs,xi): dudxic := diff(uAsC,xi):  
pertRa := evalc(int(-I*(1+phidot2)*AA^2/a + I*(1+phidot2)*AA^2/a,t=-infinity..infinity)):
FIa := (phidot^2-phidot2-3*AA^2)*AA*(2*Ur*cos(theta2)+2*Ui*sin(theta2))/a-AA^2*(2*Ur^2*cos(2*theta2)-2*Ui^2*cos(2*theta2)+4*Ui*Ur*sin(2*theta2))/a;
pertRs := evalc(int(-I*(1+phidot2)*AA^2*(t-xi)/s^2+I*(1+phidot2)*AA^2*(t-xi)/s^2, t = -infinity .. infinity));
FIs := (phidot^2-phidot2-3*AA^2)*AA*(t-xi)^2*(2*Ur*cos(theta2)+2*Ui*sin(theta2))/s^3-AA^2*(t-xi)^2*(2*Ur^2*cos(2*theta2)-2*Ui^2*cos(2*theta2)+4*Ui*Ur*sin(2*theta2))/s^3;
pertRc := int(((-I-I*phidot2)*uAs*dudcc + (I+I*phidot2)*uAsC*dudc, t=-infinity..infinity)): 
FIc := (phidot^2-phidot2-AA^2)*AA*(t-xi)*(2*Ur*cos(theta2)-2*Ui*sin(theta2))-AA^2*(t-xi)*(2*Ui^2*sin(2*theta2)-2*Ur^2*sin(2*theta2)+4*Ui*Ur*cos(2*theta2));
pertRd := int(simplify((-I-I*phidot2)*uAs*duddc  + (I+I*phidot2)*uAsC*dudd), t=-infinity..infinity):
FId := (phidot^2-phidot2-AA^2)*AA*(t-xi)^2*(2*Ur*cos(theta2)-2*Ui*sin(theta2))-AA^2*(t-xi)^2*(2*Ui^2*sin(2*theta2)-2*Ur^2*sin(2*theta2)+4*Ui*Ur*cos(2*theta2));
pertRxi := int(((-I-I*phidot2)*uAs*dudxic + (I+I*phidot2)*uAsC*dudxi), t=-infinity..infinity): 
FIxi := (phidot^2-phidot2-AA^2)*AA*(-c-2*d*(t-xi))*(2*Ur*cos(theta2)-2*Ui*sin(theta2))-AA^2*(-c-2*d*(t-xi))*(2*Ui^2*sin(2*theta2)-2*Ur^2*sin(2*theta2)+4*Ui*Ur*cos(2*theta2))+(phidot^2-phidot2-3*AA^2)*AA*(t-xi)*(2*Ur*cos(theta2)+2*Ui*sin(theta2))/s^2-AA^2*(t-xi)*(2*Ur^2*cos(2*theta2)-2*Ui^2*cos(2*theta2)+4*Ui*Ur*sin(2*theta2))/s^2;
pertRb := int((-I-I*phidot2)*uAs*dudbc+(I+I*phidot2)*uAsC*dudb, t = -infinity .. infinity);
FIb := (phidot^2-phidot2-AA^2)*AA*(2*Ur*cos(theta2)-2*Ui*sin(theta2))-AA^2*(2*Ui^2*sin(2*theta2)-2*Ur^2*sin(2*theta2)+4*Ui*Ur*cos(2*theta2)):

#Put back z-dependences so that we can do Euler-Lagrange
subi1 := ap=diff(a(z),z),dp=diff(d(z),z),cp=diff(c(z),z),sp=(diff(s(z),z)),bp=(diff(b(z),z)),xip=(diff(xi(z),z)):
subi2 := b=b(z),s = s(z),c=c(z),xi=xi(z),a=a(z),d=d(z):
sub11 := ap=diff(a(z),z),sp=diff(s(z),z),cp=diff(c(z),z),xip=diff(xi(z),z),dp=diff(d(z),z),bp=diff(b(z),z):
sub22 := d=d(z), b=b(z), xi=xi(z), a=a(z), c=c(z), s=s(z): sub33 := AA = A, theta2 = theta: 
eFIa := subs({subi1, subi2}, FIa); eFIb := subs({sub11, sub22}, FIb); eFIc := subs({sub11, sub22}, FIc); eFId := subs({sub11, sub22}, FId); eFIs := subs({sub11, sub22}, FIs); eFIxi := subs({sub11, sub22}, FIxi);

#Modified Euler-Lagrange equations of motion 
dLsp := subs({subi1, subi2},diff(Leff,sp)):
dLbp := subs({subi1, subi2},diff(Leff,bp)):
dLcp := subs({subi1, subi2},diff(Leff,cp)):
dLdp := subs({subi1, subi2},diff(Leff,dp)):
dLap := subs({subi1, subi2},diff(Leff,ap)):
dLxip:= subs({subi1, subi2},diff(Leff,xip)):
eq1 := diff(dLsp,z) - subs({subi1,subi2},diff(Leff,s)) = subs({subi1,subi2},pertRs+Is):
eq2 := diff(dLbp,z) - subs({subi1,subi2},diff(Leff,b)) = subs({subi1,subi2},pertRb+Ib):
eq3 := diff(dLcp,z) - subs({subi1,subi2},diff(Leff,c)) = subs({subi1,subi2},pertRc+Ic):
eq4 := diff(dLdp,z) - subs({subi1,subi2},diff(Leff,d)) = subs({subi1,subi2},pertRd+Id):
eq5 := diff(dLap,z) - subs({subi1,subi2},diff(Leff,a)) = subs({subi1,subi2},pertRa+Ia):
eq6 := diff(dLxip,z) - subs({subi1,subi2},diff(Leff,xi)) = subs({subi1,subi2},pertRxi+Ixi):

#Sove Euler-Lagrange ODEs simultaneously 
sol:= simplify(solve({eq1,eq2,eq3,eq4,eq5,eq6},{diff(a(z),z),diff(b(z),z),diff(s(z),z),diff(d(z),z),diff(c(z),z),diff(xi(z),z)})):
eqs := diff(s(z),z)=subs(sol,diff(s(z),z)):
eqb := diff(b(z),z)=subs(sol,diff(b(z),z)):
eqc := diff(c(z),z)=subs(sol,diff(c(z),z)):
eqd := diff(d(z),z)=subs(sol,diff(d(z),z)):
eqa := diff(a(z),z)=subs(sol,diff(a(z),z)):
eqxi := diff(xi(z),z)=subs(sol,diff(xi(z),z)):

#Generate Matlab code
subMatlab := a(z) = x(1), b(z) = x(2), c(z)=x(3), d(z) = x(4), s(z)=x(5), xi(z) = x(6), Delta = k, Ui = imag(uS), Ur=real(uS), alpha = h, beta = sigma_X: 
M1 := subs({subMatlab},(eqa)):  M2 := subs({subMatlab},collect(eqb,{a(z),s(z)},recursive)): M3 := subs({subMatlab},(eqc)): M4 := subs({subMatlab},collect(eqd,{a(z),s(z)},recursive)): M5 := subs({subMatlab},(eqs)): M6 := subs({subMatlab},(eqxi)): F1 := subs({subMatlab},(eFIa)): F2 := subs({subMatlab},(eFIb)):  F3 := subs({subMatlab},(eFIc)):  F4 := subs({subMatlab},(eFId)):  F5 := subs({subMatlab},(eFIs)):  F6 := subs({subMatlab},(eFIxi)):
with(CodeGeneration):
Matlab(M1): Matlab(M2): Matlab(M3): Matlab(M4):Matlab(M5): Matlab(M6):
Matlab(F1); Matlab(F2); Matlab(F3); Matlab(F4); Matlab(F5); Matlab(F6)

#Generate LaTeX code
subLatex := a(z) = a, b(z) = b, c(z)=c, d(z) = d, s(z)=sigma, xi(z) = xi, Ui = imag(uS), Ur=real(uS), alpha = h_phi, beta = sigma_phi: 
F1 := subs({subLatex}, eFIa); F2 := subs({subLatex}, eFIb); F3 := subs({subLatex}, eFIc); F4 := subs({subLatex}, eFId); F5 := subs({subLatex}, eFIs); F6 := subs({subLatex}, eFIxi);
latex(F1); latex(F2); latex(F3); latex(F4); latex(F5); latex(F6);
\end{lstlisting}\vskip-3pt\hrule\vskip18pt}

