\documentstyle[12pt]{article}
\renewcommand\floatpagefraction{0.9}
\renewcommand{\baselinestretch}{1.}
\topmargin 0.0in
\newcommand{\marg}[1]{\marginpar{\raggedright\scriptsize#1}}
\parskip=10pt
\setlength{\textwidth}{6.5in}
\setlength{\textheight}{8.5in}
\setlength{\evensidemargin}{0.0in}
\setlength{\oddsidemargin}{0.0in}
\renewcommand{\abstractname}{\bf ABSTRACT}


\begin{document}
\pagestyle{empty}


\begin{center}
{\large \bf
\centerline{Non-Conservative Variational Approximation}  
\smallskip 
\centerline{for Nonlinear Schr\"{o}dinger Equations and its Applications}
}
\end{center}

\vskip0.3in
\begin{center}
\bf Julia Rossi  \\
Dissertation Advisor: {\bf Prof. Ricardo Carretero} \\
\end{center}

\bigskip
\centerline {{\bf DISSERTATION DEFENSE ABSTRACT}}
\bigskip

\noindent Galley~[Phys.~Rev.~Lett.~{\bf 110}, 174301 (2013)] proposed an initial value problem formulation of Hamilton's principle applied to non-conservative ordinary differential systems.  We explore this formulation for complex partial differential equations of the nonlinear Schr\"{o}dinger (NLS) type, using the non-conservative variational approximation (NCVA) outlined by Galley.  We showcase the relevance of the NCVA method by exploring non-conservative systems such as exciton-polariton condensates.  We also study a variant of the NLS used in optical systems called the Lugiato-Lefever (LL) model applied to (i) spontaneous temporal symmetry breaking instability in a coherently-driven optical Kerr resonator and (ii) temporal tweezing of cavity solitons in a passive loop of optical fiber pumped by a continuous-wave laser beam.  For application (ii) we analyze various tweezers and their effectiveness in temporal tweezing cavity solitons which gives beneficial insights for the design of tweezers in optical information processing.


%We showcase the relevance of the NCVA method by exploring test case examples within the NLS setting and an example applied to exciton-polariton condensates that intrinsically feature loss and a spatially dependent gain term.  We also study a variant of the NLS used in optical systems called the Lugiato-Lefever (LL) model applied to (i) spontaneous temporal symmetry breaking instability in a coherently-driven optical Kerr resonator observed experimentally by Xu and Coen in Opt.~Lett.~{\bf 39}, 3492 (2014) and (ii) temporal tweezing of cavity solitons in a passive loop of optical fiber pumped by a continuous-wave laser beam observed experimentally by Jang, Erkintalo, Coen, and Murdoch in Nat.~Commun.~{\bf 6}, 7370 (2015).  For application (i) 
%
%Recently, Galley~[Phys.~Rev.~Lett.~{\bf 110}, 174301 (2013)] proposed an initial value problem formulation of Hamilton's principle applied to non-conservative systems.  Here, .  We compare the formalism of the NCVA to two variational techniques used in dissipative systems; namely, the perturbed variational approximation and a generalization of the so-called Kantorovitch method.  
%
%\noindent For application (i) we perform a detailed stability analysis and analyze the temporal bifurcation structure of stationary symmetric configurations and the emerging asymmetric states as a function of the pump power. For intermediate pump powers a pitchfork loop is responsible for the destabilization of symmetric states towards stationary asymmetric ones while at large pump powers we find the emergence of periodic asymmetric solutions via a Hopf bifurcation. 
%%From a theoretical perspective, we use Galerkin projections in order to analyze the most unstable eigenmode of the system.  
%%We also explore a NCVA capturing, among others, the evolution of the solution's amplitude, width and center of mass. Both methods provide insight towards the pitchfork bifurcations associated with the symmetry breaking.  
%For application (ii) we study the existence and dynamics of cavity solitons through phase-modulation of the holding beam.  We find parametric regions for the manipulation of cavity solitons by a tweezer in the LL model.  For both applications we also explore the ability of the NCVA method at capturing the evolution of solitary waves.


\end{document}
